% vim: tw=80
\fontfam[lm]
\input ctustyle3
\load[vlna]
\singlechars{Czech}{AaIiVvOoUuSsZzKk}
\cslang
\csquotes
\verbchar`
\picdir={figures/}
\nonstopmode

\newcount\tnotenum
\def\tnotelist{}
\def\tnote#1{\incr\tnotenum $^{\rm\_romannumeral\tnotenum}$\global\addto\tnotelist{{#1}}}
\def\tnoteprint{\par \tnotenum=0
   \ea\foreach\tnotelist
     \do{\advance\tnotenum by1 \par $^{\rm\_romannumeral\tnotenum}$##1 }\par
   \global\tnotenum=0 \gdef\tnotelist{}%
}

\def\pdfstdcite[#1]{[\rcite[iso32000-1], část~#1]}

\newdimen\halfhsize \halfhsize=\dimexpr\hsize/2-2pt\relax

\hyphenation{hon-or mark-up}

\_def\_urlskip{\_null\_nobreak\_hskip0pt plus0.1em\_relax}
\_def\_urlbskip{\_penalty50 \_hskip0pt plus0.1em\_relax}

\bibtexhook={
\_sdef{_print:misc}{%
   \_bprintb [!author]    {\_doauthor1{##1}\.\ }{\_bibwarning}%
   \_bprintb [title]      {{\_em##1}\_bprintc\_titlepost{\.\ *}\_bprintv[howpublished]{}{\.}\ }%
                                                                                     {\_bibwarning}%
   \_bprinta [howpublished]  {[*].\ }{}%
  %\_bprinta [ednote]     {\_prepareednote*\_bprintv[citedate]{}{.}\ }{\_bibwarning}%
   \_bprinta [ednote]     {\_prepareednote*\_bprintv[citedate]{}{.}\ }{}%
   \_bprintb [year]       {\_doyear{##1}\_bprintv[citedate]{}{.}\ }{\_bibwarninga}%
  %\_bprintb [year]       {\_doyear{##1}\_bprintv[citedate]{.}{.}\ }{\_bibwarninga}%
   \_bprinta [citedate]   {\_docitedate*///\_relax.\ }{}%
   \_bprintb [doi]        {\_predoi DOI \_ulink[http://dx.doi.org/##1]{##1}.\ }{}%
   \_bprintb [url]        {\_preurl\_url{##1}. }{}%
}
}

\def\optparams{\adef<##1>{\hbox{$\langle$\it##1\/$\rangle$}}}
\toksapp\everyintt{\optparams}
\toksapp\everytt{\typosize[9.5/12.4]}

\worktype [M/EN]

\faculty{F8}

\department{Katedra teoretické informatiky}
\title  {x86-64 nativní backend pro TinyC}
\titleEN{x86-64 native backend for TinyC}
\author{Michal Vlasák}
\date{XX.\,XX.\,TODO}
\supervisor{Ing. Petr Máj}
\abstractEN {
}

\abstractCZ {
}

\keywordsEN {%
TODO
}
\keywordsCZ {%
TODO
}
\thanks {% Use main language here
}

\declaration {
Prohlašuji, že jsem předloženou práci vypracoval samostatně a že jsem uvedl
veškeré použité informační zdroje v~souladu s~Metodickým pokynem o~dodržování
etických principů při přípravě vysokoškolských závěrečných prací.

Beru na vědomí, že se na moji práci vztahují práva a povinnosti vyplývající ze
zákona č.\,121/2000~Sb., autorského zákona, ve znění pozdějších předpisů.
V~souladu s~ust.\,§\,2373 odst.\,2 zákona č.\,89/2012~Sb., občanský zákoník, ve
znění pozdějších předpisů, tímto uděluji nevýhradní oprávnění (licenci) k~užití
této mojí práce, a to včetně všech počítačových programů, jež jsou její součástí
či přílohou a veškeré jejich dokumentace (dále souhrnně jen „Dílo“), a to všem
osobám, které si přejí Dílo užít. Tyto osoby jsou oprávněny Dílo užít jakýmkoli
způsobem, který nesnižuje hodnotu Díla a za jakýmkoli účelem (včetně užití
k~výdělečným účelům). Toto oprávnění je časově, teritoriálně i množstevně
neomezené.

V XX dne XX.\,XX.\,TODO
\signature
}

\draft
{\nopagenumbers
  {\pgbackground={
    \picwidth=\pagewidth \picheight=\pageheight
    \inspic{vlasami6-assignment.pdf}}
    \null\vfil\break}
  \null\vfil\break}
\makefront

\chap Introduction

\label[ragalloc]
\chap Register allocation

This chapter describes the last of the three big conceptual parts of a usual
compiler backend---the register allocation phase. Motivation, importance and
possible approaches are introduced.

The x86-64 architecture (see \cite[ref:x86]) is what we mainly care about in
this thesis. Since it is familiar, we will be using it as examples in the
following sections. It is also a good candidate because it brings some
challanges not found on other architectures, but shows the general problems just
as well as other architectures.

Note that like with instruction selection although we are already working with
target specific instructions, their form doesn't necessarily have to be target
specific. Target independent representation of target specific instructions
allows us to share also register allocation logic for all targets.

\sec Motivation

Most CPU architectures these days are register based. That means that interface
of the CPU consists of a fixed number of registers and instructions that allow
operations on these registers. For example registers may be 8 32-bit storage slots
and instructions may allow performing arithmetic on these registers or allow
loading/storing contents of register from/into memory. Memory is still an
important part of these architectures---computations can't possibly fit all into
a fixed number of registers of fixed size, it is the memory that allows us to
store large amounts of data.

During previous phases of the compiler we used a powerful abstraction, we
pretended that there is an infinite amount of registers. This is very important
for the middle end IR, since it is supposed to be platform agnostic and rather
than limiting to some fixed number of registers (per architecture or wholesale),
we might as well pretend to have infinite amount of them. But once we start
translating the middle end IR we just need to limit ourselves to fixed amount of
registers somewhere. And not only that architectures these days usually have at
least two classes of registers---general purpose registers and floating point
registers...

After instruction selection (which tells us what instructions to use) and
instruction scheduling (which refines the order these instructions are executed
in), a snippet of input to register allocation can look as follows:

\begtt
mov t1, 1
mov t2, 2
mov t3, t1
add t3, t2
\endtt

There are several things of note here. The instructions don't operate on real
machine registers (like `rax`), but on "virtual registers" or "temporaries". It
is the goal of register allocation to transform the code so that real registers
are used. The snippet corresponds to this middle end IR:

\begtt
v1 = add 1, 2
\endtt

The translation given above is suboptimal and in a reasonable compiler such code
wouldn't get as far as to the register allocator: the middle end could fold the
addition of constants into a constant, or the instruction selector could take at
least take advantage of "register plus immediate" instruction for addition.
Nonetheless the example serves us well for showing how a simple allocation of
registers might look like. Since the whole program doesn't use more than 16
registers, we have no problem assigning x86-64 registers directly, for example
in the order of the temporaries:

\begtt
mov rax, 1   // t1 = rax
mov rbx, 2   // t2 = rbx
mov rcx, rax // t3 = rcx
add rcx, rbx
\endtt

Even for such a simple example, we can notice several things about register
allocation alone:

\begitems
 * We introduce a third register `rcx` to store the result of addition. This
works well and fits into the 16 registers we have available. But we can notice
that after the addition we no longer need the value stored in register `rax`.
Thes is the core idea of register allocation, we only need to store such values
that will be needed in the future, and contraty to SSA we can use that to
"recycle" registers.

* If we were to reuse `rax` for storing the result of addition, our situation
would look like this:

\begtt
mov rax, 1   // t1 = rax
mov rbx, 2   // t2 = rbx
mov rax, rax // t3 = rax
add rax, rbx
\endtt

Move (copy) of a register to itself is a no op, the instructions doesn't have
any real effect (other than settings flags TODO). But at least in this case it
is safely possible to remove that instruction. We can notice that the "two
address code" generated from SSA "three address code" can be improved if it
turns out that the destination can be the same register as the first source (or
the second source in this case, since addition is commutative).

But a question remains, can the register allocator remove the instruction?
Or should it even? + Spill insertion.
\enditems

As we have seen, opportunities often arise for register reuse. Then what are the
situations where we can get out of registers? Well, when we can't get away with
reuse. We should consider the difference between (naive, but illustrative)
compilation of expression `1 + 2 + 3 + 4` and its possible register allocation:

\begtt
mov t1, 1  // t1 = rax
mov t2, 2  // t2 = rbx
mov t3, t1 // t3 = rax
add t3, t2

mov t4, 3  // t4 = rbx
mov t5, t3 // t5 = rax
add t5, t4

mov t6, 4  // t6 = rbx
mov t7, t3 // t7 = rax
add t7, t5
\endtt

And as opposed to that the "right associative" version: `1 + (2 + (3 + 4))`

\begtt
mov t1, 1  // t1 = rax
mov t2, 2  // t2 = rbx
mov t3, 3  // t3 = rcx
mov t4, 4  // t4 = rdx

mov t5, t3 // t5 = rcx
add t5, t4

mov t6, t2 // t6 = rbx
add t6, t5

mov t7, t1 // t7 = rax
add t7, t6
\endtt

Even though both versions use the same amount of {\em virtual registers}, their
opportunities to reuse registers and hence use of {\em physical registers}
differs considerably. While the example is trivial and the right associative
version could be transformed into the left associative version the highlighted
issue here is, that even single benign expressions can use a surprising number of
registers. Just extending the addition to more than 16 addends would leave us
out of registers. So not only for variables (which for example in C nominally
reside in memory), but even for temporaries we have to account for the fact that
we simply can get out of registers.

\sec Spilling

We can get out of registers, but we still have to store our (say intermediate)
values {\em somewhere}. The solution is to put the values into memory. Resorting
to putting the values to memory is called \"spilling".
The system stack (\"frame stack", \"call stack") is best suited for spilled
values, much for the same reasons it is the best place for storing local
variables: recursion is handled naturally and well, since each {\em invocation}
of a function gets its own locations for storing the values---by storing the
values on the stack, we allow functions to be {\em reentrant}.

Architectures usually also have a dedicated register for the pointer to \"the
top of the stack", which means that code needing to access the values not
fitting into registers can address their memory locations using relative
addressing with small offsets (say smaller than 16 bits), also a feature
efficiently supported by all common architectures these days. On the other hand
accessing static slots of memory would pose similar challanges as global
variables have - they may not be reachable with small relative offsets from any
register and especially on 64 bits systems their absolute address is not only too large
to be stored in literals in code, but using absolute addresses is also
discouraged, because it means that the code becomes {\em position
dependent}---change of the absolute address through a change in position in
address space, would mean that our code would no longer be correct. Such
absolute references require run time fix ups (called \"relocations") by the
dynamic linker, which prevents reuse of the code between multiple processes.

\label[sec:use-of-spilled]
\secc Using spilled values

As established, we will refer to spilled values through relative addressing
(using either the stack pointer or the \"base" pointer, also often called the
\"frame pointer", see TODO). But we have to get back to the problem at
hand---we store values in registers, because we want to use the values in
computations and processors (mostly) operate on registers. But now that we can't
fit all values into registers and want to store them in memory, how do we use
the values from memory and how do they even get there?

RISC processors usually employ a \"load-store" architecture. There are only a
few dedicated instructions for reading values from memory into registers (`load`
instructions) and writing register contents into memory (`store` instructions).
All other instructions operate only with registers. This means that to even have
a value to spill, it has to be in a register! Also only from that register, we
can store the value to memory and later retrieve it, but again only into a
register. The important observation here is, that resorting to storing the
values in memory doesn't in general mean that we get around of doing register
allocation. What we gain is that instead of having to store a value in a
register in a long part of the code (because it is needed in all parts of that
code), by putting the value into memory and retrieving it immediately before
each use, we have made the register allocation problem simpler---a register is
needed in more but smaller parts of the code. This means that a single machine
register can be reused for several valuse, instead of being blocked by "one".

At least the register used for the store, doesn't have to be the same one used
for the load. So even though we are constrained by the fact that registers have
to be used for spills, we are not too constrained in choosing the registers for
performing spills. Loads and stores inserted for handling spills are usually
called the {\em spill code}.

In this text, we care especially about the x86-64 architecture, which, like most
CISC architectures, doesn't have a load-store architecture. There are
instructions which can e.g. perform arithmetic directly on locations in memory.
Though at most one operand of an instruction can be a location in memory, the
other one has to be either a register or an immediate value. So on one hand, the
problem of having to use {\em registers} for storing/retrieving
spilled values reamins, but on the other hand, since one operand can be a
location in memory, we can take advantage of that and not use an intermediate
register at all.

An interesting perspective is, that even though the x86-64 architecture allows
the instructions to operate on memory operands, an implementation of the
architecture in for example some new Intel processors splits these instructions
into microoperations based on the load-store architecture, using some internal
\"architectural" registers (not normally accessible directly) for storing the
intermediate values. Because of this there probably isn't any {\em direct}
performance difference of using the memory operands. But it is still very useful
to use these more complex instructions---not only can the instruction encoding
be shorter (thus sparing the instruction cache of the processor), but we take
advantage of the internal architectural registers, that we normally wouldn't
have access to, which can mean less constraints on the use of the ordinary
\"general purpose" registers that we have access to, which may allow storing
more values into registers instead of memory and hence have a significant {\em
indirect impact} on performance.

For example, let's say that `t4` needs to be spilled in the following:

\begtt
add t3, t2
\endtt

Instead of adding a load before the instruction of `t4` and a store of `t4`
after, like this:

\begtt
mov t4, [rbp+t3]
add t4, t2
mov [rbp+t3], t4
\endtt

We can just operate on the memory location:

\begtt
add [rbp+t3], t4
\endtt

This example also showed, that as mentioned, at least with a very narrow local
view, and with the first simpler solution, spilling `t3` doesn't help with the
use of registers! We needed to introduce another pseudoregister, `t4`, to hold
the value of `t3` loaded from memory. Indeed, spilling helps only in a broader
scope, where splitting definitions / uses of one pseudoregister into single
definitions and singe uses of many different pseudoregisters help making the
register allocation solvable (due to limited number of registers) or easier.

Thus for spills, we want to introduce a new temporary for each new load and
store, but that is not the case in the example, `t4` is used also for storing
spilled `t3` back to memory. This is needed, since as we have discussed in TODO,
most instructions on the x86-64 use the "two address code", where one of the
operands is also the destination for the result. So it is not directly possible
to store the result in a different location than the location of the first
operand. Judging the snippet only locally, the transformation is fine, since the
`original` value of `t3` was also \"destroyed" in the original version. In
general, the original value of `t3`, before the addition, {\em might} be needed
even after the addition. That is why the code generator might need to output the
following sequence to add `t1` and `t2` into `t3` while preserving all of `t1`,
`t2` and `t3`:

\begtt
mov t3, t1
add t3, t2
\endtt

Now the prospect of naively spilling `t3` seems even worse:

\begtt
mov [rbp+t3], t1
mov t4, [rbp+t3]
add t4, t2
mov [rbp+t3], t4
\endtt

The code generator hoped that by assigning `t1` and `t3` the same register, the
move instruction could be eliminitaed. Since for some reason `t3` was spilled,
that is now out of the question, but there is still a suboptimality - `t1` is
copied to memory and then immediately loaded back again into `t4`, because it
needs to be used in the `add` instruction and stored back into memory. Use of
`t3`'s memory location as the first operand of the `add` instruction helps as
before:

\begtt
mov [rbp+t3], t1
add [rbp+t3], t2
\endtt

But in some sense, this is a red herring---the memory operations are still
there, the CPU still has to load the value `t3` from memory in the `add`
instruction, when we just had it in a register. Just because the addressing
modes of x86 allow use of memory locations in the instruction encodings, doesn't
mean that internally the ALU (Arithmetical logical unit) can suddenly operate
directly on memory, the CPU still has to internally load the value into a
register. So while we spare a general purpose register and instead use an
architectural one, we still excessively operate on memory. Alternatively instead of
using memory location for the `add` instruction we could move the value from
`t1` directly into memory, because store/load pair didn't have any meaning at
all---we store the result of addition to `t3` in the last instruction afterall.

\begtt
mov t4, t1
add t4, t2
mov [rbp+t3], t4
\endtt

Now, we keep the values in registers the whole time, and only store the final
result in memory on assumption that later the value is needed and storing it in
memory somehow helps the register allocator. While the first example is two
instructions, it is X bytes, while the second one is 3 instructions and Y bytes
TODO: compare lengths of instructions. But in fact this is how we started in the
first place---just have the values in the registers. The important difference
is, that the register `t4` introduced by spilling, is a temporary, it's value
ends right at the store to `t3`'s location in memory. Thus it makes the register
allocation problem much easier than previously just with `t3`. For example now
it is even more likely that `t4` and `t1` are assigned the same register, thus
making the first instruction a noop. The reason why it is easier to merge with
`t4` instead of `t3` is not really apparent from this local view, but we have
TODO less interference.

Another point of view on the issue is, that essentially, in (the last snippet)
in the first two instructions `t3` is represented by `t4`, while in the last
instruction it shifts from being represented by `t4` to be represented by `t3`,
which due to some previous decision, resides in a memory location. We have
essentially split the register `t3` into multiple registers connected by moves
and it improved the generated code. In contrast, we argued, that by merging `t4`
and `t1` we would have had the chance to eliminate the move instruction, thus
improving the code as well. Both "merging" (also called {\em coalescing}), and
\"splitting" can both improve the code in different situations, this makes it
even harder, because recklessly doing one or the other will make the code
certainly worse, while doing neither may as adding these contrasting notions
makes great register allocation an even harder problem.

\secc Interaction with instruction selection

As we have seen, the process of spilling needs to insert new instructions which
together form the so colled "spill code". In simplest scenario they just load
and store the value, but better code can be achieved if the spill code is
inserted with better thought than just load from memory and move to memory. But
both of these problems---selecting the instructions to use for operations and
additionally making nontrivial transformations based on the code at hand was
exactly the job of instruction selection.

In principle, register allocator shouldn't care about the instructions. It
should only cares about their effects on the registers. The result of register
allocation should be the assignment of register to pseudoregisters. Since spills
can be necessary, which requires insertion of new instructions, we have to
decide how this spill code will be handled. The basic options are the following,
and mostly depend on the chosen register allocation technique:

\begitems
* Insert spill code in the register allocator pass.
* Return the list of spilled pseudoregisters, and expect to be called again with
code transformed to include the spill code.
\enditems

The first option can make the register allocator depend on the target
architecture---it now needs to know about the current target, its instructions,
their meanings and how to insert them. On the other hand, register allocation is
already in the "backend", where we expect to handle things on the level of the
target, and generally hope to take advantage of that by, for example, doing
optimizations specific to the particular target. This can be application of that
idea. On the other hand, as mentioned previously, machine independent
representations of machine depednent instructions are possible, thus spill code
insertion {\em could} also be made machine independent similarly as instruction
selection.

The second makes the register allocation process pure in a sense. The register
allocator never modifies the input, its result is a either a mapping of
pseudoregisters to registers, or a list of pseudoregisters to be spilled.
Though it still has to be decided on how to insert the instructions. Some
instruction selection mechanisms which ultimately depend on the middle end IR,
are not suitable for inserting and optimizing spill code, since we are already
in the "low level" backend IR. Tree or DAG based instruction selection
mechanisms may also not be directly applicable---we may no longer be using
trees or DAGs for representing the machine code, especially efter instruction
scheduling, which sets the order of instructions in stone. On the other hand
peephole optimization is a great fit for improving inserted spill code. The
inserted spill code can be very naive, and peephole optimization run on the
spill code and its surroundings can make improvements. This is especially likely
if we are able to find patterns which spilled code creates, such as in the
example in the previous subsection (\ref[sec:use-of-spilled]).

The obvios downside of the \"pure register allocation" approach is, that it has
to start over with register allocation, if any spills need to be made. The first
approach seems more suited to register allocation where we would want self
contained and final results in possibly just one pass. The approaches used for
spill code insertion are usually very connected to the core principle of the
register allocation algorithm at hande, and choices of some approaches are
discussed in the section~\ref[sec:regalloc-techniques].

\sec Formalization

The terms used in the previous sections about register allocation were not
properly defined, and perhaps even not commonly used for the concepts. The
intention was to practically show the problems register allocation tries to
solve and what needs to do to solve it, which of course includes optimizations
that try to make the register allocation process more optimal.

One thing that has to be noted is the name \"register allocation" itself. We
infact mentioned that register allocation is meant to map pseudoregisters to
real machine registers of the target architecture, but the process isn't always
so direct, and sometimes it makes sense and brings benefits to split this
process into two parts, which really define what we mean by these names:

\begitems\style n
* {\em Register allocation}. In a narrower sense, by register allocation we mean
the process of making sure that each pseudoregister can be assigned a register.
At this point, we may not care too much about which one, but we care about spill
code, because that is what allows us to fit into the limited amount of registers
available.
* {\em Register assignment}. The assignments follows allocation---now that we
can map every pseudoregister left into at least one register, we choose the
concrete one. Although this seems much more simpler than the allocation part,
in practice register assignment is also very important, because we have seen
situations where some assignments lead to better code, for example where the
source and destination of a move instruction are assigned the same register, the
move instruction can be eliminated.
\enditems

Some register allocation algorithms intertwine both parts and don't split them.
Some algorithms strictly separate these concerns. In general simpler algorithms
usually merge both of these parts, while more complex algorithms try to take
advantage of the attacking each of those issues separately to reach more optimal
results. But this distinction is of course not definitive.

\secc Liveness, inteference

\secc Live ranges

vs value vs variable.

\secc Coalescing

\secc Live range splitting

\label[sec:regalloc-techniques]
\sec Techniques

We have already seen a few things that can distinguish different register
allocation algorithms:

\begitems
* Handling of spilling (section~\ref[ref:use-of-spilled]),
* Split or no split of allocation and assignment
(section~\ref[ref:regalloc-formalization]).
\enditems

\noindent But there are others:

\begitems
* Scope: {\em local} vs {\em global} vs {\em interprocedural} vs {\em whole
program} algorithms. Local algorithms operate on singular basic blocks and use
only information local to the basic block to decide on register allocation. The
limited scope makes the algorithms generally simpler and produces worse results
then global register allocation, which allocates registers to whole functions.
Global register allocation is global in the sense that {\em all} basic blocks
are considered at the same time. The analysis is more complex, since it has to
handle control flow. Even techniques for allocating registers across function
calls and whole programs exist. These can be less practical in practice, where
functions may be required to conform to a {\em calling convention}, which
specifies how arguments should be passed in function calls, what registers are
preserved by calls and where will the return values reside. We will not discuss
techniques operating in larger scopes than {\em global} (whole function, all
basic blocks).

* {\em Quality} vs {\em speed}. With no restrictions on time, we ideally would
like to achieve {\em optimal} register allocation. With the right definition of
optimal, it can be possible, but due to the difficulty of the register
allocation, this approach is bound to be too slow (in {\em compile-time}),
although it would produce code that would be fast (in {\em run-time}). In code
compiled ahead of time, we can probably justify spending more time on
compilation to achieve better run-time, since it is expected, that the program
will run for some time and that the investment will return.

On the other end of the spectre we may want a register allocation algorithm that
runs very fast (due to constraints on compile-time), but in that case we can't
expect good results (i.e. code that has fast run-time). This can be interesting
for {\em Just-in-time} (JIT) compilers, where the compile time is part of
run-time and hence it is not possible to spend much time on optimizations,
because 

Most of the code compiled ahead
of time can usually benefit from
\enditems

\secc Top-down, bottom-up

\secc Backward

\secc Linear scan

\secc Graph coloring

\secc Reduction (to another NP-complete problem and using a solver)

ILP, IBQP

\secc SSA register allocation


\bibchap
\usebib/s (iso690) vlasami6-dip

\bye

This sequence simulates \"three address code" (see TODO) and the second instruction
is in fact what we started with to show use of spilled pseudoregisters.


As we have discussed in TODO, most instructions on
the x86-64 use the "two address code", where one of the operands is also the
destination for the result. Generally, three address code (where operations have
two operands and one destination that may or may not coincide with one of the
operadns), are nicer from the perspective of the compiler---it is (at least in
principle, or while we are operating with pseudoregisters) non-destructive.





Coalescing vs live range splitting.



Zkratky:

DAG
JIT
