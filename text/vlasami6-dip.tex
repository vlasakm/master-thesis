% vim: tw=80 spell spelllang=en

\fontfam[lm]
\input ctustyle3
%\load[vlna]
%\singlechars{Czech}{AaIiVvOoUuSsZzKk}
\enuslang
\enquotes
\verbchar`
\picdir={figures/}
\nonstopmode
\nonumcitations

\draft

\newcount \_secccnum
\def \_thesecccnum {\_othe\_chapnum.\_the\_secnum.\_the\_seccnum.\_the\_secccnum}

\_optdef\_seccc[]{\_trylabel \_scantoeol\_inseccc}

\_def\_inseccc #1{\_par \_sectionlevel=4
   \_def \_savedtitle {#1}% saved to .ref file
   \_ifnonum \_else {\_globaldefs=1 \_incr\_secccnum \_secccx}\_fi
   \_edef \_therefnum {\_ifnonum \_space \_else \_thesecccnum \_fi}%
   \_printseccc{\_scantextokens{#1}}%
   \_resetnonumnotoc
}
\public \seccc ;

\_def \_chapx {\_secx   \_secnum=0   \_lfnotenum=0 }
\_def \_secx  {\_seccx  \_seccnum=0  \_tnum=0 \_fnum=0 \_dnum=0 \_resetABCDE }
\_def \_seccx {\_secccx \_secccnum=0 }
\_def \_secccx {}


\_def\_printseccc#1{\_par
   \_abovetitle{\_penalty-100}{\_medskip}
   {\_bf \_noindent \_raggedright \_printrefnum[@\_quad]#1\_nbpar}%
   \_nobreak \_belowtitle{\_smallskip}%
   \_firstnoindent
}

\_def\secccfont{\_scalemain\_typoscale[\_magstephalf/\_magstephalf]\_ctustyle_boldify}
\_def\_printseccc#1{\_par \_abovetitle{\_goodbreak} 
  \_smallskip\_medskip\_vskip.5\_parskip
  \_ctustyle_begitemstest\seccc
  \_line{%\Blue\_vrule height 3.5mm width4mm depth.3mm\Black 
  \_hss\_vtop{\_advance\_hsize by-0mm
     \secccfont \_noindent \_printrefnum[@\_quad]%
     \_ctustyle_nBlue#1\_rightskip=0pt plus1fil \_strut\_nbpar\_kern-4.5pt}}%
  \_nobreak\_vskip-.5\_parskip\_smallskip\_vskip2pt\_relax
  \_firstnoindent
}


%\_sdef{_tocl:4}#1#2#3{\_advance\_leftskip by2\_iindent \_cs{_tocl:2}{#1}{#2}{#3}}

\newcount\tnotenum
\def\tnotelist{}
\def\tnote#1{\incr\tnotenum $^{\rm\_romannumeral\tnotenum}$\global\addto\tnotelist{{#1}}}
\def\tnoteprint{\par \tnotenum=0
   \ea\foreach\tnotelist
     \do{\advance\tnotenum by1 \par $^{\rm\_romannumeral\tnotenum}$##1 }\par
   \global\tnotenum=0 \gdef\tnotelist{}%
}

\def\pdfstdcite[#1]{[\rcite[iso32000-1], část~#1]}

\newdimen\halfhsize \halfhsize=\dimexpr\hsize/2-2pt\relax

\hyphenation{hon-or mark-up}

\_def\_urlskip{\_null\_nobreak\_hskip0pt plus0.1em\_relax}
\_def\_urlbskip{\_penalty50 \_hskip0pt plus0.1em\_relax}

\bibtexhook={
\_sdef{_print:misc}{%
   \_bprintb [!author]    {\_doauthor1{##1}\.\ }{\_bibwarning}%
   \_bprintb [title]      {{\_em##1}\_bprintc\_titlepost{\.\ *}\_bprintv[howpublished]{}{\.}\ }%
                                                                                     {\_bibwarning}%
   \_bprinta [howpublished]  {[*].\ }{}%
  %\_bprinta [ednote]     {\_prepareednote*\_bprintv[citedate]{}{.}\ }{\_bibwarning}%
   \_bprinta [ednote]     {\_prepareednote*\_bprintv[citedate]{}{.}\ }{}%
   \_bprintb [year]       {\_doyear{##1}\_bprintv[citedate]{}{.}\ }{\_bibwarninga}%
  %\_bprintb [year]       {\_doyear{##1}\_bprintv[citedate]{.}{.}\ }{\_bibwarninga}%
   \_bprinta [citedate]   {\_docitedate*///\_relax.\ }{}%
   \_bprintb [doi]        {\_predoi DOI \_ulink[http://dx.doi.org/##1]{##1}.\ }{}%
   \_bprintb [url]        {\_preurl\_url{##1}. }{}%
}
}

\def\optparams{\adef<##1>{\hbox{$\langle$\it##1\/$\rangle$}}}
\toksapp\everyintt{\optparams}
\toksapp\everytt{\typosize[9.5/12.4]}

\worktype [M/EN]

\faculty{F8}

\department{Department of Theoretical Computer Science}
\title  {x86-64 native backend for TinyC}
\titleCZ{x86-64 nativní backend pro TinyC}
\author{Michal Vlasák}
\date{XX.\,XX.\,TODO}
\supervisor{Ing. Petr Máj}
\abstractEN {
}

\abstractCZ {
}

\keywordsEN {%
TODO
}
\keywordsCZ {%
TODO
}
\thanks {% Use main language here
}

\declaration {
I hereby declare that the presented thesis is my own work and that I have cited
all sources of information in accordance with the Guideline for adhering to
ethical principles when elaborating an academic final thesis.

I acknowledge that my thesis is subject to the rights and obligations stipulated
by the Act No. 121/2000 Coll., the Copyright Act, as amended. In accordance with
Section 2373(2) of Act No. 89/2012 Coll., the Civil Code, as amended, I hereby
grant a non-exclusive authorization (licence) to utilize this thesis, including
all computer programs that are part of it or attached to it and all
documentation thereof (hereinafter collectively referred to as the "Work"), to
any and all persons who wish to use the Work. Such persons are entitled to use
the Work in any manner that does not diminish the value of the Work and for any
purpose (including use for profit). This authorisation is unlimited in time,
territory and quantity.

V XX dne XX.\,XX.\,TODO
\signature
}

{\nopagenumbers
  {\pgbackground={
    \picwidth=\pagewidth \picheight=\pageheight
    \inspic{vlasami6-assignment.pdf}}
    \null\vfil\break}
  \null\vfil\break}
\makefront

\chap Introduction

Computers are useful tools that accept programs and input data and produce
output data. Most computers are based on the binary numeral system---they use zeros
and ones. Because of this, data and programs also had to be expressed as ones
and zeros. While representing for example numbers is relatively easy, expressing
other data, like text is more challenging. Even more so, if the computers
interact with humans who need to be able to enter data and programs and read
back results.

Constructing computers with hardwired programs, which essentially can serve only
a specialized purpose was never feasibly. Compilers are designed to be able to
read programs as well as the data. This way, they can not just solve different
variations of the same problem, but different problems altogether.

ISA

Backend vs frontend.

General instructions sets allow the same algorithm to be implemented in many
ways using many different instructions and their modes. Because we want to use
the power of our machines to the maximum, another goal of compilers is to
translate the programs as well as possible.


\chap x86-64 architecture

AKA AMD64

dlouhá historie

původně 16 bit,

TODO mnemonics vs opcodes

\sec Characteristics

CISC

Two-address code

kódování založené na osmičkách (3 bity) => 8 registrů, 8 binárních operací, 8
unárních operací, 8 shiftů, apod.

podporuje 8, 16, 32 a 64-bit operace, z pohledu efektivity (partial register
stalls) a jazyka C se vyplatí používat jen 32 a 64 bitové operace (zbytek movzx,
movsx nebo čtení z užších podregistrů)

omezení často vychází z kódování instrukcí (shifty a dělení jsou kódovány jako unární operace
a nemají v kódování místo na víc než jeden argument)

\secc Addressing modes

ModRM bajt, SIB adresování + RIP relativní adresování

Max. jeden paměťový operand

Dvouadresový kód - kódování stále stejné, jen směr otáčí jeden bit v opcodu

\sec Calling conventions

System V AMD64 ABI. 

Registry pro argumenty, návratové hodnoty, caller saved, callee saved

\sec Comparison with t86

Srovnání s t86

\label[chap:state-of-the-art]
\chap State of the art

kapitola o představení teorie za různými částmi překladače a zmínka zavedených
existujících algoritmů a jejich přístupů

\label[sec:backend]
\sec Structure of a compiler backend

Instruction selection + Instruction scheduling + Register allocation

Je potřeba zvážit middle end, jedná se o vstup

Middle end často SSA, je potřeba s tím počítat (SSA dekonstrukce) nebo to rovnou
využít (register allocation na SSA)

\secc Phase ordering

Backend fáze na sobě vzájemně závisí

regalloc vytváří instrukce

selection neví co nakonec bude v registrech

selection neví jaké registry nakonec budou použity

Klasikcé řešení selection -> scheduling -> regalloc je přijatelný kompromis

\label[sec:SSA]
\sec SSA form

SSA ({\em static single assignment}) form is a form used by most of today's
optimizing compilers, including for example LLVM and GCC. It simplifies and
makes faster a lot of classic optimizations.

Like the name suggests, static single assignment form stands on the fact
that each variable is assigned exactly once. On top of that, we are interested
only in {\em static} assignments, that is, there is only one program point that
assigns the variable (contrary to {\em dynamic} assignments, which would count
how many times the assignment is executed at runtime, i.e.\ how many times the
execution gets to the single program point which assigns the variable).

There are many important benefits to SSA form, of which we highlight a few:

\begitems
* Since each variable is assigned only once, no variable is ever reassigned.
Thus value held by the variable is always available. This means that algorithms
don't have to be cautious about using a definition that is reassigned.

* There is no ambiguity in what {\em definition} of a variable a {\em use} can
refer to, since each variable has exactly one definition. This means that
instead of using {\em use-def chains} (a data-structure that links together all
definitions that may reach a use), uses can refer directly to the unique
definition.
\enditems

As an example, here is an example that is not in SSA form, because `a` is
assigned twice:

\begtt
a = 1
b = 1;
a = 2;
return (a + b) * (a + b);
\endtt

A human can easily tell, that the first store to `a` is dead, since `a` is
reassigned. It is also easy to tell that the both of the expressions `a * b`
compute the same result, since they refer to the same `a` and `b`. However, this
is not as easy to tell for a compiler. But \"versioning" each definition of a
variable and tracking which version gets used makes the observation simpler for
an optimizing compiler:

\begtt \adef!#1{$_{\tt #1}$}
a!1 = 1
b!1 = 1;
a!2 = 2;
return (a!2 + b!1) * (a!2 + b!1);
\endtt

\def\v#1#2{\code{#1}$_{\tt #2}$}

Here definition \v{a}{1} has no uses, this can be seen by keeping {\em def-use
chains} (which link together all uses of a definition, and which are not to be
confused with {\em use-def chains} mentioned above). But, the fact that \v a2 is
defined, doesn't make \v a1 unavailable and if it held more interesting value
then a constant, an optimizing compiler could use it even after the definition
of \v a2. It is also much easier to tell that the `a + b` expressions are indeed
the same, since now they are the result of applying the same operator to the
exact same versions of variables, regardless of how many definitions these
variables had. The implementation of SSA construction by versioning the
definitions is also very easy to implement.

Constructing SSA becomes more problematic with conditional execution:

\begtt \adef!#1{$_{\tt #1}$}
b = 1;
if (a) {
    b = -b;
}
return b;
\endtt

When we try to version the variables here, the we don't know whether to continue
with \v b 1 or \v b 2 after the `if` statement, though we are able to tell just
fine that the use of `b` in the conditional branch corresponds to \v b 1:

\begtt \adef!#1{$_{\tt #1}$}
b!1 = 1;
if (a!1) {
    b!2 = -b!1;
}
return b!?;
\endtt

The problem is that multiple definitions reach a use. With use-def chains this
could have been easily represented. But with SSA we don't want to use def-use
chains, since we want only one definition to reach each use. This brings us to
the idea of actually introducing a definition after the `if` statement, which
merges \v b 1 and \v b 2 into \v b 3, and can then be unambiguously used by
the return statement:

\begtt \adef!#1{$_{\tt #1}$} \adef|{$\phi$}
b!1 = 1;
if (a!1) {
    b!2 = -b!1;
}
b!3 = |(b!1, b!2);
return b!3;
\endtt

Of course, the ambiguity is still there, just hidden behind the mysterious
$\phi$ (\"phi") function which provides the merging definition. We would
like the $\phi$ to evaluate to right version depending on the control flow which
occurs at run-time, so we define the $\phi$ function to do exactly that. Though
the semantics may seem a bit weird. Having the ambiguity in phi is is still
better than use-def chains, since there can be other uses of \v b 3 all
referring to the single $\phi$, instead of each having multiple reaching
definitions.

The simplest possible method for SSA construction inserts a $\phi$ instruction
for every variable at each merge point. This is correct, but such approach
introduces many redundant $\phi$s, that don't do nothing much useful. For
example in the example above a $\phi$ instruction would be introduced after the
conditional branch for `a`, even though there is a single definition:

\begtt \adef!#1{$_{\tt #1}$} \adef|{$\phi$}
b!1 = 1;
if (a!1) {
    b!2 = -b!1;
}
a!2 = |(a!1, a!1);
b!3 = |(b!1, b!2);
return b!3;
\endtt

Similarly redundant $\phi$ functions are created even in loops:

\begtt \adef!#1{$_{\tt #1}$} \adef|{$\phi$}
a = 1;
loop:
if (f()) goto loop;
\endtt

Here, even though there is not reference to `a` in the loop, a $\phi$ function
needed to be introduced, since execution may flow to the beginning of the loop
from two places (the code preceding it, or the end of the loop:

\begtt \adef!#1{$_{\tt #1}$} \adef|{$\phi$}
a!1 = 1;
loop:
a!2 = |(a!1, a!2);
if (f()) goto loop;
a!3 = |(a!2, a!2);
\endtt

The $\phi$ refers to itself and only one other value (\v a 1), so it also can be
safely removed and the value used instead. Similar \"cyclic" $\phi$s can occur
even indirectly---two $\phi$ referring to each other and a single value, which
could be used directly.

So even though even simple SSA construction is possible, the produced code isn't
as useful because of many redundant $\phi$ nodes. The original inventors of SSA
form and the $\phi$ function concept~\cite[Rosen1988] came up with an efficient
algorithm for SSA construction~\cite[Cytron1991], which is based on the new
concept of {\em dominance frontiers}, which simply put are exactly the blocks
where $\phi$ functions are needed. Their algorithm produces what is known as
{\em minimal} SSA form, which doesn't contain redundant $\phi$s. Despite the
name, even more \"minimal" forms of SSA exist, for example {\em pruned} SSA
doesn't contain {\em dead} $\phi$ functions.

\label[sec:SSA-value]
\secc Value-based SSA

An important observation with regards to SSA form is, that since each variable
is assigned exactly once, and the variables don't get reassigned, there is no
need for the concept of a {\em variable}. The {\em values} assigned to the
variables can be used directly instead. Values are described by their structure,
for example number literals, like `5` or operations applied to other values
like `add 1, 2` (addition of two numbers) or `add 5, (sub 6, 7)` (addition of a
number and the result of subtraction of of two numbers). Generally we can divide
the values used in programs into two categories:

\begitems\style n
* {\em Constants}. These represent number or character literals, but also (addresses
of) static objects. Examples include `3` (integer literal), `'a'` (character
literal), `f` (address of a function), `g` (address of a global variable).

* {\em Operations}. These represent values produced from other values by
applying some arithmetic or other operation. Examples include `neg 5` is a
(unary) negation operation applied to a constant, `add 5, neg neg 3` is a binary
operation applied to a constant and a negation applied to negation of a constant
number 3. Arity isn't limited, and for example function calls can be seen as
operations on several values of which the first is the called function and the
rest are the arguments, for example we can have function `f` called with 4
arguments: `call f, 1, 2, 3, 4`.
\enditems

Function parameters are constants as well---even though arguments
(actual parameters) passed to a function may just be results of some operations,
from the point of view of the function the formal parameters are constant values
that \"magically" materialize when the function begins its execution.

When represented in a compiler, the values are objects and they refer to each
other by means of references (for example through pointers or by means of
reference types). Our textual notation for operations so far was direct: a use
of value meant writing out the textual representation of the definition
(constructor) of a value. Because each value can be used many times and due to
the recursive nature (operations are able to refer to other operations), this
quickly becomes unwieldy. We will use a different notation that assigns each
value a number, $i$, and writes the references as `v`$_{\tt i}$. The definitions
are accompanied by a \"virtual assignment" whose purpose is to show what index
is used to represent references to the particular value. For example, we could
have:

\begtt \adef!#1{$_{\tt #1}$} \adef|{$\phi$}
v!1 = 3
v!2 = 3
v!3 = mul v!1, v!2
\endtt

Which differs a bit from:

\begtt \adef!#1{$_{\tt #1}$} \adef|{$\phi$}
v!1 = 3
v!3 = mul v!1, v!1
\endtt

This notation also easily supports operations that refer to themselves, like the
redundant phi functions we have seen for loops:

\begtt \adef!#1{$_{\tt #1}$} \adef|{$\phi$}
...
v!2 = phi(v!1, v!2)
...
\endtt

It should be noted that our notation for values doesn't imply anything about the
order of operations---the only theoretically imposed order is the dependencies
among the operations themselves. The references of operations to each other
actually form a {\em direct acyclic graph} (DAG) showing the data
dependencies---{\em data-flow}. TODO figure

%This representation has been pioneered in practice by LLVM and used for example
%also by the B3 JIT compiler.


\label[sec:SSA-machine]
\secc SSA for machine code

Though the {\em value-based} representation of SSA may be useful in some
contexts, it is not always possible to use it. After all, the resources for
which we may want to construct SSA form may not represent values at all. Or the
values may be constrained. In such cases transforming the intermediate
representation to SSA form is still possible, but has to be based on versions
and we will call it {version-based} SSA.

One such example is SSA form for {\em machine code} (or assembly). Here, we work
with registers, and registers no longer {\em represent} values, they {\em store}
values. Say we want to represent the following x86-64 instructions in SSA form:

\begtt
mov rcx, rax
add rcx, rbx
\endtt

This example represents the addition `rcx = rax + rbx`, where all three
registers are preserved. Obviously, in this example `rcx` is assigned twice.
We want to apply the standard SSA versioning construction. The two address
encoding actually means that `rcx` is in fact both read and written, but the
read `rcx` is different than the written `rcx`. Thus we also need to extend our
representation of our instructions to separate the register reads ({\em uses})
from writes ({\em definitions}):

\begtt \adef!#1{$_{\tt #1}$}
mov rcx!1 | rax!1
add rcx!2 | rcx!1, rbx!1
\endtt

Above we follow the convention where all definitions are written before the
\"\|" symbol, and all uses after. The representation actually became more
similar to classic three address code, where operations are non-destructive.

Other snippets of x86-64 code can also produce a bit surprising results:

\begtt \adef!#1{$_{\tt #1}$} \adef>{$\Rightarrow$}
cmove rdx, rsi   >   cmove rdx!2 | rdx!1, rsi!1

setnz al ; 2)     >   setnz rax!2 | rax!1

xor rcx, rcx     >   xor rcx!1
\endtt

In the first example we have the conditional move if equal instruction. It moves
`rsi` to `rdx` only if the zero bit is set in the flags register. For our
purposes the final value of `rdx` can be `rdx` or `rsi` and with no better
information, we must pessimistically expect both. So we have two uses and one
definition.

Second example has a set if not zero instruction, which sets the lowest 8 bits
of `rax` register if the zero bit is not set in the flags register. Since only
the lowest bits are set, the instruction actually merges the new low 8 bits with
the rest of the `rax` register as set previously, and this has to be modelled
through a use of the previous version of the register.

In the last example, the idiom for zeroing out a register is used. Xoring a
register with itself produces zero, regardless of the previous value of the
register. Compared to move of 4-byte immediate zero, this is much shorter, so
the idiom is very common. But, since the register is overwritten with zero,
there is actually no reason for the instruction to {\em use} the former value of
the register--- the previous value doesn't matter. This is why we want to model
the instruction as only {\em defining} the register. Actually, the processor
itself doesn't have to wait for the used register, and for example Intel
considers using xor to zero out a register a {\em
dependency-breaking-idiom}~\cite[IntelOptimization].

SSA on machine code accomplished its goal---there is now only one assignment to
each {\em register version}. But the versions of one register are still tied together.
When the SSA form is deconstructed back into machine code, essentially only the
version numbers are dropped. So while SSA helped with {\em analysis}, because
tracking definitions became easier, in reality there are still multiple
assignments to each resource.

\label[sec:ssa-deconstruction]
\sec SSA deconstruction

SSA dekonstrukce potřeba někde

Problém kdy a jak (copy instrukce v SSA middle end IR? (nejsou SSA), nebo
naopak backend IR v SSA i s phi nody, to ale zase špatně podporuje to, že
některé instrukce prostě SSA neumožňují - two address code, binární operace,
`setcc` instrukce, je to řešitelné, ale...)

Swap problem, lost copy problem \cite[Briggs1998]

Problém nejen korektnost, ale i rychlost (spousta kopií škodí)
\cite[Boissinot2009].

Zbytek SSA dekonstrukce nečíst, velmi hrubý začátek:

No common processor provides $\phi$-functions on which the SSA form relies. Thus
replacement of $\phi$-functions with other instructions having same effect is
needed. This is step is called {\em SSA deconstruction}.

Since $\phi$-functions are the means of achieving single static
assignment, necessarily eliminating them means that the program will no longer
be in SSA form. Being able to multiply assign a variable.

$\phi$-function's semantics say that it evaluates to whichever values is
appropriate to the control flow happening at runtime. In a representation based
on control flow graphs, this is made explicit by basic blocks

We can achieve the effect of a $\phi$-function with multiple assignments through
a copy instruction in each of the predecessor blocks. TODO figure.

This is the original way of eliminating $\phi$ functions and was introduced
by~\cite[Cytron1991]. Though seemingly simple, there are multiple subtle
problems, which have demanded improvements in this area.

One of the problems is, that there may be no predecessor, that is suitable for
the copy. This can be seen on the following program in SSA form:

\begtt
TODO
\endtt

Inserting copies in to predecessor blocks would look like this:

\begtt
TODO
\endtt

But the problem is that copy inserted into block Z for deconstructing the
$\phi$-function X in block Y takes effect even in case the execution doesn't
actually go to block Y, but goes to block W instead. The problem stems from the
fact, that control flow can transfer from block with {\em multiple successors} (where
execution continues in only one of the successors) to a block with {\em multiple
predecessors} (where $\phi$ functions may be needed to merge multiple reaching
definitions). In a control flow graph where possible transfers of control flow
across blocks are embodied by edges, this is known as a {\em critical edge}.
It is possible to split a critical edge by replacing it with an intermediate
block. The block will have only a single predecessor (the block with multiple
successors, block Z in our example) and a single successor (the block with
multiple predecessors, block Y in our example):

\begtt
TODO
\endtt

Now the copies can be inserted into the new basic block and are executed only
when control flow actually goes from block Z to block Y:

\begtt
TODO
\endtt

Another problem with eliminating $\phi$-functions correctly is that semantically
it is as if all $\phi$-functions in a block executed in parallel at the time the
control flow enters the block. This is due to the fact, that $\phi$-functions
evaluate to the right value based on control flow alone. But copy instructions that
we use for replacing $\phi$-functions execute sequentially. With multiple
$\phi$-functions in a basic block and certain dependencies a naive insertion of
copies into predecessor can lead to incorrect translation.

Briggs~\cite[Briggs1998] identified two examples where this happens, the first is
known as {\em lost-copy} problem and occurs when deconstructing SSA in code like this:

After SSA deconstruction:

\begtt
TODO
\endtt

A copy is missing. TODO. But the example actually has an unsplit critical edge
and splitting it fixes the problem:

\begtt
TODO
\endtt

Though this demonstrates that splitting critical edges can be fairly expensive
in loops where it introduces additional unneeded jumps. The lost-copy problem
can be prevented in a much simpler way:

\begtt
TODO
\endtt

The second example identified by~\cite[Briggs1998] is the swap problem and
can be demonstrated on this example:

\begtt
TODO
\endtt

There are two $\phi$-functions one of which depends on the result of the other.
Naively inserting the copies fails to preserve the right semantics of
$\phi$-functions:

\begtt
TODO
\endtt

Additionally, splitting the critical edge doesn't help:

\begtt
TODO
\endtt

Both problems are ultimately due to dependencies among the $\phi$-functions and
their arguments. Bad order of copies can overwrite needed values. Simplest
possible solution is to do the copies in two steps: copy all phi node arguments
into temporaries and only then copy from temporaries into the right
destinations:

\begtt
TODO
\endtt

The approach is correct, since the first step is non-destructive (unlike the
naive copy insertion) and in the second (destructive) step only new temporaries
are read, which are not involved in any $\phi$ functions. While this is correct,
it produces a large number of copies, most of which are not needed.

Briggs~\cite[Briggs1998] approaches the problem as a scheduling problem. Since copy
instructions have a destination (definition) and a source (use), we try to find
an order such that all uses of a virtual register precede its definition. If
this is not possible due to cycles (like in the {\em swap} problem), then a
temporary and an additional copy needs to be inserted to break the cycle.

Briggs also makes an important observation---the problems only manifest after
some of the more aggressive operations are run (like copy folding), and not
after others (like copy propagation or dead code elimination).
Sreedhar~\cite[Sreedhar1999] calls the form of SSA constructed
by~\cite[Cytron1991] and other algorithms {\em conventional SSA} and notes that
it has the important property that all virtual registers in the same {\em phi
congruence class} (containing virtual registers connected via
$\phi$-instruction) can be replaced by one representative and the
$\phi$-instruction eliminated. This means that for example the following code
(in conventional SSA):

\begtt
TODO
\endtt

Can be translated just by \"omitting the versions":

\begtt
TODO
\endtt

Importantly, in~\cite[Sreedhar1999] they also consider {\em transformed SSA},
which doesn't have the {\em phi congruence property}, and present three methods
for transforming transformed SSA into conventional SSA.
The first method is remarkably simple. In addition to the copies in predecessor
blocks a copy of the $\phi$-instruction is added. But these are different copies
than in previous algorithms! Here, we are still in the realm of SSA, so these
copies can actually be considered {\em aliases} for the original values and they
obey the single assignment property:\fnote{In fact, exactly because these SSA
values are aliases, they are folded by algorithms like {\em copy folding}, that
result in transformed SSA form, which exhibits problems with naive SSA
deconstruction.}

\begtt
TODO
\endtt

The next two methods gradually build on top the first one and additionally make
use of liveness (described later in section~\ref[liveness]) and interferences
(also described in a later section, \ref[interference]), to prevent inserting
too many unnecessary copies.

Full SSA deconstruction based on Sreedhar's approach thus consists of
transformation to conventional SSA form, merge of virtual registers in the same phi
congruence class (\"drop of versions of variables") and simple deletion of the
$\phi$-instructions.

The most comprehensive treatment of SSA deconstruction was done
by~\cite[Boissinot2009]. Their approach highlights more subtle issues with SSA
deconstruction, and presents an algorithm partly based on~\cite[Sreedhar1999],
which mainly aims for mainly correctness, but also quality (of the generated
code) and efficiency (of the SSA deconstruction). The signature of their
algorithm are three passes, which strictly separate between correctness and
optimization. Their approach also makes use of some advantages of SSA, like fast
liveness checking.

Copies can be eliminated by {\em coalescing}. Intuitively, if we merge the
source and destination of a copy instruction into a single virtual register,
then no copy instruction is needed (since it would be a copy to itself).
Basic approaches to SSA deconstruction insert copies freely and depend on
eliminate the copies through coalescing. But doing some coalescing in SSA
deconstruction stage can help greatly, since many fewer virtual registers need
to be considered in register allocation stage (speeding up all aspects of it).
Also, while in SSA form analysis, many analysis are faster and cheaper to
compute and more information may be available to guide the coalescing. Because
of these reasons both~\cite[Sreedhar1999] (methods 2 and 3)
and~\cite[Boissinot2009] do coalescing as part of SSA deconstruction.

Briggs describes extensions to SSA deconstruction that are able to correctly
translate the two examples, but 

These two examples demonstrated that naive SSA deconstruction as described
by~\cite[Cytron1991]. 


For example, this program in SSA form:

When $\phi$-functions are naively eliminated with copies we get:

But the problem is that although the 


phi functions in parallel

lost copy

swap






A lot of intermediate representations may use single static assignment and
$\phi$-functions. The

Since the single static assignment property of SSA can be achieved in any
intermediate representation, they can have different other needs apart from
replacing $\phi$-functions. 

For example in value based SSA form it may be
desirable to also 

Replacing $\phi$-functions is not the only thing SSA deconstruction has to deal
with. 

Code in SSA form is not directly executable by any common processor. Not only
because the backing IR representation may not represent actual machine
instructions, but because even if it did, $\phi$-functions 

There are two problems with SSA form that make it problematic to translate to
non-SSA form. First is, that in SSA form instructions represent {\em values}.
While they can be thought of as virtual registers, they are not registers, since
they cannot be assigned. The second problem is that SSA form relies on
$\phi$-functions, which don't have equivalent instructions in any usual
instruction sets (like x86-64). Both problems make SSA deconstruction a mandatory step,
that has to happen sometimes before machine code is emitted.

%This is the first
%disconnect of SSA form from real instruction sets, like x86-64, which operate on
%registers. For example, in SSA form, addition of two numbers looks like this:
%
%\begtt
%v1 = add 3, 4
%\endtt
%
%The name `v1` represents the value, which is the result of operation adding
%numbers $3$ and $5$ together. The equal sign doesn't really represent an
%assignment---what is right to the equal sign represents the value entirely. The
%reason why we use the equal sign is that other operations might want to refer to
%previous values, and in textual notation we need to write that somehow. For
%example, `v1` might be used to produce `v2` like this:
%
%\begtt
%v2 = add v1, 5
%\endtt
%
%Implementations often represent values as \"objects" (instances of classes,
%structs) and operations refer to other values through {\em pointers}. While even
%for implementations it might be beneficial to number values like we do in the
%textual representation (because then a simple array can be used for mapping the
%values to some other secondary information), there is no real semantic reason to
%do so.

%So in SSA form values are produced from other values and these values sometimes
%represent operations, instructions that execute and compute the values,
%otherwise values just represent static objects. In machine code we only have
%a few physical registers and instructions which usually read from registers and
%memory and also write to registers and memory. To translate SSA form we need to
%somehow map the values into registers, translate SSA instructions (value
%computations) to machine instructions, and we also need to somehow translate the
%static objects and be able to take their addresses.
%
%A powerful abstraction that pretty much all compilers have used is, to pretend
%in early stages that there is an infinite amount of registers available, the so
%called \"virtual registers". A later stage, called {\em register
%allocation}, is responsible for translating the use of unlimited amount of
%virtual registers into a fixed amount of physical (machine) registers, which is
%available for the particular kind of processor we are targeting. Even though
%there are techniques which do register allocation in SSA form and deconstruct
%SSA form only after (or during) register allocation, we will focus more on the
%traditional approach of deconstructing SSA form before register allocation and
%thus we will have the luxury of unlimited supply of (virtual) registers. We will
%number the virtual registers from $1$ and we will prefix them with the letter
%\"t" (for {\em temporary})\fnote{Using \"v" as \"virtual register" would be
%possible, but would clash with our notation for numbering SSA {\em values}.} for
%example `t1` represents the first virtual register.
%
%With the constraints given above, translating SSA to something like a three
%address code is relatively straightforward. The first register in three address
%code is the destination, where the result of the computation is written, while
%the other two registers are the sources, the arguments of the instruction. Each
%SSA value (which is an operation) can thus be assigned its own virtual register,
%which will hold the SSA value and like in SSA it will be assigned statically
%just once. But other instructions translated later will be able to refer to
%it through that virtual register. So something like:
%
%\begtt
%v3 = mul v1, v2
%v4 = div v1, v2
%\endtt
%
%Can be translated to:
%
%\begtt
%mul t3, t1, t2
%div t4, t1, t2
%\endtt
%
%For readability purposes, we used one-to-one mapping from value numbers (used
%for the textual notation of SSA form) to virtual registers. This makes it easy
%to follow by human, but the compiler itself doesn't have to do it this way, it
%just has to ensure to use a consistent mapping of values to virtual registers.
%The mapping is not as straightforward as it might seem though, because 
%
%Another so far neglected aspect of SSA form are the $\phi$ instructions. No real
%processors offers something like $\phi$ instruction, so we have to translate
%them specially.

\sec Instruction selection

Instruction selection is the process which translates a compiler's intermediate
representation to machine instructions. While compiler middleend intermediate
representations ({\em IRs}, which are machine independent) are usually made to
be easily optimizable, machine instructions are usually designed to be
conveniently executable by a processor.

To ease translation and split concerns instruction selection is usually allowed
to assume that there is an infinite amount of registers available. A later
phase, called {\em register allocation}, then.

If the middle end IR and machine instruction set are relatively close, then even
something as simple as one-to-one mappings are possible. For example, this
middle end IR:

Can be translated to the following RISC-V code:


The mapping is easy, because RISC machine instructions, where all operations are
done on registers, and only load and store instructions touch memory, closely
resemble the three address code used by the IR above. Even on CISC
architectures (which are known by allowing most instructions to use memory
operands and various addressing modes), translation of RISC-like middleend IRs
is straightforward, since simple forms of instructions are still available and
the complex features will simply not be used. For example, this is how the
piece of IR above could be translated to x86-64:

But use of advanced addressing modes and memory operands in arithmetic
instructions can improve the generated code considerably:

Even though we can intuitively tell that the code is \"better", it is very
important to consider exactly why. {\em Fewer instructions} are used, but is the code
{\em smaller}? Does it execute faster? Does \"faster" consider throughput as well
as latency? These questions should be considered by instruction selection and
their answers should guide the instruction selector to find better code. Often,
instruction selectors operate in a {\em cost} based model, where each
instruction is given a cost  and, if possible, instructions with the lowest cost
are chosen. This cost should ideally consider all mentioned factors, but can be
hard to balance between speed and size trade-off, and since both can be important
in different scenarios, having different instruction costs for different
situations can make sense.

Different orders of instructions can have impact on execution speed as well.
Usually, this problem is not considered as part of instruction selection, and
separate {\em instruction scheduling} pass is used to reorder instructions to a
more favorable order. Instruction scheduling is described
in~\ref[sec:instruction-scheduling].

\secc Techniques

The first two instruction selections presented above, can be done pretty simply
programatically. Each IR instruction is in turn {\em expanded} into one or more
machine instructions. Registers (or more precisely {\em virtual registers} in this
stage before register allocation) are used as means of abstraction---each
instruction reads operands from registers and writes results to registers.
Different operand kinds (such as immediates or memory locations) are handled by
special few instructions that are able to load immediate or perform load from
memory or store to memory. The use of registers has to be consistent, e.g.
if an instruction maps to multiple instructions which need a temporary register,
none of the input registers should be used, since further IR instructions may
also read from them. Either a fresh virtual register has to be allocated, or one
of the destination registers can be used, since they will be overwritten
anyways.

This method produces a simple one-to-many translation. While the instruction
sequence emitted for each single opcode can be optimal, most of the interesting
instruction selection opportunities stem from the fact that there are
instructions which combine several operations. In same cases this means
many-to-many or even many-to-one mappings.

There are three fundamentally different schools of handling instruction selection.
Most existing algorithms for instruction selection fall into one of these three
categories, though even algorithms falling into the same category can differ
significantly in their speed (how fast they run), quality (how fast does the
produced code run) and capabilities (e.g. ability to handle multiple
destinations).

\seccc Peephole optimization

Peephole optimization is a general optimization technique (see
section~\ref[sec:peephole]). It is applicable to any compiler stage, but in the
backend peephole optimization is often associated to final improvements to
machine code after instruction selection, instruction scheduling and register
allocation. But peephole optimization has been applied even to perform
instruction selection. To distinguish this, the technique is often called {\em
instruction selection by peephole optimization}.

The idea for instruction selection by peephole optimization is simple. If the
naive expansion method produces a simple one-to-many translation, we can improve
it, by a peephole optimization pass that can find patterns and replace multiple
instructions with shorter instruction sequences or more efficient ones. Since a
lot of expansions are mere translations one-to-one even a relatively small
peephole window can actually examine many instructions in the original IR.

But, while the peephole optimization can be done on the machine instructions
themselves, they are not much suitable for optimization. Apart from that, they
are also machine specific, and while improvements to instruction selection are
machine specific by nature, it is preferable if a single mechanism can drive
the optimization for many different instruction sets. Also, writing the peephole
optimization patterns by hand is tedious and does not scale for a compiler
targeting many architectures.

Influential in this are was the algorithm by Davidson and
Fraser~\cite[Davidson1980]. Their retargetable peephole optimizer is based on
{\em register transfers}, which describe simple low level machine
operations---like actual transfers among registers, but also arithmetic and
settings of flags. Each machine instruction has an associated {\em register
transfer list}, which describe the effects of the instruction. Their algorithm
then operates in three passes:

\begitems \style n
* Translate all machine instructions to register transfer lists.

* Iterate over the register transfer list backwards and determine the observable
effects (register transfers) of each list.

* Iterate over the register transfer lists forwards and check for each pair of
whether there is an instruction whose register transfer list has the combined
effects of the pair. If there is, it combine the two register transfers.

* Iterate forward over the code and for each register transfer list find an
instruction that implements it and emit it.
\enditems

The first step is actually optional, since the compiler can generate the
register transfers directly, but nominally their algorithm both reads and writes
machine instructions. The register transfers are only an immediate step, and
importantly the backwards pass deletes unobservable effects, which don't need to
be implemented by the combined instructions. The effects are mainly assignments
to registers and flags, which they both handle in a unified way. Since an
assigned (virtual) register or a flag may not ever be read before being assigned
again, it may be deleted. The deletion may take form of either combining into an
instruction which doesn't have the extraneous effect (in phase 3) or replacement
by a single final instruction (phase 4) which also doesn't need to have the
extra effect. This is a form of dead code elimination (since for example an
empty register transfer list doesn't translate to any instruction) but the
extraneous assignments are kept if there isn't an instruction that doesn't have
them (for example, most arithmetic instructions set flags, even if they are not
needed, there may be no other instruction that doesn't set the flags).

Important aspect of Davidson's and Fraser's approach is that the correspondence
between register transfers and machine instruction is described by a textual
file called {\em machine description}. The compiler interprets the file and uses
it to map instructions to register transfers (phase 1), replace pairs of
register transfer lists (phase 3) and to map register transfer lists back to
instructions (phase 4). All of these phases use the textual description and are
thus based on operations with strings. This has the advantage that the compiler
is easily retargetable just by swapping out the machine descriptions. Since in
the last phase register transfer lists are mapped back to instructions the
algorithm has to obey an invariant, that any produced register transfer list is
implementable by at least one instruction, this is why phase 3 is guided by the
machine description and why the machine description has to be written with care.
In phase 4, if more machine instructions can implement the same register
transfer list the first one in the machine description is chosen (simply because
the algorithm tries to match with all instructions one-by-one). This can be used
to model (relative) costs of instructions with equivalent behavior.

Later improvement of the algorithm by the same authors~\cite[Davidson1984a]
better formalized the phases of the algorithm (figure~\ref[fig:TODO]). The {\em
expander} produces register transfers, {\em combiner} combines pairs of
instructions and {\em assigner} translates register transfers back to machine
instructions (and in their approach also performs register allocation).
Combiner and assigner are now improved and based on finite automaton generated at compile
time of the compiler ({\em compiler-compile time}), making them much faster.
Though the simulation of combined effects of instructions happens still on
strings. {\em Cacher} is a new addition and performs (local) common
subexpression elimination by keeping track of what values are available in what
registers in a basic block. Another powerful addition was consideration of not
only {\em lexicographically adjacent} instructions, but also {\em logically
adjacent} instructions---so instead of considering instructions in their {\em control
flow} order, they are considered based on {\em data-flow}. This allowed to
consider simple definitions (like constants) to be folded into instructions,
even if they fell out of the small peephole window.

Davidson's and Fraser's design turned out to be very influential in the area of
peephole optimization, in particular the idea of expanding the instructions
into a machine independent representation, that exposes even the smallest
operations, performing optimizations and then combining the operations into
machine instructions. This aspect is kept by a lot of compilers using peephole
optimization for instruction selection. Even though it makes bigger peephole
windows necessary, using data-flow helps considerably. The idea of a
machine description that is preprocessed at compiler-compile time also stuck,
since it can be slow (as compiler is rarely compiled often by the end user), but
can speed up the compiler itself tremendously.

Other aspects of the original Davidson and Fraser didn't catch much. For later
algorithms opted for a {\em fixed set of patterns}, in place of deriving them on
the fly with symbolic execution. These patterns used to be written by
hand, but considering the existence of the machine descriptions and symbolic
executors based on them, Davidson and Fraser~\cite[Davidson1984b] used them to
derive the patterns. They do this by introducing a set of programs known as the
{\em training set}, over which they run their classical peephole
optimizer~\cite[Davidson1984a]. With each (unique) performed optimization they
write it to a file, which is then used to run a more limited peephole optimizer,
which operates only based on these patterns. While the files are still text
based, they use {\em string interning} through a hash based data structure to
avoid comparing strings and to speed up strings comparisons.

TODO: figure data-flow vs control flow

TODO: examples

%Apart from important improvements to the algorithm itself, in
%\cite[Davidson1984a] they point out important interactions between the middleend
%and backend that make their approach work well

%Machines have many specifics and special cases that are just too hard to handle
%at the code generator ({\em expander}) level. On x86-64 these can for example be
%zeroing out registers with `xor reg, reg` instead of with `mov reg, 0`, or using
%the `lea` (intended for address computations) to perform ordinary arithmetic.
%Handling these specifics in when selecting instructions would mean a lot of case
%analysis. If instead the code generator is as naive as possible and based on a
%relatively simple.

%Only the expander depends
%
%Though
%the string operations and no preprocessing of the machine descriptions with
%combination of symbolic combination of effects of instructions meant that the
%compiler wasn't that fast.
%
%However, k
%
%Their peephole optimizer traverses the list of instructions and
%tries to find for each pair a single instruction that has the combined effects. 
%
%Many of these problems have been solved by Davidson and
%Fraser~\cite[Davidson1980, Davidson1982, Davidson1984]. Their peephole optimizer
%operates on {\em register transfer lists}, which are a machine independent
%representation of low level operations performed by instructions, many of which
%are indeed register transfers, but include also arithmetic and manipulations
%with flags. The register transfers describe {\em effects} of instructions and
%their peephole optimizer examines each pair (and later triplet) of instructions
%and replaces them with a single equivalent instruction if it exists. The
%combined effects of more instructions are simulated symbolically and
%single instructions having the combined effects are looked up in a {\em machine
%description}. 
%
%
%
%With this machine independent representation, they represent machine
%dependent instructions---the same representation can be used for many
%architectures. Additionally, their algorithm is driven by a symbolic machine
%description, so a single algorithm works for all architectures and support for
%new architectures can be easily added. The peephole patterns and optimizations
%themselves are not present in these machine descriptions. The peephole optimizer
%bases the optimizations entirely on 
%
%---as the peephole
%optimizer goes through the code, it simulates pairs (and triples) of
%instructions and, if applicable, replaces them by more efficient single
%instruction.
%
%Since the 



%This late peephole optimization doesn't have many opportunities for
%optimization, since instructions have already been selected, ordered and
%registers allocated. On top of that, the three big backend phases in a
%production grade often use global techniques, but the peephole optimizer has
%only a too local view. This 

\seccc Covering

\seccc Reduction

Dva hlavní přístupy tiling (tree, DAG, dynamic programming, bottom up rewrite,
LR parsování, ...) a peephole optimalizce (RTL, expanze a komprese)

Oba přístupy v podstatě to samé, jakmile peephole přestane pracovat s
instrukcemi, kterou jsou po sobě v CFG, ale začne využívat dataflow

\secc Tiling

Základní idea

Asi rychlé představení pár algoritmů tilingu (DP, LR, BURS)

Potřebuje DAG nebo tree.

Nevýhody - instrukce s více výstupy.

\secc Peephole

Původní návrh Fraser a Davidson. Jejich cíl bylo mít jednoduchý generátor
naivního kódu pro každou architekturu, univerzální \"register transfer"
reprezentaci instrukcí a machine description, který jeden stejný peephole
optimizer interpretoval (jako stringy). Každá dvojice a trojice nahrazena za
nejoptimálnější sekvenci. Výhody - žádná case analysis v základním code
generatoru, ani nikde jinde - peephole optimizer bere informace z machine
description. Problémy - založeno na interpretaci stringů, v podstatě dělá
abstract interpretation na všech instrukcích.

Další jejich vylepšení: expander, cacher, combiner. Uvažování logicky
následujících instrukcí (dataflow). Vygenerování patternů na
základě běhu s abstraktní interpretací na nějaké sadě programů ("rule
inference").

Pokud je stejná reprezentace použitá i pro register allocation, lze znovu
spustit jako cleanup fázi i po register allocation


\label[sec:regalloc]
\sec Register allocation

Register allocation is the last of the three big conceptual parts of a usual
compiler backend. Motivation, importance and possible approaches are introduced.

The x86-64 architecture (see \cite[x86]) is what we mainly care about in
this thesis. Since it is familiar, we will be using it as examples in the
following sections. It is also a good candidate because it brings some
challenges not found on other architectures, but shows the general problems just
as well as other architectures.

Note that like with instruction selection although we are already working with
target specific instructions, their form doesn't necessarily have to be target
specific. Target independent representation of target specific instructions
allows us to share also register allocation logic for all targets.

\secc Motivation

Most CPU architectures these days are register based. That means that interface
of the CPU consists of a fixed number of registers and instructions that allow
operations on these registers. For example registers may be eight 32-bit storage slots
and instructions may allow performing arithmetic on these registers or allow
loading/storing contents of register from/into memory. Memory is still an
important part of these architectures---computations can't possibly fit all into
a fixed number of registers of fixed size, it is the memory that allows us to
store large amounts of data.

During previous phases of the compiler we used a powerful abstraction, we
pretended that there is an infinite amount of registers. This is very important
for the middle end IR, since it is supposed to be platform agnostic and rather
than limiting to some fixed number of registers (per architecture or wholesale),
we might as well pretend to have infinite amount of them. But once we start
translating the middle end IR we just need to limit ourselves to fixed amount of
registers somewhere.

After instruction selection (which determines what instructions to use) and
instruction scheduling (which refines the order of the instructions), a snippet
of input to register allocation can look as follows:

\begtt
mov t1, 1
mov t2, 2
mov t3, t1
add t3, t2
\endtt

%The snippet could correspond to this middle end IR:
%
%\begtt
%v1 = add 1, 2
%\endtt
%
%The translation to instructions given above is suboptimal and in a reasonable
%compiler such code wouldn't get as far as to the register allocator: middle-end
%could fold the addition of constants into a constant, or instruction selection
%could take advantage of the \"register plus immediate" instruction for the addition.
%Nonetheless the example serves us well for showing how a simple allocation of
%registers might look like.


There are several things of note here. The instructions don't operate on real
machine registers (like `rax`), but on {\em virtual registers} (often also called
\"pseudoregisters" or \"temporaries"). It is the goal of register allocation to
transform the code so that {\em physical} (\"machine") registers are used.
Since the whole program doesn't use more than 16 registers, we have no problem
assigning x86-64 registers directly, for example in the order of the
temporaries:

\begtt
mov rax, 1   // t1 = rax
mov rbx, 2   // t2 = rbx
mov rcx, rax // t3 = rcx
add rcx, rbx
\endtt

Even for such a simple example, we can notice several things about register
allocation alone:

\begitems
 * We introduce a third register `rcx` to store the result of addition. This
works well and fits into the 16 registers we have available. But we can notice
that after the addition we no longer need the value stored in register `rax`.
This is on of the big ideas in register allocation, we only need to store those
values that will be needed in the future, and we can use that to \"reuse"
registers.

* If we were to reuse `rax` for storing the result of addition, our situation
would look like this:

\begtt
mov rax, 1   // t1 = rax
mov rbx, 2   // t2 = rbx
mov rax, rax // t3 = rax
add rax, rbx
\endtt

Move (copy) of a register to itself is a no-op, the instructions doesn't have
any real effect. It doesn't even change flags, to it is safely possible to
remove it. We can notice that the two address code generated
from SSA three address code can be improved if it turns out that the
destination can be the same register as the first source (or the second source
in this case, since addition is commutative).

Though this brings a question to which we will come back later: Can the register
allocator remove the instruction? Does it have sufficient information to do so?
Or should it even be concerned about the semantics of instructions it is working
with?
\enditems

While we have shown that opportunities for register reuse arise, it doesn't mean
we can't get out of registers. After all, there is a limited number of them, and
opportunities for reuse come only when a virtual register is no longer needed
later. Even relatively simple expressions can produce code which requires
surprising amount of registers, while not allowing much reuse. For example,
compilation of the expression `1 + (2 + (3 + 4))` can produce code like the one
below:

\begtt
mov t1, 1  // t1 = rax
mov t2, 2  // t2 = rbx
mov t3, 3  // t3 = rcx
mov t4, 4  // t4 = rdx

mov t5, t3 // t5 = rcx
add t5, t4

mov t6, t2 // t6 = rbx
add t6, t5

mov t7, t1 // t7 = rax
add t7, t6
\endtt

Possible register allocation is noted in the example. The right associative
nature of the expression means, that for each of the additions while the left
hand side (the immediate numbers) are evaluated first, its result has to be kept in
registers until the right hand side is evaluated and ready for the addition.
Register reuse is possible but only after the additions, because each has at
least one argument that is not needed further. The example could be extended to
exhaust all available registers. Contrary to that the left associative version
(i.e. `((1 + 2) + 3) + 4`) needs just two registers:

\begtt
mov t1, 1  // t1 = rax
mov t2, 2  // t2 = rbx
mov t3, t1 // t3 = rax
add t3, t2

mov t4, 3  // t4 = rbx
mov t5, t3 // t5 = rax
add t5, t4

mov t6, 4  // t6 = rbx
mov t7, t3 // t7 = rax
add t7, t5
\endtt

Since the operands are kept in registers only for short time in between the
additions, there are more possibilities for reuse. So even though both versions
use the same amount of {\em virtual} registers, they need different amounts of
{\em physical} registers. While instruction scheduling, or perhaps instruction
selection or middle-end optimizations can transform the right associative
version to the left associative one, or just fold the computation entirely
(since it's a sum of four constants), the example illustrates that virtual
register which are needed for a long time are a problem, since they prevent
register reuse and that because of that even simple examples can get out of
registers.

\label[sec:regalloc-spilling]
\secc Spilling

We have to make all values in virtual register available wherever they are
needed. But there may be too little of {\em physical} registers to do so. One
possibility of reducing the {\em register pressure} is to use memory.
In the simplified view of a compiler, there is essentially infinite amount of
memory available, so storing values does not deplete a limited resource as much
as using physical registers does.

%Memory
%isn't all-saving, since computations usually must involve at least some
%registers, but we can use memory to store values temporarily---between places in
%which the values need to be present in registers.

Techniques that involve using memory to reduce register pressure are usually
called {\em spilling}---alluding to the fact that what does not fit into
physical registers is put somewhere else, and that in fact it is an undesirable
thing and we use it only when absolutely necessary.

The best place for storing spilled values is on the system stack (also often
called \"frame stack" or \"call stack") in the function's call frame. This is
due to the same reasons why it is a good place for local variables---each
invocation of a function gets its own locations for storing the values. This
keeps the functions reentrant and for example naturally supports recursion.

Architectures usually also have a dedicated register for the pointer to the
top of the stack, which means that code needing to access the values not
fitting into registers can address their memory locations using relative
addressing with small offsets, also a feature efficiently supported by all
common architectures these days. On the other hand accessing static slots of
memory would pose similar challenges as accessing global variables does (see for
example section~\ref[TODO] for more details).

\label[sec:use-of-spilled]
\seccc Using spilled values

Putting values into memory brings in the problem of using them in instructions.
The generated code used operations involving registers and often instructions
don't allow memory locations to be used everywhere where registers are allowed
to be used. In fact, on processors employing the \"load-store" architecture
(which is one of the signatures of RISC processors), there are only few
instructions for loading values from memory into registers and a few
instructions for storing register contents into memory and no other instruction
can address memory locations. But, the computation of a value still needs to
store it into a register, just like the use of the value needs it to be present
in a register---though between the definition and use, the value can reside in
memory. To achieve this load and store instructions are introduced. These
inserted loads and stores are called {\em spill code}.

What makes spilling beneficial, is that the registers involved in the spill code
(and in the associated definitions and uses of the spilled values) are only used
very locally. Additionally, the registers used for storing and loading are
essentially completely independent, because each load and store can use a
different register. This makes register allocation much easier
(or even possible), since this essentially introduces new and much less
\"constrained" virtual registers.

%Involving memory in places where access to registers was presumed means adding
%new instructions that perform the loads and store to the memory, these
%instructions are called {\em spill code}. This code is necessary, since many
%instructions require at least one of their operands to be in a register. RISC
%architectures usually employ so called \"load-store" model, where there are only
%a few instructions that are able to load data from memory into a register or
%store register contents into memory. Any computations involving operations on
%spilled values have to still use registers and in order to use a spilled value
%as an operand, it 

%Indeed
%especially on modern CPU architectures using memory instead of registers has big
%negative performance implications.
%So not only is use of registers a must due to 


%One of the
%most common approaches of using memory instead of registers is called 

%Storing some values into memory and then fetching them back helps by making
%more registers available.

%The solution is to put the values into memory. Resorting
%to putting the values to memory is called \"spilling".
%The system stack (also often called \"frame stack" or \"call stack") is best suited for spilled
%values, much for the same reasons it is the best place for storing local
%variables: recursion is handled naturally and well, since each {\em invocation}
%of a function gets its own locations for storing the values---by storing the
%values on the stack, we allow functions to be {\em reentrant}.

%As established, we will refer to spilled values through relative addressing
%(using either the stack pointer or the \"base" pointer, also often called the
%\"frame pointer", see TODO). But we want to use the values in computations and
%processors (mostly) operate on registers. But processors mainly support
%using registers for computations, not locations in memory. This is especially
%true 
%
%RISC processors usually employ a \"load-store" architecture. There are only a
%few dedicated instructions for reading values from memory into registers
%(\"load" instructions) and writing register contents into memory (\"store" instructions).
%All other instructions operate only with registers. This means that to even have
%a value to spill, it has to be in a register! Also only from that register, we
%can store the value to memory and later retrieve it, but again only into a
%register. The important observation here is, that resorting to storing the
%values in memory doesn't in general mean that we get around of doing register
%allocation. What we gain is that instead of having to store a value in a
%register for a significant part of the code, by putting the value into memory and
%retrieving it immediately before each use, we have made the register allocation
%problem simpler---a register is needed in {\em more} but {\em smaller} parts of
%the code. This means that a single machine register can be reused for virtual
%registers instead of being blocked by one.

%At least the register used for the store, doesn't have to be the same one used
%for the load. So even though we are constrained by the fact that registers have
%to be used for spills, we are not too constrained in choosing the registers for
%performing spills. Loads and stores inserted for handling spills are usually
%called the {\em spill code}.

In this text, we care especially about the x86-64 architecture, which, like most
CISC architectures, doesn't have a load-store architecture. There are
for example instructions, which can perform arithmetic directly on locations in memory.
Though at most one operand of an instruction can be a location in memory, the
other one has to be either a register or an immediate value. So on one hand, the
problem of having to use registers for storing/retrieving spilled values
remains, but on the other hand, since one operand can be a location in memory,
we can take advantage of that and not use an intermediate register at all.

Even though the x86-64 architecture allows
the instructions to operate on memory operands, an implementation of the
architecture in for example some new Intel processors splits these instructions
into micro-operations based on the load-store architecture, using internal
{\em architectural} registers (not accessible directly) for storing the
intermediate values. Because of this there probably isn't any {\em direct}
performance difference of using the memory operands. But it is still very useful
to use these more complex instructions---not only can the instruction encoding
be shorter (thus sparing the instruction cache of the processor), but we take
advantage of the internal architectural registers, that we normally wouldn't
have access to, which can mean less constraints on the use of the ordinary
{\em physical registers} that we have access to, which may allow storing more
values into registers instead of memory and hence have a significant {\em
indirect} impact on performance.

For example, suppose that `t3` needs to be spilled in the following:

\begtt
add t3, t2
\endtt

The straightforward solution is to add a load before and store after:

\begtt
mov t4, [rbp+s3] // s3 = an offset to t3's spill stack slot (immediate integer)
add t4, t2
mov [rbp+s3], t4
\endtt

We have to be careful about actually inserting loads and stores, because `t3` is
both {\em used} and {\em defined} in this instruction---the instruction
essentially does `t3 := t3 + t2`. Because of this, the register used for loading
`t3` from memory is the same one that will hold the result that needs to be
stored back into memory, so the store needs to use the same virtual register
the load does, here it as `t4`.

This example shows, that as mentioned, at least with a very narrow local
view, and with the first straightforward solution, spilling `t3` doesn't help with the
use of registers. We introduced another pseudoregister, `t4`, to substitute
`t3`, but the original instruction just became surrounded by memory operations.
Indeed, spilling helps only in a broader scope, where for example `t3` had more
definitions and uses.

We can alternatively just operate on the memory location:

\begtt
add [rbp+s3], t2
\endtt

This seems beneficial, since the processor will likely do the same three fetch modify
write operations we had with with our own spill code, it will use an
architectural register to do so, and we don't need any physical register
(represented by above by `t4` virtual register) to do so. But we shall look at
this in bigger context than a single instruction. The x86-64 two address code is
often generated from three address code (likely from SSA form), for that the
code generator likely had to introduce a copy to preserve the value of the first
operand\fnote{Technically, the lowering step doesn't have to generate the copy
to preserve the value if it is not needed. But, as mentioned further, this
requires non-trivial liveness analysis and eliding (coalescing) the copies is
what register allocators try to be good at, so emitting the copy unconditionally
is reasonable solution.}, so in fact assuming that the original IR was:

\begtt
v3 = add v1, v2
\endtt

the full x86-64 code would be:

\begtt
mov t3, t1
add t3, t2
\endtt

Now the prospect of naively spilling `t3` seems even worse:

\begtt
mov [rbp+s3], t1
mov t4, [rbp+s3]
add t4, t2
mov [rbp+s3], t4
\endtt

The code generator hoped that by assigning `t1` and `t3` the same register, the
move instruction could be eliminated. Since for some reason `t3` was spilled,
that is now out of the question, but there is still a suboptimality---`t1` is
copied to memory and then immediately loaded back again into `t4`, because it
needs to be used in the `add` instruction and then also stored back into memory.
Use of `t3`'s memory location as the first operand of the `add` instruction
helps:

\begtt
mov [rbp+s3], t1
add [rbp+s3], t2
\endtt

But the issue is just hidden---the memory operations are still
there, the CPU still has to load the value `t3` from memory in the `add`
instruction, when we just had it in a register. Just because the addressing
modes of x86 allow use of memory locations in the instruction encodings, doesn't
mean that internally the ALU (Arithmetical logical unit) can suddenly operate
directly on memory, the CPU still has to internally load the value into a
register. So while we spare a general purpose register and instead use an
architectural one, we still excessively operate on memory.

Alternatively, starting from the straightforward 4 instruction sequence, we can
just forward the load and use `t1` to populate `t4` directly:

\begtt
mov [rbp+s3], t1
mov t4, t1
add t4, t2
mov [rbp+s3], t4
\endtt

Now the dead store to `[rbp+s3]` is even more apparent, and can be optimized
away:

\begtt
mov t4, t1
add t4, t2
mov [rbp+s3], t4
\endtt

Now, we keep the values in registers the whole time, and only store the final
result in memory on assumption that later the value is needed and storing it in
memory somehow helps the register allocator. While the first example is two
instructions, it is X bytes, while the second one is 3 instructions and Y bytes
TODO: compare lengths of instructions.

Another important observation is that, the \"optimal" code we ended up with, is
very similar to what we started with---there is just a store instruction in the
end. Whether spilling `t3` helped is impossible to tell from this context. But
we can something about the possible benefits of spilling `t4`---there are none.
By spilling `t4` we would get the same instruction we already have, just with a
different pseudoregister (say `t5`). We are at the `t3` spilling {\em fixed
point}. For this reason we should prevent the register allocator from thinking
that spilling `t4` might be a good idea, since it might lead it to an infinite
loop. The reason why spilling `t4` is not beneficial is the fact that the potentially
inserted loads and stores are redundant. The more general pattern is, that any
definition immediately followed by use is not a possible spill candidate. This includes
virtual registers inserted for spill code (so prevents potential infinite loops
spilling spill registers), but can also prevent spilling other virtual
registers, which otherwise might look like plausible spill targets, but in fact
aren't.

A different point of view is, that essentially, in (the last snippet)
in the first two instructions `t3` is represented by `t4`, while in the last
instruction it shifts back to being represented by `t3`, which due to some
previous decision, resides in a memory location `[rbp+s3]`. It is as if we have
originally split the register `t3` into multiple registers (`t3` and `t4`
connected by moves) and only spilled `t3`. It results in the same great code as
our optimized spilled version improved the generated code, because some of the
multiple registers can be much less constrained than the original one and are
thus less likely to be spilled and more likely to get assigned a physical
register.

In contrast, we argued, that by merging `t4` and `t1` we would have had the
chance to eliminate the move instruction, thus improving the code as well. Both
"merging" (called {\em coalescing}) and \"splitting" (called {\em live range
splitting}) can improve the code in different situations, this makes it
even harder, because recklessly doing one or the other will make the code
certainly worse, while doing neither may be just as bad. This makes great register
allocation even harder.

\seccc Interaction with instruction selection

As we have seen, the process of spilling needs to insert new instructions which
together form the so called "spill code". In simplest scenario they just load
and store the value, but better code can be achieved if the spill code is
inserted with more thought than just load from memory and store to memory. But
both of these problems---selecting the instructions to use for operations and
choosing best ones in the current context is exactly the job of instruction
selection.

In principle, register allocator shouldn't care about the instructions. It
should only care about their effects on the registers. The result of register
allocation should be the assignment of register to virtual registers. Since spills
can be necessary, which requires insertion of new instructions, we have to
decide how this spill code will be handled. The basic options are the following,
and mostly depend on the chosen register allocation technique:

\begitems\style n
* Insert spill code in the register allocator pass.
* Return the list of spilled virtual registers, and expect to be called again with
code transformed to include the spill code.
\enditems

The first option can make the register allocator depend on the target
architecture---it now needs to know about the current target, its instructions,
their meanings and how to insert them. On one hand, register allocation is
already in the "backend", where we expect to handle things on the level of the
target, and generally hope to take advantage of that by, for example, doing
optimizations specific to the particular target. On the other hand, as mentioned
previously, machine independent representations of (machine dependent)
instructions are possible, thus spill code insertion {\em could} also be made
machine independent similarly just like the register allocation.

The second approach also makes the register allocation process pure in a sense. The register
allocator never modifies the input, its result is a either a mapping of
virtual registers to physical registers, or a list of virtual registers to be spilled.
Though it still has to be decided on how to insert the instructions. Some
instruction selection mechanisms which ultimately depend on the middle end IR,
are not suitable for inserting and optimizing spill code, since we are already
in the low level backend IR. Tree or DAG based instruction selection
mechanisms may also not be directly applicable---we may no longer be using
trees or DAGs for representing the machine code, especially after instruction
scheduling, which sets the order of instructions in stone. On the other hand
peephole optimization is a great fit for improving inserted spill code. The
inserted spill code can be very naive, and peephole optimization run on the
spill code and its surroundings can make improvements. This is especially likely
if we are able to find patterns which spilled code creates, such as in the
example in the previous subsection (\ref[sec:use-of-spilled]).

The obvious downside of the \"pure register allocation" approach is, that it has
to start over with register allocation if any spills need to be made. The first
approach seems more suited to register allocation where we want a self
contained fast single pass register allocation.

The approaches used for spill code insertion are usually very connected to the
core principle of the register allocation algorithm at hand, and choices of some
approaches are discussed in section~\ref[sec:regalloc-techniques].

\secc Formalization

The terms used in the previous sections about register allocation were not
properly defined, and used a bit vaguely. The intention was to practically show
the problems register allocation tries to solve and what it needs to do to solve
it, which includes optimizations that try to make the register allocation
process more optimal. In this section, we try to more properly introduce the
terms used for concepts connected to register allocation.

One thing that has to be noted is the name \"register allocation" itself. We
mentioned that register allocation is meant to map virtual registers to
physical registers of the target architecture, but the process isn't always
so direct, and sometimes it makes sense and brings benefits to split this
process into two parts, which really define what we mean by these names:

\begitems\style n
* {\em Register allocation}. In a narrower sense, by register allocation we mean
the process of making sure that each virtual register can be assigned a physical
one. At this point, we may not care too much about which one, but we care about
spill code, because that is what allows us to fit into the limited amount of
registers available.
* {\em Register assignment}. Assignments follows allocation---now that we
can map every virtual register to at least one register, we choose the
concrete one. Although this seems much more simpler than the allocation part,
in practice register assignment is also very important, because we have seen
situations where some assignments lead to better code, for example where the
source and destination of a move instruction are assigned the same register, the
move instruction can be eliminated.
\enditems

Some register allocation algorithms intertwine both parts and don't split them.
Some algorithms strictly separate these concerns. In general simpler algorithms
usually merge both of these parts, while more complex algorithms try to take
advantage of attacking each of those issues separately to reach more optimal
results. But this distinction is of course not definitive.

\seccc Liveness, interference

Already in previous sections we hinted that we may reuse a physical register if
the virtual register occupying has no further use. Liveness is the property that
captures this formally.

A virtual register is {\em live} at a program point if it {\em may} be used in
future.

The definition of liveness captures the \"not used further" aspect that we found
beneficial for register reuse. If at a program point a virtual register stops
being live (becomes {\em dead}) the physical register allocated to it becomes
available. The definition of liveness carefully says \"may be used in future",
because in general it is undecidable whether a virtual register {\em will} be
used. TODO: Rice's theorem, halting problem.

For computing liveness analysis the means of classical data flow analysis may be
used. For each program point we can compute the liveness property on the control
flow edges coming to it ($\hbox{LIVE}_{\hbox{\it in}}$) and control flow edges
coming out of it.

Liveness is the most basic property on which register allocation is based.
Liveness of a virtual register represents whether from a 
Liveness is a property of virtual registers and tells us at which points in the
control flow a virtual register is {\em live}.

\seccc Live ranges

vs value vs variable.

\label[sec:regalloc-coalescing]
\seccc Coalescing

Coalescing in register allocation tries to assign virtual registers a common
physical register. Usually we only care about coalescing where it is beneficial.
In the low level these are the situations which may manifest as copies from one
register to the other---if the two registers are allocated the same physical
register, the copy is redundant and can be optimized out.

In practice this is connected to several high level concepts that generate low
level copies:

\begitems
* Translation from three address code to two address code. In two address code
the first operand is the same as the destination. When translating three address
code to two address code, we need to introduce a copy to not clobber the first
operand in case it {\em lives out}. For example this three address code
instruction:

\begtt
sub t1, t2, t3
\endtt

Can be translated to:

\begtt
mov t3, t1
sub t3, t2
\endtt

* SSA deconstruction (section~\ref[sec:ssa-deconstruction]). Effects of $\phi$
instructions are usually modelled with copies in predecessor blocks. For example
the following instruction in block 3, which merges virtual registers `t1` and
`t2` into virtual register `t3`:

\begtt
phi t3, t1, t2
\endtt

May translate to the following instructions in the two predecessors blocks:

\begtt
// predecessor 1
mov t3, t1

// predecessor 2
mov t3, t2
\endtt

Note that just these copies may be insufficient in some situations, see
section~\ref[sec:ssa-deconstruction].

* Live range splits. A virtual register can be split into multiple virtual
registers, and copies are introduced to connect them. These can include splits
due to register constraints (including calling conventions). See
section~\ref[sec:regalloc-splitting] for more details.

As coalescing is the exact opposite of live range splitting, virtual registers
that are the result of live range splitting should not be carelessly coalesced,
since that would practically undo the splits.

%* Register constraints. Sometimes instructions require their arguments to reside
%in specific registers or put the results into specific registers or for example
%calling conventions can require arguments to be passed in specific registers. If
%our register allocator is not able to do live range splits on demand  we may want to 
%
%* Callee saved registers.
%
%
%This also
%applies to calling conventions which say which arguments are used for passing
%functions, and which are caller-saved or callee-saved.
\enditems

The removal of the move instructions itself is a simple task that can be left to
the peephole optimizer. The real role of coalescing in register allocation is
to try to allocate move related nodes to the same register. But too much
coalescing can be counterproductive---choosing to allocate two virtual registers
to the same physical one means that effectively the registers are combined into
one. This means that their interferences add together and it becomes much harder
to find a common register.

\label[sec:regalloc-splitting]
\seccc Live range splitting

Live range splitting is essentially the opposite of coalescing. By splitting
one virtual register into multiple, we hope to achieve better register
allocation, since the split virtual registers can be allocated separately and
they can be much less constrained (have less interferences) than the original
virtual register.

A big disadvantage of our model of register allocation being a mapping from
virtual register to physical registers is that a virtual register can be
assigned only one physical register. Live range splitting essentially remedies
it, because the splits are allocated separately.

Another advantage is that, in the simple spilling implementation, every use and
definition of a register is replaced accordingly by a load or store.

Naive spilling of an entire virtual register is simple to implement, but doesn't
consider the nature of the register's use. For example the virtual register may
see significant use in the beginning of the procedure and then in the end of the
procedure and not be used much otherwise. Such virtual register will interfere
with practically all other virtual registers, since it is live during the whole
function. Splitting it to three registers (beginning, middle and end of the
procedure) not only allows different physical registers to be allocated to it in
the different parts, but it also allows {\em independent} spill decisions. For
example, it can be spilled during the middle of the function cheaply with only
one load and one store. The splits here would mean splitting say `t10`:

\begtt
...
mov t10, ...
...,  t10

[...] // a loop

..., t10
\endtt

To `t11`, `t12` and `t13`:

\begtt
...
mov t11, ...
...,  t11

mov t12, t11
[...] // a loop
mov t13, t12

..., t12
\endtt

Splits can be especially beneficial around loops, since spills of any values
used inside loops are costly. Even a lot of shuffling of registers and memory
can be beneficial if a loop is executed many times. Moving expensive operations
out of the loops has been the task of code motion in the middle-end and
instruction scheduling in the backend. Since register allocation generally
follows them, we should do our best to not hinder it.

If the splits turn out to be unnecessary, they can be coalesced away. Though
introducing too many splits can reduce the chance of coalescing in
general---coalescing has to be careful about splits in order to not undo them
carelessly, additionally many register allocation techniques avoid excessive
coalescing.

Spilling can be seen as a very primitive form of live range splitting, which
introduces a new virtual register for each use and definition and spills all of
them. Splitting the uses and definitions manually, and spilling them only when
needed may prove to be better, since even in case the copies across the splits
don't get coalesced, they are presumably cheaper than touching the memory.

Some allocators are able to rethink the register allocation of a virtual
register at its every use or definition. This means that the algorithm can
essentially perform live range splitting everywhere. Other algorithms are not
able to do splitting \"online" and require it to be done as a preprocessing
step, in that case live range splitting can be even more important.

\label[sec:regalloc-constraints]
\seccc Register constraints

Some instructions require the arguments to be in particular registers or produce
values in particular registers. Typically, there are two such constraints:

\begitems\style n
* One concrete register has to be used.
* One register belonging to a particular class can be used.
\enditems

Register classes are discussed in section~\ref[sec:regalloc-classes]. Here we
discuss only constraints which require one particular register.

The x86-64 architecture has two notable examples where concrete registers have
to be used for operands or results:

\begitems\style n
* {\em Shifts}. Shift instructions where the shift amount is not an immediate
value, require the shift amount to be in the `cl` register (that is, the low 8
bits of the `rcx` register).
* {\em Long multiplication and division}. On x86-64 the only available
division instructions (for signed and unsigned division) divide a 128-bit number
by a 64-bit value.%
%
\fnote{As with other instructions on
x86-64, the 32-bit variants are available. So for example, the long division
divides 64-bit number by a 32-bit number, but it still operates on two registers
(`eax` and `edx`).}
Upper 64-bits of the 128-bit dividend are expected in `rdx`,
the lower 64-bits in `rax`. The division instructions store the remainder in
`rdx` and quotient in `rax`. The divisor can be a (64-bit) register or memory
location.
\enditems

But there are restrictions imposed not necessarily to the instruction set
itself, but by {\em calling conventions} of the platform. For example on x86-64
Linux the System V ABI prescribes that the registers `rdi`, `rsi`, `rdx`, `rcx`,
`r8` and `r9` are used for passing parameters and `rax` and `rdx` are used for
return values.

Register allocator needs to conform to these requirements. But if the physical
registers are allocated for the entire lifetime of the constrained virtual
registers, then there can be conflicts if there are multiple such constrained
contexts and the virtual registers interfere. For example, if there are the
x86-64 shift instructions:

\begtt
sal t1, t2 // t2 needs to be allocated to cl
sar t3, t4 // t4 needs to be allocated to cl
\endtt

Here `t2` and `t4` both need to be allocated to `cl`, but they are live at the
same time (interfere), so they cannot be assigned the same register. This is the
case regardless of the number of available registers, so this is a problem of
the {\em register assignment} part of register allocation. Spilling can save the
situation, for example spill of `t4` helps if `t2` isn't needed after this
snippet of code:

\begtt
sal t1, t2 // t2 = cl
mov t5, [rbp+s4]
sar t3, t5 // t5 = cl
\endtt

Spilling both of course also helps, but spill of `t2` alone doesn't, because the
newly introduced temporary (`t5`, constrained to `cl`) would still be alive at the
same as `t4` and thus they would interfere:

\begtt
mov t5, [rbp+s2]
sal t1, t5 // t5 = cl
sar t3, t4 // t4 = cl
\endtt

The problem with spilling here is, that it is mainly used as means of reducing
register pressure. Because of that spilling heuristics are meant to spill
virtual registers that are e.g. not going to be used for the \"longest" (which
frees a physical register for the longest time) or which interfere with a lot of
other virtual registers (so the other virtual registers have much higher chance
of being allocated themselves). Nothing usually makes constrained registers good
spill candidates, and it shouldn't, since because of the constraints the virtual
registers {\em need} to be assigned.

Live range splitting is a much better choice for handling constrained registers.
By introducing a new virtual register (and a copy to it) for the short time of
the constrained use, the constrained use doesn't have any chance of interfering
with other constrained uses and is so short lived that it even isn't a plausible
spill target (see section~\ref[sec:regalloc-spilling]). In our example, it looks
like this:

\begtt
mov t5, t2
sal t1, t5 // t5 = cl
mov t6, t4
sar t3, t6 // t6 = cl
\endtt

Here indeed `t5` and `t6` don't interfere and are unspillable. If, like we
assumed earlier, `t2` doesn't {\em live-out}, then the best assignment would
assign `t5` the `cl` register (low 8 bits of `rcx`), like this:

\begtt
mov rcx, rcx
sal rax, cl
mov rcx, rdx
sar rbx, cl
\endtt

Here it was possible to keep all values in registers. The first copy can be
easily optimized by subsequent peephole optimization pass.

Live range splitting is good solution for making register constraints not
constrain the register assignment much. The same ideas that apply to live range
splitting apply also here, in particular allocators which are unable to perform
live range splitting on demand should split constrained uses beforehand and
coalescing may be able to remove the copies if it turns out they are not needed.

Calling conventions also dictate the coordination of registers between the {\em
caller} (calling function) and the {\em callee}. Both want to use machine
registers (and ideally all of them), but if callee uses the registers, it
overwrites values the caller has stored. For that purpose registers are
classified as either caller-saved (caller has to save the registers, if it uses
them) or callee-saved (callee has to save the registers, if it wants to use
them).

The fact that a register is caller saved means that when a function performs a
call, it has to pessimistically assume that the callee changes all the caller
saved registers.\fnote{The callee doesn't necessarily change any of the caller
saved registers, but it is allowed to do so. Calling conventions are general,
and don't try to specialize for some cases. They are the interface functions
need to conform to, and allow code produced by different compilers to cooperate
together. But if a compiler is sure that the function doesn't have to conform to
the interface (perphaps because the function is `static` and thus not callable
by from other separately compiled modules), then it can can try to do better by
employing whole module register allocation, which considers register allocation
even across the calls. Often though a much simpler technique helps with calling
convention constraints---not performing any calls at all! Inlining the called
function into the call site means that the registers used be the called function
are allocated as part of the function procedure and the calling convention
constraints don't apply at all. Leaf functions (functions not calling other
functions) are especially good candidates for inlining especially because they
don't impose calling convention constraints themselves.}
This can be modelled like an register constraint on the call instruction. For
example on x86-64, where for example `rax`, `rcx` and `r11` are caller saved we
can make the `call` instruction {\em define} the registers:

\begtt
call f % defines rax, rcx, r11 and others
\endtt

Coincidentally `rax` is at the same time used for passing the return value, so
it would have been defined by the call instruction already. But `rcx` for
example is used for parameter passing and is thus as mentioned above {\em used}
by the call instruction to model that. Adding definition of `rcx` to the call
means, that the caller shouldn't expect the `rcx` register to be preserved by
the caller. While this model works well for {\em correct allocation}, a lot of
registers are caller saved and virtual registers that {\em live through} the
call can not be allocated to them. Callee saved registers are needed in that
situation. But for example on x86-64 there are 9 caller saved registers and only
7 are callee saved, and out of those 2 are usually reserved for special purposes
(stack and base pointer). Having only a few available registers for values
living across calls means very high register pressure, which will have to be
mitigated by spills. Spills are correct and needed here, since if the registers
are reserved for the caller, we don't have any other choice for storing values
than memory. But naively spilling virtual registers everywhere just because of
high register pressure at a call site is not ideal. Once again this is a place
where live range splitting helps---just like splits around loops were useful,
splits around calls can be useful for minimizing damage implied by spilling
virtual registers at every use or definition.

Callee saved registers can be modelled through uses and definitions as well.
Having definitions of callee saved registers at the entry point of a function
and uses at the exit point (return instruction) models the fact, that the callee
saved register has to be preserved for the entire duration of call. This not
only requires physical registers to be somehow represented, but also blocks the
callee saved register for the entire duration of the function. Live range
splitting is again very useful here, because we can introduce virtual registers
for holding the values in callee saved registers during the function, e.g. if we
consider just `rbx` and `r12`:

\begtt
mov t20, rbx
mov t21, r12
[...]
mov rbx, t20
mov r12, t21
\endtt

This is much better, since the virtual registers can be spilled, which frees up
a callee saved physical registers, which can possibly be used by many (short
lived) virtual registers or perhaps just one virtual register with more uses and
definitions which would be more expensive to spill. The virtual
registers introduced for callee saved registers are in fact ideal spill
targets---there is only one definition and one use, both outside of any loop.
Spilling them could look like this:

\begtt
mov [rbp+s20], rbx
mov [rbp+s21], r12
[...]
mov rbx, [rbp+s20]
mov r12, [rbp+s21]
\endtt

Here we refer to some abstract \"stack slots". If the stack slots are chosen
well, `push` and `pop` instructions can even be used for realizing those spills:

\begtt
push rbx
push r12
[...]
pop r12
pop rbx
\endtt

Which essentially gets us the code one would use to free up callee saved
registers. But modelling it through register constraints can nicely take care of
only using the callee saved registers when beneficial and so it is more
flexible.

\label[sec:regalloc-classes]
\seccc Register classes

TODO

\label[sec:regalloc-techniques]
\secc Techniques

We have already seen a few things that can distinguish different register
allocation algorithms:

\begitems
* Handling of spilling (section~\ref[sec:use-of-spilled]),
* Split or no split of allocation and assignment
(section~\ref[ref:regalloc-formalization]).
\enditems

\noindent But there are others:

\begitems
* Scope: {\em local} vs {\em global} vs {\em interprocedural} vs {\em whole
program} algorithms. Local algorithms operate on singular basic blocks and use
only information local to the basic block to decide on register allocation. The
limited scope makes the algorithms generally simpler and produces worse results
then global register allocation, which allocates registers to whole functions.
Global register allocation is global in the sense that {\em all} basic blocks
are considered at the same time. The analysis is more complex, since it has to
handle control flow. Even techniques for allocating registers across function
calls and whole programs exist. These can be less practical in practice, where
functions may be required to conform to a {\em calling convention}, which
specifies how arguments should be passed in function calls, what registers are
preserved by calls and where will the return values reside. We will not discuss
techniques operating in larger scopes than {\em global} (whole function, all
basic blocks).

* {\em Quality} vs {\em speed}. With no restrictions on time, we ideally would
like to achieve {\em optimal} register allocation. With the right definition of
optimal, it can be possible, but due to the difficulty of the register
allocation, this approach is bound to be too slow (in {\em compile-time}),
although it would produce code that would be fast (in {\em run-time}). In code
compiled ahead of time, we can probably justify spending more time on
compilation to achieve better run-time, since it is expected, that the program
will run for some time and that the investment will return.

On the other end of the spectre we may want a register allocation algorithm that
runs very fast (due to constraints on compile-time), but in that case we can't
expect good results (i.e. code that has fast run-time). This can be interesting
for {\em Just-in-time} (JIT) compilers, where the compile time is part of
run-time and hence it is not possible to spend much time on optimizations,
because it is possible that they wouldn't pay off (though they could, we don't
know ahead of time).

* {\em Control flow sensitivity}. Some global algorithms may completely
disregard the actual control flow of the program and just use (global) liveness
and/or interferences to do register allocation and assignment. But use of
control flow information in an algorithm is likely to steer it to better
results---spills in hot or nested loops are undesirable. Control flow sensitive
register allocation (not assignment) may for example even try to purposefully do
spills or splits before loops to make more registers available in loops.
\enditems

\label[sec:regalloc-top-down]
\seccc Local top-down and bottom-up

Two of the most basic algorithms for register allocation are described by Cooper
and Torczon~\cite[Engineering]. They operate only on single basic blocks, but
form a good baseline to improve upon. Also, surprisingly, their ideas of how to
handle spilling have their equivalents in more powerful algorithms.

Both algorithms investigate uses and definitions of virtual registers inside a
basic block, allocate some map some virtual registers to physical registers and
spill the others. They differ in how they choose which virtual registers to
spill:

\begitems
* {\em Top-down.} In the top-down view those virtual registers which are used
most often (i.e. their number of uses and definitions is the highest) should be
the ones that get assigned registers, others should be spilled.

While this is simple to implement, there are glaring problems. The algorithm is
not able to reuse registers---since it maps virtual registers to physical
registers one-to-one, it is not able to reuse a physical register once a virtual
register becomes dead.

Also, most instructions require at least some arguments to reside in registers,
but it can happen that top-down register allocator spills all used and defined
virtual registers of a particular instruction. To solve this, sufficient number
of registers has to be set aside and not be allocated, they will be used for
realizing loads of uses and stores of definitions of spilled virtual registers.
This of course makes the allocation results even worse, because only a lesser
number of registers are available for allocation, and spills may be
introduced just because some registers are reserved for spill code realization.

* {\em Bottom-up.} In the bottom-up approach instructions are investigated in
order, one by one, and registers are allocated to supply the demand of each
particular instruction. In general, each instruction is an operation with
multiple input registers and multiple output registers. Hence for each
instruction the algorithm ensures that input virtual registers are allocated
into physical registers and allocates registers for output virtual registers.

It may seem, that since the algorithm operates on a single basic block, that each
use of an virtual registers should have a preceding definition, which should be
the one, which allocates register for it, and that allocation of registers for
arguments is not necessary. But it is necessary for handling spills---allocation
of a physical register (whether for input or output virtual register) may find
that none of the registers is free, so it has to choose one of the assigned
registers, and spill it, by moving the value from that register to memory. Later
the spilled register may be needed again, and so it has to be assigned register
again and the previous value has to be reloaded from memory into the new
register. The newly allocated register doesn't have to be the old one. This is a
great advantage of this approach over {\em top-down}, while spilling, it is able
to effectively split a live range and allocate it different physical registers
or memory locations.

Because the algorithm considers each instruction, it is able to much better deal
with machine constraints. For example, if an instruction needs its operand to
reside in a particular register, or puts the result in a particular register,
then the allocator may just forcibly allocate that particular register. One
example of such constraint are that of the shift instructions on x86-64, which
require the shift size to be specified in the `cl` register:

\begtt
shl t1, t2 % t2 has to be allocated to cl
\endtt

If we suppose that `t1` already resides in a register (say `rax`), and `t2` is
already in `rcx` (the register of which `cl` is the lowest 8 bits), then the
allocator doesn't have to do anything:

\begtt
shl rax, cl % t2 has to be allocated to cl
\endtt

If `t2` is assigned say the `rdx` register, and the value in the `rcx` register
is no longer needed, just a copy is sufficient:

\begtt
mov rcx, rdx
shl rax, cl
\endtt

If however, the virtual register which occupies `rcx` (say `t3`) is live after
the instruction, then we need to find it a new register. And if conveniently
`t2` (the shift amount) is not needed after the shift, then we can just reuse
the newly freed `rdx` register by swapping the registers:

\begtt
xchg rcx, rdx
shl rax, cl
\endtt

And so on. Similarly we could deal with platform calling conventions. For
example, if the called function needs arguments in registers `rdi` and `rsi`,
then we might just forcibly allocate them. If the function call doesn't preserve
other registers (like `rax` or `rcx`), these registers should also be forcibly
allocated---though the call-site will not use them for anything, the called
function might, and hence we need to preserve values in them, and using the
\"allocate a register" mechanism, we elegantly also handle the necessary spills.

When a physical register is needed and none is available, one has to be spilled.
Good choice is to spill the virtual register whose next use is the furthest
away~\cite[Engineering]. This is akin to Bélády's MIN algorithm for page
replacement~\cite[Belady1966]. The benefit of the approach is, that if we need
to spill, the register we free up will be available for other purposes for the
longest time, hence it will hopefully prevent other spills.

The bottom-up algorithm has to do two passes over the code---first one to derive
liveness information, second one to actually do the allocation. Liveness
information provides the information necessary for choosing spills.
\enditems

An interesting twist to the bottom-up algorithm described by Mike
Pall~\cite[Pall2009]. He does the allocation in a single pass over the code in
SSA form, though in {\em reverse}. In programs in SSA form SSA values naturally
correspond to live ranges, and reverse order is natural for computing liveness.
Pall's algorithm essentially combines register allocation with the liveness
computation. While iterating in reverse order definitions are processed first,
while uses are processed next, contrary to the bottom-up algorithm. Also
contrary to the bottom-up algorithm, where {\em definitions} was what derived
the assignment, it is the {\em uses} that drive the assignment in this
algorithm. When a first use of a virtual register encountered, it is allocated a
register, and when (the only) definition of a virtual register is reached, then
it is freed. This is often better, since more often the uses are constrained by
machine constraints, so by discovering the uses first, the allocation can be
targeted more easily.

The problem with all these three approaches is that they are too local. Their versions
as presented above work only in a single basic block. Extensions to global
(whole-procedure) allocation are possible by using memory---a simple extension
does register allocation on each basic block separately and all virtual
registers are stored at the end of each block and loaded back at start of a
each block. This still only requires only local analysis, but produces very inefficient
code. It is possible to improve this by performing global liveness analysis and
to store only virtual registers that live-out and to load only live-in virtual
registers. Though at the point where global analysis is feasible, some of the
global register allocation algorithms is probably feasible as well.

\label[sec:regalloc-linear-scan]
\seccc Linear scan

Poletto and Sarkar~\cite[Poletto1999] introduced o called {\em linear scan} register
allocation. It can be seen as an extension of the bottom-up approach described
in the previous section~(\ref[sec:regalloc-top-down]). The canonical version of
the bottom-up is able to allocate registers only for a single basic block,
because it depends on many of the linear aspects of basic blocks, such as that
the instructions are ordered, live ranges are also ordered and it can be
determined which is \"furthest away", so that something akin to Bélády's
algorithm~\cite[Belady1966] can be used. Linear scan register allocation extends
the bottom-up approach to perform global (whole procedure) register allocation
by imposing an ordering over the instructions by (globally) numbering them. This
ordering induces a linear sequence, where live ranges can be represented as simple
intervals starting at the number of the first instruction where the virtual register
is live and ending at the number of the last instruction where the virtual
register is live. This essentially makes the procedure into a single \"basic
block" on which something akin to the bottom-up allocator can be run.

But the algorithm described by Poletto instead operates on the live intervals.
The algorithm orders the intervals by increasing start point and iterates over
them. The algorithm effectively iterates over the starts of live ranges, keeping
the set of {\em active} intervals (those whose start is {\em before} the start
of the current interval, and end {\em after} the current interval). For each
encountered interval, those intervals in the active set which end before the
current interval's start are expired and their registers freed, and a new
register is a allocated from the pool of free registers for the current live
interval. If the number of intervals live at some point exceeds the number of
available registers a live range needs to be spilled. Poletto and Sarkar choose
to spill the live range which ends furthest away---if that live range is the one
which is currently being allocated, it is not allocated a register, but a memory
location instead, otherwise the current live range is assigned the register of
the spilled interval and the spilled interval is assigned a memory location.

There are many problems with linear scan. It has been designed as a fast and
simple alternative to graph coloring register allocators
(see~\ref[sec:regalloc-graph-coloring]), mainly for JIT compilers which value
greatly run-time of {\em the compiler} and can sacrifice run-time of the
compiled code. The relatively poor allocation quality is intentional.

The reasons for the poor quality is that live ranges are really really
imprecise, since this simple live ranges don't represent live ranges
truthfully---there may be many instructions in the middle of the interval, where
the virtual register is not live. In fact trivial live range $[1, n]$, where $n$
is the number of instructions is correct for each virtual register, but of
course produces unsatisfactory allocations. Another problem is, that unlike the
bottom-up approach linear scan has more trouble with handling of spilled code as
well as machine constraints. This is because virtual registers (live intervals)
are assigned either a register for their entire duration, or a memory location.
For use of the memory locations in instructions registers have to be used, and a
few registers would have to be set aside for that (like with the top-down
allocator from section~\ref[sec:regalloc-top-down]). Basic form of linear scan
doesn't handle machine constrains at all 

While in some sense linear scan register allocation can be seen as a extension
of the bottom-up register allocator, it suffers from many of the issues of the
top-down allocator. Some of the follow ups on linear scan are much better suited
to practice. For example in~\cite[Traub1998] they are able to additionally deal
with holes in live intervals and can assign a multiple registers to a single
live range (at different times) and other approaches are able to better handle
machine constraints and use properties of SSA form~\cite[Mossenbock2002,
Wimmer2010].

\label[sec:regalloc-graph-coloring]
\seccc Graph coloring

Even though the idea of using graph coloring for register allocation is older,
first notable use of the technique is by Chaitin~\cite[Chaitin1981, Chaitin1982].
The core of the idea is to construct an interference graph from the
interferences of virtual registers---all virtual registers become nodes in a
graph and there is an edge between virtual registers if and only if they
interfere. Then on this graph we aim to find a {\em coloring}---mapping of nodes
to colors, such that no neighbouring nodes get the same color. In our case the
colors ultimately constitute the machine registers. Because edges designate
interference, it is guaranteed that no virtual registers that interfere are
assigned the same physical register. If we have $k$ machine registers available,
then we are looking for a $k$ coloring---the coloring needs to use at most $k$
colors (registers).

By reducing the register allocation to graph coloring we may seemingly not
gain much, since graph coloring is an NP-complete
problem\fnote{In~\cite[Chaitin1981] authors argue further that register
allocation is also NP-complete, this has since been
disputed~\cite[Bouchez2007].}. However in~\cite[Chaitin1981] they rediscover a
technique that can simplify and make graph coloring practical for register
allocation. It is based on the observation that a node which has fewer than $k$
neighbours can be always assigned a color distinct from all of its neighbours.
Because the node has less neighbours than there are available colors, even if
all neighbours used different colors, there would still be a free color left.
This simple, yet important observation is the base for their and derived
techniques.

Since incorporating a node with degree (number of adjacent nodes) less than $k$
(a so called {\em insignificant}
node) into an already colored graph is easy, initially we do the
opposite---remove from the graph all insignificant nodes, such that later, in
the reverse process we can add them back to the graph and color them trivially.
Removing low degree nodes from the graph causes the degrees of neighbouring
nodes to decrease as well and may thus lead to more simplifications. In
Chaitin's algorithm this phase is called {\em simplify} and the
removed low degree nodes are pushed onto a stack. In the final stage, called {\em
assign}, the nodes have colors assigned in the reverse order simply by popping
them from the stack and assigning them a color not used by any of the already
colored neighbours in the now being rebuilt graph. Use of this heuristic is not
all saving---it is possible that after after simplification (removal of low
degree nodes) there will still be high degree nodes left. In that moment, push
of any of the remaining nodes on to the stack, could mean that there won't be a
color left for it. Chaitin's solution is to calculate spill costs of all the
remaining high degree nodes, choose the one with the lowest cost, mark it as
to be spilled and remove it from the graph. Due to the removal, the
simplification process may find more simplifications, otherwise another spill
decisions may be made. The removal of the to be spilled node from the graph
simulates its replacement by loads and stores, which although will introduce new
virtual registers, they will have very short live ranges, with (hopefully) much
less interferences, so it suffices as an approximation.

Spill of any node means that the code needs to be updated with spill code, and
the register allocation process repeated. Since the program is now different,
and new pseudoregisters were introduced to accommodate spill code,
liveness analysis and building of interference graph have to be repeated as
well. Then simplification can be tried again. This entire process is tried
until the simplification is able to reduce the graph to an empty graph, which is
trivially colorable. Since each iteration is very expensive, it is important
that there can be multiple spill decisions made in a single {\em simplify} run,
this way the process can often finish in 1 or 2 iterations, if the first
iteration successfully finds all nodes that need to be spilled and the second
iteration finalizes the assignment.
Being able to spill only one node on each iteration would mean that graphs with
many high degree nodes would need {\em many} expensive iterations to finish.
Proceeding from {\em simplify} only after all spill have been handled makes it
possible to push only low degree nodes, guaranteeing that in the reverse {\em
assign} stage, every node popped from top of the stack will have at least one
free color.

But, it is possible to do better. The fact, that a node has a {\em significant}
number of neighbours doesn't mean, that it will be uncolorable in the {\em
assignment} stage. There is a chance that the already colored neighbours will be
assigned less than $k$ distinct colors (TODO: figure), in that case the popped
node could still be colored even though at the time it was pushed it was
significant. Because of this, in {\em simplify} we may optimistically try to
remove and push high degree nodes on to the stack, instead of pessimistically
spilling them. If in the assignment phase it turns out that there isn't a color
left for the popped node, we spill it only then. We call these pushed high
degree nodes {\em potential spills}, since they become {\em actual spills}
spilled in the {\em assign} stage. This strategy is called {\em optimistic
coloring} and was devised by Briggs~\cite[Briggs1992, Briggs1994]. Even in this
strategy, it can happen that more than one (potential) spill will be necessary.
Like with Chaitin's \"pessimistic" spilling, the best possible node for
potential spill is the one with the lowest spill cost---first potential spill
will be processed last in the {\em assign} stage, and will encounter a more
complete interference graph, than potential spills pushed later, which are
assigned colors earlier. Like before, we want to capture all {\em actual}
spills, before we repeat the whole process with spill code inserted. To do this,
in the {\em assignment} stage we don't allocate the actual spills any color,
just mark them for spilling and proceed. This can make more some nodes
neighbours of actual spills colorable, since by not coloring the actual spill,
they effectively have one less interfering node. Like with Chaitin's spilling in
{\em simplify} stage, this approximates the actual effect of spilling, which
splits a single node into many temporaries whose interferences are more local
and hopefully easier to deal with.

As we have seen before (in section~\ref[sec:regalloc-spilling]), spilling can always be
necessary, reducing the register allocation problem to graph coloring doesn't
change that. No matter how $k$ is big, there are always graphs which need more
registers to be colored successfully. Even an exact graph coloring algorithm
that tries all possibilities can fail to find coloring because of this. Spilling
is thus {\em not} only due to Chaitin's heuristic. Though the heuristic even
with Briggs' optimistic coloring can introduce more spills than a more exact
algorithm would.

The interference graph is a really great data structure, because apart from
being able to represent the \"live at the same time, and thus unallocatable to
the same register" constraints, it can express also other restrictions. Machine
constraints and calling
conventions can both be
modelled by interferences (edges in the interference graph) if we also add
physical registers as nodes. For example, we can force a virtual register node
to be allocated to a particular physical register by making it interfere with
nodes corresponding to all the other physical registers. This is often called
{\em precoloring}. In fact, since we need to be careful about not accidentally
allocating physical register a different physical register, we need to make all physical
registers interfere with each other, this way they are all guaranteed to be
allocated their color (register). But these additional constrains can lead
to uncolorable (\"overconstrained") graphs if the live ranges of precolored
registers are too long. Since graph coloring maps each virtual register to a
single physical register, it needs the precolored live ranges to be short and
non-interfering, which can be done with {\em live range splitting} (see
section~\ref[sec:regalloc-splitting]).

But avoiding uncolorable graphs with splits means a lot of copy instructions,
which if allocated different registers, will not be optimized by peephole
optimization and thus can incur significant unnecessary overhead. This increases
the need for {\em coalescing} (see section~\ref[sec:regalloc-coalescing]).
Chaitin already realized the need for coalescing. His
solution~\cite[Chaitin1981] was to
coalesce every {\em copy-related}, {\em non-interfering} pair of virtual
registers in a pass called {\em coalesce}, before simplification. A pair of
virtual registers `t1` and `t2` is copy related when there is a copy (move)
instruction between them (i.e. `mov t1, t2` or vice versa), which captures the
goal of eliminating these moves. The virtual registers have to be
non-interfering, since otherwise the coalesced node would be uncolorable, and
also, since the temporaries interfere, they wouldn't be assigned distinct colors,
and elimination of the copy wouldn't be possible anyways. Because of the
non-intefering criterion, we have to be careful not to create artificial
interferences for the operands of a copy instruction, for example `mov t2, t1`
alone shouldn't imply that `t1` and `t2` interfere! Chaitin essentially does
coalescing everywhere where it is possible and where it {\em might} be
beneficial. Because of its nature, this form of coalescing has later become
called {\em aggressive}. The problem with it, is that the node created by
coalescing two virtual registers has interferences of both of the former nodes,
notably this means that the node's degree will be the sum of the two degrees and
such nodes easily become significant (\"high degree", not trivially colorable).
High degree nodes are problematic in the following {\em simplify} phase, because
apart from being blocked from simplification themselves, they prevent
simplifications on a high number of other nodes. Often this means spills of
these high degree nodes. Aggressive coloring can make colorable graphs
uncolorable, and depends on spilling to make the graph colorable again.
Since a spill of a node essentially splits the node into many low degree nodes,
this effectively undoes coalescing, but also adds memory operations that weren't
there originally.

Briggs improved on this by employing so called {\em conservative coalescing}.
Instead of coalescing all nodes that can be coalesced, he uses a filtering
heuristic, which allows only those coalesces, that can't make the graph
uncolorable. The filtering is done using a heuristic, because exactly predicting
the effect on colorability is a hard problem and would be too time consuming.
Since the effect of aggressive coalescing can be so severe, the heuristic was
made conservative, i.e. it never allows coalescings which would make the graph
uncolorable, but may also not allow coalesces that would be perfectly fine. The
heuristic says that `t1` and `t2` can be coalesced only when the merged node
`t12` would have no more significant (high-degree) neighbours, than $k$ (the
number of available registers). This implies, that after simplification of
(low-degree) neighbours, the node will have at most $k$ neighbours left, which
makes it simplifiable itself. Though it is easy to imagine a situation where a lot
of the neighbours have common neighbours, so simplification may do much better
than conservatively assumed, and thus a lot of the moves remain uncoalesced.

Appel and George~\cite[George1996] found that for their use aggressive
coalescing produced too
many spills, while conservative coalescing was too conservative, i.e. there were
are too many uneliminated move instructions left, even though coalescing would
be fine. They suggest an improvement called {\em iterated register coalescing}.
The idea is to still use only Briggs' conservative coalescing (to prevent making
the graph uncolorable), but instead of doing all the coalescing upfront, they
iterate the simplify and coalesce phases repeatedly. What is important is, that
simplify precedes coalescing---this alone improves the coalescing phase a lot,
since the conservative heuristic is based on degrees of nodes, and
simplification can decrease them significantly (and Briggs' coalescing heuristic
is too local to notice that otherwise). But importantly after coalescing
there may be nodes which become insignificant. For example, if `t3` interferes
with both `t1` and `t2`, and the two are coalesced into `t12`, then in effect
`t3` loses a neighbour and it's degree is decreased, and it might just become
insignificant (\"low degree") and simplifiable (leading to more simplifications,
which in turn might lead to more coalescings, etc.). While this may seem like a
perfect positive feedback loop, it is important to recall from TODO ref, that
coalescing two nodes creates a node of higher (even significant) degree.

Park and Moon note that even iterated coalescing can be too conservative and not
combine nodes that could be safely combined. They also note that that the
positive effect of coalescing explained in the previous paragraph is not to be
underestimated. Their approach is called {\em optimistic
coalescing}~\cite[Park2004], not to be confused with {\em optimistic coloring}
due to Briggs~\cite[Briggs1992].
Park and Moon's idea is to do aggressive coalescing like Chaitin did, to exploit the
positive effect of coalescing, but their improvement lies in being able to
revert coalescing of a particular node, if it would have to be spilled. Briggs'
optimistic coloring delayed actual spilling until the {\em assign} phase, since by
then it may turn out that the concrete assignment isn't as unfavorable as it
could have be just by judging from the interference graph and the simplification
heuristic. Similarly Park and Moon moves decisions to {\em not coalesce} into
the {\em assign} stage, and they are able to do better just because the concrete
assignment is known. For example, in case a color is not available for `t12`
(the result of coalescing `t1` and `t2`), it may be possible to find a color for
`t1` or `t2` (or both, though it will not be the same color), which effectively
undoes the coalescing. Though in practice, while Briggs' optimistic coloring
improvement was simple addition and a sure improvement, optimistic coalescing
and especially an efficient implementation is not simple.

Nice thing about Chaitin's scheme (and Briggs' improvement) is that even the
introduced spill code with new virtual registers gets the same general treatment
as other virtual registers - they are allocated by the next iteration of graph
coloring, so although the spill code needs to be inserted separately, it is
{\em not handled specially}. The price for this is that multiple expensive
iterations may be needed to finalize the allocation. An alternative would be to
(like with the top down allocator in section~\ref[sec:regalloc-top-down])
reserve a few registers off the side and use them to perform the loads and
stores around spilled variables. This could be used to rewrite the program into
final form after just one iteration of the graph coloring register allocator.
While this potentially saves multiple expensive iterations, it is less
flexible than coloring the spill code in a new iteration. In particular, since
the few spill handling registers have to be set off the side for the whole
program, we are not able to assign them, so in fact we are looking for a $k$
coloring for a smaller $k$ than the number of available registers, which
potentially means more spills by itself. On architectures like x86-64, where
some instructions only work with certain registers there is another difficulty
in choosing the on the side registers. If the registers needed by the
constrained instructions are put off the side, they would prevent any
allocation. But keeping them in the regular allocatable set would mean that
they won't be available for handling the spills of the values constrained to
such registers, the off the side registers would have to be used to somehow swap
the values with the needed registers.

\label[sec:regalloc-chordal]
\seccc Graph coloring of chordal graphs

While graph coloring in general is an NP-complete problem, for certain classes
of graphs, it can be easier. Notable example are chordal graphs, which have
posses useful properties for efficient graph coloring. It turns out, that
chordal graphs are the exact class of graphs for which exists a so called
\"perfect elimination order". Importantly for graph coloring, by assigning
colors to nodes in the perfect elimination order, the graph can
be colored with $k$ colors in a single greedy pass---of course provided that the
graph is indeed $k$-colorable. Similar greedy coloring pass was the {\em assign}
phase of Chaitin's algorithm~\cite[Chaitin1981], there the node ordering was
determined by simplifications based on a heuristic and colorability was ensured
by spilling nodes. Perfect elimination order can be found in $O(n^2)$ time using
the maximum cardinality search algorithm and {\em guarantees} optimal coloring.
Chordal graphs also offer improvements for spilling, because like with perfect
graphs of which they are a subset, the number of colors needed to color a
chordal graph is given by the size of the largest clique. This is powerful,
because it gives the possibility to do enough of spills or live range splits
ahead of time, before actually starting with coloring.

The first application of these ideas to register allocation are due to Pereira
and Palsberg~\cite[Pereira2005], who noticed that $95\,\%$ of
interference graphs in the Java 1.5 library had chordal interference graphs
(when compiled with JoeQ compiler). The algorithm proposed by them operates in a
few independent phases:

\begitems\style n
* {\em pre-spilling}. Spill code is inserted in order to decrease the size of
the largest clique to $k$, which makes it $k$ colorable.
* {\em greedy coloring}. The graph is greedily colored without limiting the
number of available colors.
* {\em post-spilling}. If the number of used colors exceeds $k$, additional
spills are done.
* {\em coalescing}. Move related nodes are coalesced, if possible.
\enditems

Both {\em pre-spilling} and {\em coalescing} are entirely optional---spills can
be handled by post-spilling phase and coalescing is not a necessary port of any
register allocation algorithm. But they both improve the quality of the
generated code. In particular due to the properties of the chordal graphs
described above, pre-spilling is a much better place for introducing spills, and
if done properly, it is guaranteed that post-spilling doesn't need to do
anything at all. Case when {\em post-spilling} comes into play are when the
optional pre-spilling isn't run, or when {\em non-chordal} graph is being
colored---Pereira and Palsberg noticed that the same algorithm can be used also
for non-chordal graphs, though the register assignment is not optimal, it is
competitive according to them~\cite[Pereira2005].

One of the important benefits of the algorithm is, that it isn't iterated, it
finishes after running each phase only once. Though to be more precise, the
post-spilling phase is not a single step---if more than $k$ are used, all nodes
of one color are spilled (transforming a graph using $m$ colors to one using
$m-1$), so this needs to be iterated until the only $k$ colors are used. Though
this is bounded, usually fast and may not be needed at all if {\em pre-spilling}
phase is run.

While there are interesting similarities to Chaitin's classical approach, in
particular being able to spill enough so that greedy algorithm can find a
coloring, there are also interesting differences---in Chaitin's and derived
algorithms coalescing is done {\em before} assignment, this can have both
positive and negative impacts on colorability, and different variations
approached it differently. Pereira's algorithm does all coalescing {\em after}
assignment because after coalescing an interference graph can become
non-chordal, though they report that in their experiments their approach
does better in coalescing than Appel's Iterated register coalescing.

While Pereira's algorithm can be used even for non-chordal interference graphs,
following researched by e.g. Hack~\cite[Hack2006] showed, that programs in
(strict) SSA form have chordal interference graphs. This is seems like an
excellent result, because SSA form is great for middle-end optimizations and it
simultaneously seems to be good for register allocation. Additionally SSA form
allows much more efficient computation of the liveness property. The basis
structure of Hack's algorithm is similar to Pereira's---do spilling before
coloring to make the program $k$ colorable, then color the graph in a perfect
elimination order. However, there are substantial improvements: as they prove
that interference is directly connected to the notion of dominance deeply
associated with SSA, it is possible to derive the perfect elimination order from
the control flow graph and dominator tree alone. Also, if we assume that copy
propagation has been run before register allocation, the only coalescing that
has to be done on SSA is the coalescing of $\phi$ node operands---by assigning
the operands the same color as the phi node, SSA deconstruction doesn't have to
insert any moves. However, in fact SSA deconstruction and coalescing have to be
integral parts of their algorithm, since inserting arbitrary moves for SSA
deconstruction could make the graph non-chordal (and in fact SSA incompatible)
or increase register demand---which would be detrimental, because now SSA
deconstruction is done {\em after} register allocation.

The fact that Hack's algorithm is able to take advantage of the many guarantees
of SSA form, is also it's great disadvantage---it depends on the program on
being in SSA form, which is not the norm. Usually, compilers do register
allocation only after instruction selection during which concrete machine
instructions are chosen to implement the behavior of the SSA-based intermediate
representation. $\phi$ instructions are not real instructions real
architectures, and certain operations are incompatible with SSA. TODO elaborate.

\seccc Reduction (to another NP-complete problem and using a solver)

ILP, IBQP

\chap Design and implementation

This thesis looks into practical issues of taking a modern middle end
intermediate representation and translating it to executable machine code of a
real architecture, x86-64. This is not something that hasn't been done before.
The main benefits of this thesis are doing it as part of a TinyC compiler. TinyC
is like C, but simpler in some aspects, though still keeping many challenges for
implementations of middle ends and back ends alike.

In the NI-GEN course at FIT CTU, where TinyC originated, students write
compilers from TinyC to {\em Tiny 86}.

After careful consideration we set the following goals for the implementation of
a TinyC backend for x86-64 and a runtime:

\begitems

* The compiler should explore how TinyC features translate to constraints of a
real architecture.

* Execution on real hardware (x86-64 CPU) and operating system should be
possible.

* Advanced global optimizing techniques should be used.

* The source code may be read by students of the NI-GEN course, who have not yet
written a compiler back end yet.

* Simplicity of the compiler structure and of the code shall be one of the main
goals, because the backend will serve as a demonstration of compiling TinyC to a
real machine architecture.

* The number of intermediate representations should be low. This should make the
code more digestible and prevents much of the code being just translations from
one IR to another.

* The back end should not have many external dependencies, since while they may
simplify development and implementation, they hide details that are necessarily
part of the implementation. For educational purposes it is beneficial to fully
show everything.

* As most compilers usually target multiple architectures, they either need
separate back ends for each architecture or need to make their back end machine
independent and driven only by machine descriptions. The goal of this thesis is
not true machine independence, as we target only one architecture, but
extensibility to other architectures should be considered.

* The back end should be a separate program, independent of the TinyC front end.
This allows the TinyC front end used in the NI-GEN course to adapt to
circumstances without having to worry about our back end.

* The produced x86-64 assembly should be optimized not only for machine
consumption by an assembler, but also for consumption by a {\em
humans}---the assembly should be approachable by students.

* Since real programs rarely live in vacuum, the produced code should be able to
interface with the operating system and foreign code.

* The backend targets real machine code, comparisons with other compilers will
be possible and the relative performance of our solution should be evaluated.

\enditems

Next subsections focus on the design and implementation of a TinyC backend for
x86-64. Many of the algorithms used in the state of the art compilers have been
introduced in a previous chapter (chapter~\ref[chap:state-of-the-art]). This
chapter focuses mainly on evaluating the algorithms and their practicality and
possibility of use in our back end.

This chapter also focuses on {\em data structures} in
section~\ref[sec:data-structures]. Data structures have been so far neglected
and  whose careful design is very important for any project. The main goal of
the data structures is to serve the algorithms, but in this thesis we also focus
on approachability and simplicity. Motivation and design of the data structures
are presented to allow the reader to imagine the implementation of an algorithm,
even if it is not presented in full.


The main part of our design and solution is the back end. However, to even
translate code to.

\sec Architecture

TODO The TinyC frontend used in the NI-GEN course

Rozvržení fází, nějaké diagramy, přechody mezi reprezentacemi atd.

The input to our back end is a TinyC middle end IR, and the output is x86-64
machine code. This gives us our two main interfacing points. Everything in
between is the goal of this thesis and the architecture should be designed to
fulfill the goals presented above.

TODO: nasm

Traditionally compilers have been split into three main parts---front end,
middle end, and back end (see chapter~\ref[TODO compiler structure]. Our compiler
will have a bit of middle end (as it constitutes the input) and focus mainly on
the back end. Often these three stages operate on completely different
intermediate representations. In particular middle ends usually aim for
representation suitable for (machine independent) optimization and thus often
employ SSA form. On the other hand, representations in middle ends usually aim
for producing the best machine code possible, thus they focus on finding
patterns in the compiled programs as well as the instructions sets, which allow
choosing the most appropriate instruction sequence.

No instruction scheduling TODO.

\secc Middle end

Our compiler will follow this tradition, and have a separate middle end IR and
back end IR. In particular, our middle end representation will internally be
based on {\em value-based SSA} (see section~\ref[sec:ssa-value]). It is proven
in practice in other compilers like LLVM, and is relatively straightforward,
while it can still be considered state of the art even today.

TODO existing TinyC frontend

\secc Back end

In the back end our task is to mainly select the instructions, order them and
allocate registers.

Main interest of this thesis is a backend. As introduced in
section~\ref[sec:backend] the main goal of a compiler backend is to translate
from a (preferably machine independent) intermediate representation into machine
instructions that can be executed by the processor.

For instruction selection, back ends usually use a specialized intermediate
representation. For example, in the case of instruction by peephole
optimization, these are the register transfer lists, or for tiling it can be
trees, DAGs or general graphs.

While often register allocation is done on selected and ordered instructions, it
doesn't have that many requirements on the actual IR. Though depending on the
allocator, the IR has to to easily support calculation of {\em liveness}, which
for iterative data-flow analysis requires the availability of control flow
information and the ability to iterate over {\em used} and {\em defined}
registers.

In our back end, we are dealing with the x86-64 architecture, its machine
constraints and calling conventions:

\begitems

* Multi-output instructions like `idiv` (which produces both the quotient and the
remainder) are not truthfully representable with trees and require DAGs.

* Some instructions like shifts require operands to be in certain registers.

* Not only are there two completely independent register classes (the general
purpose registers and the \"vector" registers), but both of them have aliasing
subclasses which allow accessing only parts of the registers and operands of
concrete instructions are limited to one concrete register subclass.

\enditems

The restrictions on registers often mean that register allocators have to either
be able to represent {\em physical register} in addition to {\em virtual
registers}, or otherwise be able to restrict register assignment (like with {\em
precoloring} in graph coloring register allocators described in
section~\ref[sec:regalloc-graph-coloring]).

Instruction selection is thus more decisive when it comes to intermediate
representations. Advanced intermediate representations like DAGs and register
transfer lists also come with additional problems---it has to be ensured that
in every situation the nodes or transfer lists are translatable back to machine
instructions. This is a big problem for ensuring correctness.

The popularity of SSA for middle end optimizations and recent developments in the
field of register allocation on SSA form make it very appealing even for a
backend. However, methods for instruction selection on SSA are currently based
on reduction to other problems. Similar reductions are possible for other
components of a back end. Though for educational purposes, we want to focus on
dealing with instruction selection and register allocation directly, and not
focus on heuristics for solving NP-complete problems or offload the work to
external dependencies.

Considering all these requirements came an idea to use machine instructions as
the intermediate representation throughout the back end. This is based on
several observations:

%Even though multiple different intermediate representations in the back end can
%be beneficial, our idea is to use only {\em one intermediate representation}
%based on {\em machine instructions}. The idea is based on a few
%observations:

\begitems

* We have to be able to represent machine instructions, since they are our
output.

* Often a peephole optimizer step is run on the machine instructions in the
final stage of a back end TODO ref and this is done over machine instructions.

* Instruction selection by peephole optimization TODO ref is a very competitive
technique. It can also be used on a stream of machine instructions, though
data-flow has to be used to achieve better quality.

* Correctness of instruction selection is more easily achieved if at all times
the IR directly corresponds to machine instructions.

* Expressing machine constraints and calling conventions is straightforward in
an IR based on actual machine instructions.

* Even though the machine instructions are {\em machine dependent}, their {\em form}
and representation can be {\em machine independent}, still allowing for
potential expansion to other targets.

\enditems

If we use machine instructions for our back end, then full architecture of
compiler can consist of the following steps, also highlighted in figure TODO:

\begitems

* {\em SSA deconstruction}. Even though value-based SSA is convenient for the
middle end, in the back end we work with machine instructions, which work with
{\em registers}. Also, we need a place, where we replace $\phi$ functions with
equivalent copies.

* {\em Code generation}. Translating from middle end to back end IR is often
in general called {\em lowering}, because often we transform from a higher level
representation to a lower level one. However, in our case, since back end IR
represent the machine instructions directly, we can just call the phase {\em
code generation}, since it will generate appropriate code for the target
architecture. The code generator  has to be machine specific, because it handles
machine instructions, as well as calling conventions and other machine specific
constraints.

%The code generator will preferably be as simple as possible, and generate the
%most straightforward, inefficient, but correct code. Since we 

* {\em Instruction selection}. Even though we produced valid instructions, the
instructions are not necessarily the best possible ones. To improve the
selection of instructions, we employ instruction by peephole optimization,
inspired by Davidson and Fraser (see section~\ref[sec:davidson]).

* {\em Register allocation}. To simplify previous stages of our back end, they
will mainly work with {\em virtual registers}, but they will also refer to
{\em physical registers} to realize machine constraints and calling conventions.
Before the machine code is finalized we have to rewrite the machine code to
refer to only physical registers.

* {\em Peephole optimization}. Assignment of physical registers may unlock new
possibilities for optimization. For example, due to coalescing (see
section~\ref[sec:regalloc-coalescing]) copy instructions may become redundant,
which can quickly cascade to other improvements, e.g. due to instructions now
fitting into a peephole window.

\enditems

More complicated instruction selection mechanisms on specialized IRs have
problems with ensuring correctness, because they may not be able to find a
suitable instruction for a piece of the IR (subtree, register transfer list). In
an extreme case the IR can be limited so that each piece of IR corresponds to
one instruction. But, this has to be ensured across all targets, thus
limitations in one target limit an IR common for all targets.
%Our approach takes
%this idea to the extreme---
Because we choose the initial instructions in the {\em code generation} step
based on the middle end IR, we have to ensure that all our middle end IR nodes
correspond directly to one or more machine instructions. This brings the
limitations to the extreme---our middle end IR is essentially limited by all
targeted architectures. This means that in the middle end IR we may have to
limit ourselves to the most basic RISC-like operations, which are offered on
all relevant architectures.

Limiting middle end, based on target architectures might seem like a bad thing,
but we argue otherwise. We can use a simple, regular RISC-like IR is easy to
reason about in the middle end, and in fact maybe even preferable even if we had
a free choice. Additionally, as seen in section TODO ref, peephole optimization
benefits from {\em expansion} into fundamental operations, because it can match
them together into machine instructions based on the target architecture's
available instructions and capabilities.

In our case, that role could be filled by a very naive code generator, which
would generate the most simplest sequences of instructions and thus produce the
\"expanded IR" the peephole optimizer could then operate on. This makes the code
generator simple, because not much case analysis is required. A smarter code
generator employing more case analysis would even be undesirable, because it
would hide low level patterns that could have been exploited by the peephole
optimizer.

%With the overall back end architecture introduced in this section, we introduce
%details of the different back end parts in the following subsections.

%%%


%if the code generator generates the most simplest
%straightforward code possible for each middle end operation (without any case
%analysis), the code will be inefficient, but 
%
%
%because only simple, fundamental operations are available, 


%With tiling or the expansion based instruction selection, this has to be done in
%the back end and the problematic IR was the back end one---for example for each
%subtree or register transfer list, there has to be an instruction that can
%implement it.
%
%But in our approach the problem shifts somewhat---we have to
%ensure that each middle end operation can be mapped to either one or more
%machine instructions, because that is where we do the initial selection of
%machine instructions. In case of more target architectures, we have to ensure
%that for all of them, potentially making the IR very constrained for the
%middle end. Though 
%
%Following the ideas of Davidson (see section TODO ref), we 

%Having machine instructions as the back end representation means that the
%lowering step, which translates code from middle end IR to back end IR, is
%actually the {\em code generator}---it outputs an instruction sequence that
%implements the middle end IR for the target architecture. By employing
%instruction selection by peephole optimization we can have the simplest code
%generator possible, generating the most naive, but obviously correct sequences
%of instructions implementing the middle end IR. Then, through sequence of many
%simple transformations, the peephole optimizer improves the instructions.
%
%Though like many
%compilers, even in ours we do register allocation separately, so the code
%generators outputs only almost valid machine instructions, because there are
%still.
%
%* Physical registers have to ultimately be used by the produced code. Though
%before register allocation virtual registers are used, and due to machine
%constraints (see section~\ref[sec:regalloc-constraints]), physical registers are
%usually also used. Since both kinds of registers need to be supported at the
%same time, we may as well represent them in a common manner, and allow
%machine instructions to use both. Register allocation is then responsible for
%transforming all uses of virtual registers into only uses of physical registers
%(with appropriate spill code).
%
%Therefore either the IR has to be
%designed such that each single node in the IR corresponds to at least one
%instruction or 

\seccc Lowering

\seccc Peephole optimization

\seccc Register allocation

The high level means of operation of register allocation are simple---map
virtual registers to physical registers. The details are much more complicated
and explained in section~\ref[sec:regalloc]. Here we focus on how to do register
allocation in practice on our back end representation (machine instructions) for
our target architecture (x86-64).

Since our back end representation is not in SSA form, we can't use register
allocation techniques requiring it. Though possibilities to use SSA form even
for machine instructions exist (see section~\ref[sec:ssa-machine]). But in our
opinion, the technique is little too disconnected from the traditional state of
the art compiler structure, which deconstructs SSA before register allocation.
For educational purposes we don't think it would be a great idea to do register
allocation on SSA form.

Local methods like top-down or bottom-up register allocation
(section~\ref[sec:regalloc-top-down]) are unsuitable, because their local nature
just doesn't allow them to produce great register assignment.

Linear scan, though a global algorithm does also not quite fit our situation.
As mention in section~\ref[sec:regalloc-linear-scan], the algorithm was designed
for JIT compilers, which care very much about {\em run time} of the algorithm
and for which more sophisticated methods already available at the time were too
slow. We are designing an ahead of time compiler, which doesn't have to worry
about compile time as much as JITs do.

Reduction of register allocation to another problem is interesting and produces
good results, but as explained in section~\ref[] we prefer to avoid reductions
to another problems and instead like to stay close to the original register
allocation problem.

Register allocation by graph coloring (see
section~\ref[sec:regalloc-graph-coloring] for more details) can also be seen as
a reduction to another problem. What makes it different, is that usually
implementations of graph coloring register allocation stay specialized---they
don't just offload the work to an external graph coloring solver, but they code
specialized solvers exploiting many of the details of register allocation like
possibility of coalescing. Register allocation also allows great flexibility
with regards to machine constraints and also offers solutions for handling
complex register class hierarchies. For this reason we think register allocation
is the most suitable register allocation technique for our purpose.

There are many different improvements to Chaitin's original graph coloring
register allocation. For example, optimistic coloring is a big improvement,
while being very simple and efficient. Other improvements to Chaitin's algorithm
mainly focus on improving coalescing and taking advantage of the special
properties of interference graphs (see section~\ref[sec:regalloc-chordal]).

Ultimately, we decided to implement Iterated register coalescing as formulated
in~\cite[George1996]. It has been regarded as the state of the art for a long
time, and other newer algorithms take it as a reference (for example~\cite[Park2004]
or~\cite[Pereira2005]). While optimistic coalescing seems great, it is not as
straightforward extension as optimistic coloring. In fact, the exact approach is
asymptotically impractical, and instead authors~\cite[Park2004] use heuristics
instead. Graph coloring in perfect elimination of~\cite[Pereira2005] order is
in today's eyes a remnant from time before special properties of SSA were
understood. As a result it would be much preferable to use SSA-based register
allocation due to~\cite[Hack2006], which also focuses on machine constraints and
other problems encountered by practical register allocators.

%One disadvantage of graph coloring register allocation is, that it can't do on
%the fly live range splitting.

Other advantage of Iterated register coalescing is that, like some other formulations,
but unlike for example~\cite[Pereira2005], it has an existing flexible and
comprehensive method for handling register classes due to~\cite[Smith2004].

\sec Technology

There 


Přístupnost, jednoduchost =>

C (známé, žádné závisloti, možnost reprezentovat datové struktury přesně podle
potřeby)

malé množství IR (middleend + backend)

pokročilé a dobré algoritmy, ale ne úplně dnešní state of the art


\sec Architecture

\sec Data structures

\label[sec:data-structures-middle-end]
\secc Middle-end

Even though middle-end is not a direct part of a backend in a compiler, they
interact with each, since the 

SSA IR, inspirované LLVM

The entire middle end is based on control flow graphs (CFGs) and the idea of {\em
value-based SSA form} (presented in section~\ref[sec:SSA-value]). Most of the
things the middle end works with can be represented as values. Including
functions (represented by their addresses, i.e. they are pointers to code),
blocks (also pointers to code), static variables (also represented by addresses,
i.e. pointers to `.data` or `.bss` sections).

Representing everything with values is great, because it is uniform and
easy to work with. But as there isn't a single kind of value, there are
potentially very many if we count each kind of distinct operation as a separate
kind of value. Different languages have different approaches for expressing the
idea that a type has many different variants. Object oriented
languages like C++ use {\em inheritance}. There, we would have a top level class
`Value` and other classes like `Function`, `BasicBlock` or `Constant` that would
inherit from it. Each subclass holds data ({\em fields}) appropriate for the particular
kind of value, but also ones inherited from the base class. Not only are
different subclasses of different sizes, but code often wants to work with them
opaquely---as such it only works with {\em references} to the values (either
with real references or pointer as in C++ or through the classes being {\em
reference types} in Java). Behavior supported for all values is defined in
methods on the top level `Value` class, but subclasses can override it (or even
may have to override it, if the method is abstract). Choosing the appropriate
method for a reference to `Value` that may well be `BasicBlock` or a `Function`
with different overrides is achieved with {\em dynamic dispatch} often implemented
with {\em virtual tables}. Adding new value variants means introducing new
subclass, which overrides the methods from its parent class as appropriate.
Adding new behavior is not that easy, since possibly all subclasses have to be
extended to handle the new behavior.

Languages which support algebraic data types support a \"type that has different
variants" through {\em sum types}. With sum types, we can introduce a (top
level) `Value` which would be a sum type of the types introduced for all the
variants (like `Function` or `Constant`). Internally, this is usually
represented either very similarly as the class hierarchy, with a separate
representation for the different variants, each having only its fields. But there
would be also once common field---a {\em discriminator} (small integer) which
distinguishes the variants. Though sometimes more efficient representation
consists of a {\em tagged union}---the sum type is big enough to store all the
variants (like in a C `union`), but also contains the discriminating integer
(the {\em tag}). Since the representation is uniform and the sum type has enough
storage for all variants, it can be used to represent the type directly, there
is no need for a level of indirection (so for example, array of `Value` would be an
array of structs not array of pointers to structs). Behavior would be
implemented in functions, which would check the discriminator to choose the
behavior for the variant at hand. Many languages aid the discriminator checking
with more powerful pattern matching. Adding new behaviors is easy, since just
one more functions is added. Adding new variants is not as easy, because all
functions have to be extended with the behavior for the new variant.

The class hierarchy and sum type approach are essentially complements  each
other---one makes it easy to add new variants and the other makes it easy to add
new behavior. This is known as the {\em expression problem} and a lot of common
languages (including C, C++, Java and Rust and OCaml) don't offer any real
solution. Since our implementation language is C, which can implement both, we
can decide on the approach freely and most appropriately to our use case.

Our `Value` type has a relatively {\em fixed set of variants}---there are
arithmetic operations and constants like functions, basic blocks and global
variable addresses. Extending with new variants is imaginable, but there isn't a
large space for extensions here---there are just so many operations that a
middle end can consider.

On the other hand, there are many different {\em behaviors} for middle end
values. We want to print them, translate them to potentially many different
architectures, and most importantly {\em optimize them}. There can be many
different optimization passes over the middle end. These can be largely
independent and benefit from being fully contained in a single place (either
function or module).

For both of the above reasons sum types seem better for this use case. However,
we can achieve the same with class hierarchy thanks to the {\em visitor
pattern}. With visitor pattern we introduce a central dispatching place (the
base `Visitor` class, which makes it hard to introduce new independent variants,
the benefit of the class hierarchy approach), but allows new behaviours to be
introduced by subclassing the `Visitor`. Different visitors can be introduced
independently and also can group all the code implementing a single behavior,
instead of interspersing it all over the subclasses constituting the variants.

However there is still a difference in the two and it is in how we work with the
variants and how we {\em distinguish} between them. In our middle end, our
optimizations often want to check for the nature of the values. For example, a
peephole optimizer may want to check whether a value is an operation on
two constants and replace it with the constant equal to the result (to perform
{\em constant folding}). Elsewhere we may need to check whether a load
loads from stack slot. This means that apart from checking the variant of a
single value, we often want {\em nested checks}. Sum types with their
discriminator allow such checks easily and languages with pattern matching allow
even nice nested checks. In a class hierarchy for such questions we would have
to either introduce methods for performing these checks (like `isConstant` or
`isLoadFromStackSlot`) or employ some mechanism of investigating the variants,
like with `dynamic_cast` in C++ or `instanceof` in Java (both of which are seen
as very unidiomatic in their respective languages).

For our use of `Value`s, sum types seem to be better match. But still, there are
two options of implementation---{\em inheritance} (through struct embedding) and
{\em tagged unions}. Ultimately we decided to go with inheritance for several
reasons:

\begitems
* Values are {\em recursive types}---values contain other values. For example an
unconditional branch \"contains" (has as a field) the destination of the jump.
And this is true for essentially all operations which are just values produced
from other values. Recursive types usually have to be implemented with a layer
of indirection (in C this is done explicitly with{\em pointers}). Because of
this, tagged unions which normally allow less indirection are not that
beneficial.

* The sizes of individual values can differ greatly. For example there are
usually not that many functions, but we may need to store a lot of information
about them. On the other hand there may be many numeric constants, which
essentially store just their tag and the constant itself. Tagged unions would
have required us to have them all of the same size. Differences in size of
the variants can be mitigated by adding a level of indirection to same variants
and represent them with pointers. But having some variants with indirection and
others not makes the structure slightly less nice to work with and since we need
the level of indirection because of the recursivity, we may as well go with
inheritance.

* Tagged unions require all the types used to represent the variants to be
listed in the `union`. With inheritance new types can be introduced more
independently, since most consumers of the types don't care about the size of
a \"`Value`", they work with pointers to values, i.e. \"`Value *`". The benefit
of this independence is not big, since the tags have to be listed in one central
place (commonly an `enum` in C).
\enditems

Currently, the `Value` representation looks like this:

\begtt
typedef struct Value Value;
struct Value {
	ValueKind kind;
	u8 visited;
	u8 operand_cnt;
	Type *type;
	size_t index;
	Value *parent;
	Value *prev;
	Value *next;
};
\endtt

`Value` is the base type other other types used to represent values inherit
from. As such, it can have fields common to all variants, the obvious one being
the discriminator, called `kind` in our case. In our implementation
we found it useful to have also other fields:

\begitems
* All values have a type (one of the TinyC types, e.g. \"`int`" or \"`char *`").

* Each is given a unique index in its scope. For example, all operations in a
function have distinct indices in the particular function. Each function has a
distinct index from all other functions and basic blocks in one function also
have distinct indices. Indices are assigned from zero and serve a few purposes"

\begitems
* Textual representation (see section~\ref[sec:ssa]).
* Assignment to virtual registers (see section~\ref[sec:ssa-deconstruction]).
* Indices to off the side arrays containing information (temporarily) associated
with values. Indices can essentially provide much faster and compact way of
mapping values to something, then for example a hash table could.
\enditems

Since operations link to each other directly with pointers, the indices don't
have much {\em semantic} meaning (except for the SSA deconstruction stage). This
for example means that the assignment of indices to values can be changed
arbitrarily, without changing the meaning of values.

* Each value contains a link to its parent. For example, operations link to
basic blocks they are contained in and basic blocks link to the functions they
are in. This is can sometimes be convenient.

* Each `Value` has a link to next and previous one. This is mostly convenient
for operations, which due to our control flow graph based representation have to
be ordered. These links make up the explicit order. Since the linked list is
contained in the structure itself, this is called an {\em intrusive list}, (as
opposed to for example something like a {\everyintt={}`std::list<T>`} in C++).
Intrusive lists are a convenient way for representing linked lists in C.

* Even though not all values are operations, we still want to associate an
operand count with each value. For values which are not operations, this count
is simply zero. For some operations the operand count can be determined
implicitly from the value's kind---for example multiplication of two values has
always the operands. But operations like $\phi$ nodes or calls, which can have
different number of operands (depending on the number of predecessor blocks or
call arguments respectively) require the number of operands to be determined in
other way than just with a check of the value's kind.

Originally, we were able to determine the number of operands for $\phi$ nodes
through parent link to basic block, which stores the number of block
predecessors. The number of operands of a call operation was determined from the
type of the first operand (the called function)---it's type also knows the
number of arguments the function takes. However this became problematic when
function with variable number of arguments became supported---for them the
number of {\em parameters} is only lower bound for the number of parameters. So
the operand count became stored explicitly.

* A `visited` field of values can be used by graph algorithms over values. This
is used for example by the depth first traversal which computes the postorder
of basic blocks in a function.
\enditems

The linked lists deserve a bit of attention. Operations in basic blocks are
ordered through linked lists, because insertions, deletions and reordering in
the middle are convenient with them. Since the middle end is incomplete and
doesn't actually do much, this amount of flexibility is not needed. But linked
lists are convenient to work with in C, since C for example has no conveniences
for dynamic arrays.

With ordinary doubly linked lists, there is {\em head} and often also a {\em
tail} pointer which are like the entry points to the linked list---they are the
pointers from outside to inside the list. However, these bring several special
cases to an doubly linked list implementation. For example a deletion of a first
node in a linked list should (apart from changing the `prev` field of the second
node) change also the head of the linked list. Even just construction of the lists
encounters a few special cases.

{\em Circular doubly linked lists} solve most of the problems. Since \"last"
node points to the first and vice versa, there is no longer a special case. All
nodes can be treated as if they were in the middle. The problem with circular
linked lists is that there can not easily be something like a `head` pointer
from the outside to the linked list---or rather there could be, but it would
bring back the special cases of non-circular doubly linked list. But in our use
case, we want to link together operations inside a basic block to a linked list
rooted in the basic block---the basic block should contain a pointer to the head (or
also the tail) of the linked list of instructions. Our solution is to make basic
blocks part of the doubly linked list of instructions they contain. This can be
done, since basic blocks inherit from `Value` as well and thus have `next` and
`prev` fields and can be part of the list. But in that case actually `next` and
`prev` of the basic block serve as the `head` and `tail` pointers to the list of
instructions. But thanks to the linked list being circular, they are updated
transparently.

Because the `next` and `prev` fields of basic blocks are used for holding the
instruction lists, they may not be used to link together all basic blocks in a
function, thus we need to handle it differently, see
section~\ref[TODO-function-representation].

Here we show how some concrete value variants are represented:

\begtt
typedef struct {
	Value base;
	i64 k;
} Constant;

typedef struct {
	Value base;
	Str name;
	Value *init;
} Global;

typedef struct {
	Value base;
	Value *operands[];
} Operation;
\endtt

All values inherit `Value` through {\em struct embedding}. For consistency, all
value variants inherit `Value` in a first field called `base`. This makes it
easy to both downcast (`(Global *) value`) or upcast (`&global->base`), since
for example pointer to `Constant` and its `Value` base are the same.

Both `Constant` (which represents actual integer or character constants, number
literals) and `Global` represent constant values. Their fields should identify
them sufficiently. For numeric constants, this can be done with a single field
holding a 64-bit integer (the largest size supported by TinyC). Global values
are more interesting as they are named. With external linkage, we need to
preserve their names. Also, unlike normal variables allocated on the stack,
global values are special, because they may only be initialized with {\em
constants}. For simplicity, we store the initializer with the `Global`. If there
is no initializer (`init` is `NULL`), then the global variable is meant to be
zero initialized (i.e. it will be put into `.bss` section and the initializer
doesn't have to be stored in the resulting binary). Allowing only constant
initializers is the same behavior as in the C programming language.\fnote{If a
global variable were to be initialized with a non-constant value it would be
questionable {\em when} the value should actually be evaluated---for example the
global variables should certainly be initialized before `main` runs, but in what
order, and how to run code before `main`? For instance C++ allows non-constant
initializers and runs them through hooks running before `main` offered by the C
runtime which is the one calling `main`.} Type of a `Global` is actually a {\em
pointer} to the type of the global variable---in a sense the value represents the address
of the global variable, not the global variable itself.

Operations are the only values which reference other values. Operations consist
purely of the base and an array of operands. The array of operands uses a
feature of the C programming language called {\em flexible array member}, which
allows the array to be part of the structure, but to be dynamically sized. This
way, the same `Operation` representation can work with arbitrarily many
operands---though the structure has to be dynamically allocated with the correct
size. The actual number of operands is given by the `operand_cnt` field of the
`base`, as explained above.

This representation of operations is very uniform and allows tasks as iteration
over all operands very easily. This simplifies operations on the middle end
representation quite a lot and requires no case analysis, since values that are
not operands simply have `operand_cnt` of zero. Iteration over all operands is
very common for many operations on the middle end IR---printing, translation to
machine code, use analysis, etc.

On the other hand, consider how much case analysis and special provisions for
iteration would have to be made if for example unary and binary operations were
represented like this:

\begtt
typedef struct {
	Value base;
	Value *arg;
} Unary;

typedef struct {
	Value base;
	Value *left;
	Value *right;
} Binary;
\endtt

Though the advantage of this representation is, that operands are available
through have human readable. For our operation structure, this can be remedied
through macros:

\begtt
#define OPER(v, i) (((Operation *) (v))->operands[i])

#define UNARY_ARG(v)     OPER(v, 0)
#define BINARY_LEFT(v)   OPER(v, 0)
#define BINARY_RIGHT(v)  OPER(v, 1)
\endtt

Though the macros are much useful in cases where there the order of operands is
not so obvious. These are mainly the operations which don't correspond to
classic arithmetic, and which we introduce here:

\begitems

* {\em Load}. The load operation has one operand, the address for the load,
while the load operation itself represents the loaded value.

\begtt
#define LOAD_ADDR(v) OPER(v, 0)
\endtt

* {\em Store}. Store operation has two operands, one is an address (target of
the store) and the other is the value that is to be stored. Unlike most other
operations, store doesn't evaluate to anything---it is only useful for its
{\em side-effect} on memory.

\begtt
#define STORE_ADDR(v)  OPER(v, 0)
#define STORE_VALUE(v) OPER(v, 1)
\endtt

Currently, we model the fact that an operation doesn't return anything with the
TinyC `void` type. This is a bit weird, since in fact the store operation itself
represents a value, while `void` should represent an absence of a value
altogether. A bit more flexibility could be achieved if stores would return a
{\em unit} value of the {\em unit type}, but so far we kept the type system used
in the middle IR to exactly the same as the one used for the TinyC programming
language.

* {\em Call}. Call operation has as its first operand the function to be called
(a value with type of function pointer), and the arguments are in the operands
starting at index one. The call operation evaluates to the return value of the
function and has its type.

\begtt
#define CALL_FUN(v)  OPER(v, 0)
#define CALL_ARGS(v) (&OPER(v, 1))
\endtt

* {\em Jump}. Jump operations perform unconditional jump to a basic block. The
basic block is the only operand of the operation.

* {\em Branch}. Branch operations perform conditional jump to one of two basic
blocks, based on a condition. The condition is stored as the first argument, the
basic block which is the destination in case the condition evaluates to true
(non-zero) is the second argument, and the destination in case the condition
evaluates to false (zero) is the third argument.

\begtt
#define BRANCH_COND(v)  OPER(v, 0)
#define BRANCH_TRUE(v)  OPER(v, 1)
#define BRANCH_FALSE(v) OPER(v, 2)
\endtt

* {\em Return}. There are actually two kinds of returns---those that return a
value (i.e. `operand_cnt` is 1) and those that don't return a value
(i.e. `operand_cnt` is 0). Absence of a return value could also be represented with a
return of a unit value of the unit type, but this was again rejected to stay
in line with the TinyC type system.
\enditems

Currently, the type of the control flow changing ({\em terminator}) instructions
is the TinyC `void` type. Some languages employ a {\em bottom type} (sometimes
also called {\em zero type} or {\em never type}), which has no values and thus
can be more suitable for representing control flow changing operations, since
they don't ever return.

Basic blocks apart from being the heads of the linked lists of instructions
through their `next` and `prev` pointers in `base` also explicitly hold the
{\em predecessors} through a dynamically growable array:

\begtt
typedef struct {
	Value base;
	MBlock *mblock; // link to corresponding machine block
	Block **preds_;
	size_t pred_cnt_;
	size_t pred_cap_;
	size_t depth; // loop nesting depth (0 means outside of all loops)
} Block;
\endtt

In TinyC IR blocks can gain {\em successors} only through jump and branch
terminating operations. Thus block's successors can be found implicitly by
investigating it's last operation through the `base.prev` field. Explicitly
storing successor blocks would only lead to duplication, which can easily result
in inconsistencies. One special value held by basic blocks is the loop nesting
depth which is used to calculate spill costs TODO explain somewhere spill cost
calculation. There is also a link to a machine block, which is further explained
in section~\ref[sec:data-backend].

Other operation that deserves mention is the $\phi$, denoted with `phi` in the
textual form of the IR and often simply written as \"phi". They are deeply
connected to blocks---while still ordinary operations, their operands correspond
to values from predecessors. In our representation this correspondence is
implicit: $i$th operand of a phi operation is the value from the $i$th
predecessor. This needlessly requires care from any code tries to change control
flow, and may encounter phi operations. However such code will always require
special care, because of the non-standard nature of phis. 

\label[sec:data-structures-middle-end-function]
\secc Representation of functions

Representation of functions is show below:

\begtt
typedef struct {
	Value base;
	size_t index;
} Argument;

typedef struct {
	Value base;
	Str name;
	Argument *args;
	Block *entry;
	Block **blocks;
	Block **post_order;
	size_t block_cap;
	size_t block_cnt;
	size_t value_cnt;
	MFunction *mfunction; // link to corresponding machine function
} Function;
\endtt

As mention in a previous section, we usually don't want to iterate over the
blocks in some random order or the order they were created in. Instead, we want to
iterate over them in an order that is beneficial for performed analysis and
optimization passes. Often this order is either reverse postorder (where all
non-cyclic predecessors are visited {\em before} the block itself) or postorder
(where a block is visited {\em after} all its non-cyclic predecessors).

Postorder is easily computed with depth first search TODO cite dragon book,
and an array with (pointers to) basic blocks is stored in functions. The array
is dynamically growable, since blocks can be added or deleted to functions. The
only explicitly stored basic block for a functions is the {\em entry} basic
block. All useful blocks are (recursively) reachable through it and listed in
the postorder. By using the post order to iterate over the blocks, unreachable
blocks are automatically skipped. Iteration over the postorder either is
possible also easily possible in reverse, so in a way a reverse postorder is
simultaneously also available.

Special values are function arguments. From the perspective of the called
function they are constants. Because of this, they don't appear in the control
flow graph in basic blocks---they are not operations that run. Through pointers,
other values like operations can link to them freely, but since sometimes
iteration over all arguments is needed, they are also explicitly linked from the
function itself. The number of function arguments doesn't have to be stored
explicitly---it is derivable from the `type` stored in `base`. Arguments could
be linked in an intrusive linked list, but storing them in an array and
explicitly storing the index of each argument proved to be convenient.

Function with variable number of arguments are currently not allowed to be
defined in TinyC, but they wouldn't require many special provision
here---handling of variable arguments such as with C's `va_arg` is stateful
(arguments are extracted one at a time ) and thus a corresponding `va_arg`
would have to be an operation over a `va_list`, thus no longer a constant, and
very distinct from `Argument`.

\secc Critical edge splitting

Critical edges and their implications for SSA deconstruction are explained in
section~\ref[sec:ssa-deconstruction]. In our implementation we chose to split
all critical edges.

The splitting can be realized in a single linear pass over all basic blocks in a
function. For each block, if it has multiple predecessors, we check for each
predecessor whether it has multiple successors---if it does, we have found 
critical edge. The edge is split by introducing a new basic block, which
contains just one jump operation into the successor, and has one successor---the
original predecessor of the edge. The new block has to also be set as the new
successor to the original predecessor and new predecessor to the original
successor.

Full implementation critical edge splitting in our implementation is shown in
listing~\ref[TODO]. The algorithm is rather simple and shows the use of our
design and data structures.

A few key points in the implementation deserve a mention:

\begitems

* Iteration over blocks is based on reverse post order, by iterating over the
precomputed post order in reverse. New blocks are {\em not} added to the post
order on the fly. We don't need to visit them in our algorithm, since they are
created without critical edges.

* Since the post order isn't updated during the run of the algorithm, it is
updated after it finishes.

* Adding an operation to the end of a basic block can be realized with a
`prepend_value` function, which prepends a value to a doubly linked list of
values. Since the block serves is part of the circular doubly linked list of
instructions, anything prepended to it will become the last instruction.

* We have to iterate linearly to find a predecessor/successor to replace in the
list of them.

\enditems

\begtt
void split_critical_edges(Arena *arena, Function *function) {
	for (size_t b = function->block_cnt; b--;) {
		Block *succ = function->post_order[b];
		if (block_pred_cnt(succ) <= 1)
			continue;

		FOR_EACH_BLOCK_PRED(succ, pred_) {
			Block *pred = *pred_;
			if (block_succ_cnt(pred) <= 1)
				continue;

			Block *new_block = create_block(arena, function);
			block_add_pred(new_block, pred);
			Value *jump = create_unary(arena, VK_JUMP, &TYPE_VOID, &succ->base);
			jump->parent = &new_block->base;
			jump->index = function->value_cnt++;
			prepend_value(&new_block->base, jump);

			FOR_EACH_BLOCK_SUCC(pred, s)
				if (*s == succ)
					*s = new_block;

			FOR_EACH_BLOCK_PRED(succ, p)
				if (*p == pred)
					*p = new_block;
		}
	}

	compute_postorder(function);
}
\endtt

We don't have to split {\em all} critical edges. From the perspective of our
back end, only edges to blocks with $\phi$ functions are problematic (see
sections~\ref[ssa-deconstruction, impl-ssa-deconstruction]). TODO

\secc SSA deconstruction

In the middle end we use value-based SSA (introduced in
section~\ref[ssa:value]) and also $\phi$-functions. In the back end we work with
machine instructions and virtual and physical registers. Thus we need to map
values to virtual registers (to be mapped to physical register later) and
replace uses of $\phi$-functions with copy (`mov`) instructions.

Assignment of virtual registers is simple in our representation. Since we
already assign an index to a `Value` for printing and array indexing purposes
(see section~\ref[sec:data-structures-middle-end]), then we can use the integer
indices directly as virtual registers. Since the virtual registers are assigned
directly from SSA form, the virtual registers will obey the single assignment
property and thus also correspond to {\em live ranges} for which we want to
allocate registers.

In our design we will do SSA deconstruction on the value-based representation
with method 1 from~\ref[Sreedhar1999] (see section~\ref[sec:ssa-deconstruction]
for more details). Doing it one the middle end value-based representation turns
out to be more straightforward and is also how Sreedhar's method nominally
works.

Sreedhar's method 1 consists of adding a copy instruction to each predecessor of
the block holding the $\phi$-operation and also an extra copy after the $\phi$
itself. Next all virtual registers involved in the $\phi$-instruction itself are
given the same virtual register and the $\phi$-instruction can be safely
removed.

In our implementation we operate with {\em values}, not instructions themselves.
So realizing copy instructions is not possible in the strict sense. Though we
can introduce {\em identity} operations. These are operations with single
operand, that just copy the operand. Since we will also be using the `index`
field of values as the virtual registers, we can assign the same virtual
registers to all the identity operations (to-be copy instructions), and also
replace the $\phi$-operation itself with a copy instruction, which copies from
the same virtual register by inserting a dummy value with the right index.

Consider for example the following function `f`, which returns 4 or 3 depending
on the truthiness of the first integer argument:

\begtt
f:
        v0: int = argument 0
block0:
        branch v0, block2, block5
block5: block0
        jump block4
block2: block0
        jump block4
block4: block2, block5
        v4: int = phi 4, 3
        ret v4
\endtt

We deconstruct the phi by introducing copies to block 2 and block 5 and changing
the phi to be a copy itself. All the copies need to be based on the same index
(which will become a virtual register):

\begtt
f:
        v0: int = argument 0
block0:
        branch v0, block2, block5
block5: block0
        v6: int = identity 3
        jump block4
block2: block0
        v6: int = identity 4
        jump block4
block4: block2, block5
        v4: int = identity v6
        ret v4
\endtt

Our value based SSA form always stays in SSA, since values can't be assigned.
But by making the indices semantically meaningful, we can deconstruct the SSA
with essentially copy operations.

Doing the SSA deconstruction on the SSA form has one significant advantage over
doing it on the machine form---the control flow graph is fully built already,
and we can easily just insert copies into predecessors. Doing it in the code
generator would not be as straightforward. We plan our code generator to be very
simple and single pass. Requiring copies in predecessor blocks becomes
troublesome in such generator, because at the time of lowering a block with
$\phi$-operations the predecessor block may have not been translated yet and
there may be no place for the $\phi$ insertion! Alternatively, we could flip the
idea of SSA deconstruction, and instead of inserting copies into predecessors
while processing the block, we can insert the copies to the predecessors, while
lowering {\em them}---this is easier, since we base this on the existence of
$\phi$-operations in {\em successor} blocks in the middle end representation.
However, in Sreedhar's method we have to introduce a new virtual register for
the copies, and suddenly coordinating when and how the virtual register is
allocated becomes more messy than doing it in the middle end IR.

We considered even doing the two step copying method (see
section~\ref[ssa-deconstruction]), which works well even when done when
processing the predecessors---the two rounds of copies are fully self contained
in the predecessors, no indexes have to be changed or coordinated elsewhere.
However, as the two copy method puts a lot of unnecessary burden on register
coalescing, we decided to go with Sreedhar's method 1 instead. Other methods are
able to save even more on future coalescing by inserting the copies in a smarter
way, by essentially doing the coalescing in the SSA deconstruction stage, but we
haven't yet found the need to justify a much more complicated algorithm for not
as that many benefits.

%We shall highlight one issue our simple $\phi$ deconstruction would have if we
%didn't split critical edges. Split critical edges guarantee us, that a block
%with multiple predecessors (i.e. a block which {\em can} have a
%$\phi$-operation), those predecessors only have {\em one successor}---the block
%with the $\phi$. In other words, all the predecessors end with a `jump`, not
%`branch`. If we had something like our example above, but without the split of
%critical edge:
%
%\begtt
%f:
%        v0: int = argument 0
%block0:
%        branch v0, block2, block4
%block2: block0
%        jump block4
%block4: block2, block0
%        v3: int = phi 4, 3
%        ret v3
%\endtt
%
%\begtt
%f:
%        v0: int = argument 0
%block0:
%        v5: int = identity 3
%        branch v0, block2, block4
%block2: block0
%        v5: int = identity 4
%        jump block4
%block4: block2, block0
%        v3: int = identity v5
%        ret v3
%\endtt

\label[sec:data-backend]
\secc Back-end

This section describes the reasoning, motivation and design of the data
structures core to the backend.

%Even though compiler backends are usually considered architecture specific,
%often we wa

%The basic data structure of the backend is a {\em machine instruction},
%represented in the compiler as struct type `Inst`, whose definition is show in
%TODO. The name has been kept deliberately short, since {\em everything} in a
%backend works with instructions. The goal of the `Inst` type is to represent
%machine instructions.

Machine instructions usually consist of two essential parts:

\begitems
* opcode

* operands
\enditems

The opcode usually packs together a few things:

\begitems
* the operation itself,

* the number of operands and their kinds,
\enditems

Operand kinds are usually:

\begitems
 * registers,
 * immediates
 * memory locations
\enditems

One possible representation for instructions contains the opcode and the
operands, where the operands also encode their kind. In such representations,
tagged unions can be used with advantage: the `kind` field in a structure
encodes the operand kind, while the (anonymous) `union` allows the storage for
the data (\"payloads") of all the different kinds of operands. This
representation of x86-64 instructions is sketched here TODO:

\begtt
typedef uint32_t Register;
typedef uint64_t Immediate;

typedef enum {
	ML_REG,
	ML_REG_DISP,
	ML_BASE_INDEX_DISP,
	ML_RIP_DISP,
	[...]
} MemoryLocationKind;

typedef struct {
	MemoryLocationKind kind;
	union {
		struct {
			Register reg;
		} reg;
		struct {
			Register reg;
			Immediate displacement;
		} reg_disp;
		struct {
			Register base;
			Register index;
			Immediate displacement;
		} base_index_disp;
		struct {
			Immediate displacement;
		} rip;
		[...]
	};
} MemoryLocation;

typedef enum {
	O_REG,
	O_IMM,
	O_MEM,
} OperandKind;

typedef struct {
	OperandKind kind;
	union {
		Register reg;
		Immediate imm;
		MemoryLocation mem;
	};
} Operand;

typedef enum {
	[...]
} OpCode;

typedef struct {
	OpCode opcode;
	Operand operands[];
} Instruction;
\endtt

Registers and immediates are represented directly with integer types.
Since we want to use the representation for instructions both {\em before} and
{\em after} register allocation, the integers representing registers will change
meaning---before register allocation they could either mean physical or virtual
registers and after register allocation only references to physical registers
would be allowed. Physical registers are used even before register allocation to
express {\em machine constraints} (see section~\ref[sec:regalloc-constraints]).

Operands array is represented with a flexible array member, which means that
number of allocated operands may be customized for each instruction. On
x86-64, instructions have at most three operands, so the array could be
statically sized, if wasting a bit of memory is acceptable for each instruction
is acceptable.

The number of operands can be usually derived from the opcode. (This also
implies that two and three operand multiplications have to be distinguished by
the {\em opcode}, even though they use the same {\em mnemonic} in assembly. See
section~\ref[x86-instructions] for more details.)

One problem with the representation is, that it allows invalid forms of
instructions to be easily represented. For example even three memory operands are
allowed, even though x86-64 only allows at most one memory operand. Since
`MemoryLocation` is the biggest member of the `union`, this also makes the
struct larger than necessary if only one memory operand is ever used.

The representation of x86-64 memory locations poses a bit of a problem, since
there are many ways of specifying a memory location (see
section~\ref[sec:x86-memory]). In the usual SIB mode essentially all components
(the base register, index register, scale and displacement) are optional.
Having a variant for each combination soon becomes unwieldy---not just to list
the variants, but to work with them. For example a function checking whether the
memory location uses an index register would have to check whether the kind of
the instruction is one of the ones that use an index register. Iteration over
all involved registers is also a bit cumbersome, because the registers are mixed
with immediates and different variants of the union store registers differently.

Instead a more flat representation, where all the fields are in the struct
(without any unions) can simplify matters, because it is more uniform:

\begtt
typedef struct {
	MemoryLocationKind kind;
	Register base;
	Register index;
	Immediate scale;
	Immediate displacement;
} MemoryLocation;
\endtt

This representation is however also problematic, because the `union` essentially
became implicit. Not all fields are valid, and this depends on the kind, which
still has to be investigated. Even iteration didn't become much easier because
of that. However, the differences between the memory location variants are
mostly due to fields being optional.

Instead of deriving the validity of the fields from the `kind`, we could reserve
one special value for each field to mean that the field is actually not present.
For displacements this is very natural, since absence of a displacement is very
similar to the displacement being zero.\fnote{It is very different in
the actual {\em encoding} of the instruction, since modes with displacement need
the displacement to be present even if it is zero. But this is a detail that can
be handled by a future stage of the compiler. Just like the fact that on the
x86-64 the displacement can be either 8 bit or 32 bit. Choosing the 8 bit form
may be deferred to much later stage of the compilation. This is different with
64 bit displacements. Since no instruction allows 64 bit displacement, the late
stage of compilation shouldn't ever need to encode such displacement, this
should be handled in instruction selection phase, because different {\em
instructions} have to be used.} For scale, similar thing applies---absence of
scale and scale being one are essentially equivalent. For registers a special value would
have to be reserved meaning that the register is not used. Normally, the
compiler would have to be very careful not to end up with an invalid combination
of absent fields, but since with x86-64 memory locations {\em all} fields are
optional, it is not even a problem. The representation could thus be simplified
to:

\begtt
typedef struct {
	Register base;
	Register index;
	Immediate scale;
	Immediate displacement;
} MemoryLocation;
\endtt

Iteration over the registers is a bit easier---as long as the invalid register
value is handled specially everywhere, it is possible to iterate over both
`base` and `index` unconditionally. Though still it is not ideal, because the
iterator would either have to be callback based or would have to employ code
duplication to work on both of the register fields. Arrays are much nicer for
iteration, which leads to the following idea:

\begtt
typedef struct {
	Register reg[2];
	Immediate imm[2];
} MemoryLocation;
\endtt

Arrays are ideal for iteration. Though the problem is now accessing specific
register or immediate, because hardcoded indices have to be used to access them
in the arrays. Some kind of nested anonymous `struct` and `union` combination
would have to be used to make both possible:

\begtt
typedef struct {
	union {
		struct {
			Register reg[2];
			Immediate imm[2];
		};
		struct {
			Register base;
			Register index;
			Immediate scale;
			Immediate displacement;
		};
	};
} MemoryLocation;
\endtt

Or macros could be used to abstract away the ugly indexing:

\begtt
typedef struct {
	Register reg[2];
	Immediate imm[2];
} MemoryLocation;

#define BASE(mem)         ((mem)->reg[0])
#define INDEX(mem)        ((mem)->reg[1])
#define SCALE(mem)        ((mem)->imm[0])
#define DISPLACEMENT(mem) ((mem)->imm[1))
\endtt

We have essentially {\em flattened} the memory location representation to a form
that allows easy addressing of individual fields as well as iteration.

Going back to the full instruction representation, we can apply similar
flattening principles. In particular, we can also make the representation even
more uniform, by noticing that even memory locations, the most structured kind
of operand, are made up of just registers and immediates. And registers as well
as immediates can be represented by integers. Now the memory locations,
immediates and registers can all fit into one uniform structure containing an
array of integer \"operands":

\begtt
typedef long Operand;

typedef struct {
	OpCode opcode;
	Operand operands[];
} Instruction;
\endtt

The meaning of individual slots in the operands array could either depend on the
opcode, or be the same for opcodes if we are willing to sacrifice memory and
keep representation of all instruction uniform---in that case the flexible array
member could also be replaced by fixed size array.

In the end, the following representation is what we arrived at:

\begtt
typedef uint32_t Oper;

typedef struct Inst Inst;
struct Inst {
	Inst *next;
	Inst *prev;
	uint8_t kind;
	uint8_t subkind;
	uint8_t mode;
	[...]
	Oper ops[];
};
\endtt

The split of opcode into `kind` and `subkind` is beneficial, because there are
opcodes that have very similar characteristics and handling them all at once
through the `kind` field is convenient. For example while there are 16 different
conditional jump opcodes, most of the time we don't care much about the
particular opcode (`subkind`), just the fact that it is an indirect jump
(`kind`). Other groups form nicely:

\begitems
* binary ALU operations (`add`, `sub`, `xor`, `and`, `or`, `test`, `cmp`, `imul`),
* unary ALU operations (`not`, `neg`),
* shifts (`shl`, `sar`),
* conditional moves (`cmovz`, `cmovl`, \dots),
* conditional jumps (`jz`, `jl`, \dots),
* set on condition (`setz`, `setl`, \dots),
* long division and multiplication (`idiv`, `imul`),
* etc.
\enditems

Conveniently conditional jumps, conditional moves, and conditional set
instructions are all based on the same 16 {\em condition codes} (see
section~\ref[x86-cc]), so `subkind` can be the condition code for all three of
these.

The `mode` field is what gives the meaning to the individual slots of `ops`
(operands). In theory, modes could just differentiate between the kinds of
operands used. For example, two operand instructions usually have the following
modes:

\begitems
* register, register
* register, memory
* memory, register
* memory, immediate
\enditems

However, for register allocation we want more information about the involved
registers. In particular, we want to iterate over all the {\em defined} and all
{\em used} registers separately. Registers used for forming memory locations are
only used, never defined, even if the instruction is store instruction which
writes a value---the value is stored to {\em memory}, not the register.
But other register operands may actually be either just defined, just
used or both. The  For example, consider the following instructions:

\begtt
mov rax, 1       ; rax just used
add rax, 2       ; rax used and then defined
imul rbx, rax, 4 ; rbx just defined, rax just used
cmp rax, rbx     ; rax just used, rbx just used
mov [rax], rcx   ; rax just used, rcx just used
mov rax, [rcx]   ; rax just defined, rcx just used
xor [rbp+16], -1 ; rbp just used
\endtt

Interestingly, `add`  and `cmp` which on the x86-64 are very similar and can be
grouped under the same `kind` use different mode even in the two register form:
`cmp` unlike `add` doesn't write to the first register. This is one of the
benefits of separating `mode` and `kind` instead of grouping them into a single
`opcode` field.

`kind`, `subkind` and `mode` are all small integers. And while
they have their meaning by themselves, other information may be associated with
them by considering them as indices into arrays with associated information for
each. Such arrays can be used to hold for example string representations of
kinds and subkinds.

These \"descriptor arrays" are more interesting for modes. They list for each
mode which part of the operands correspond to {\em defined} and which to {\em
used} registers:

\begtt
typedef struct {
	uint8_t def_start;
	uint8_t def_end;
	uint8_t use_start;
	uint8_t use_end;
	[...]
} ModeDescriptor;
\endtt

The `def_start` field gives the starting index of {\em defined} registers in `ops`,
`def_end` the index one past the last defined register. Analogously for {\em
used} registers. In practice, mode descriptors can look like this:

\begtt
ModeDescriptor formats[] = {
	// first only defined, first two used,
	// e.g. add rax, rcx
	[M_Rr] = { 0, 1, 0, 2, },
	// none defined, first two used
	// e.g. test rax, rcx
	[M_rr] = { 0, 0, 0, 2, },
	// first defined, second used
	// e.g. mov rax, rcx
	[M_Cr] = { 0, 1, 1, 2, },
	// first defined, first three used
	// e.g. add rax, [rcx+rdx]
	[M_RM] = { 0, 1, 0, 3, },
}
\endtt

The naming convention for modes has one operand per each letter after
underscore. Here the letter `r` means only used register, `R` used and defined register,
`C` only defined register and `M` stands for memory. This representation with
indices is nice for x86-64, because it allows {\em overlap} between the used and
defined registers. For example, the first mode used and later defines the same
register, hence it is covered by both ranges.

The reason for representing modes with small integers as indices into arrays, as
opposed to, for example, pointers to the descriptors (which would allow even
more opaqueness and similar target independence of the representation), is
not only because the integers can be much smaller than pointer, but ultimately
in our peephole optimizer we want to do {\em pattern matching} over the
instructions. Small integers more flexible in this regard: for
example matching one of multiple modes can be done with bitwise operations
instead of sequential comparisons of pointers. The integers are also more
flexible since more tables can be associated with the same indices (or the same
table can be used for subkinds of different kinds of instructions, like with the
condition codes for `cmovcc` and `jcc` x86-64 instructions, which proved to be
useful and elegant in the implementaiton).

One of the important aspects of the representation is, that each register
logically present in an instruction is present {\em only once} in the
representation. Alternatively a different representation could have forbidden
any overlap between used and defined registers, and if a register is both used
and defined, then it would be listed twice in the representation. However, since
there is only one register logically, only one of these two would get serialized
(e.g. or printed to assembly) and both of them would have to be kept in sync.

Keeping the two mentions of the same registers in sync is tricky in situations
like spilling, where uses and definitions of a register are replaced by loads
and stores through fresh virtual registers. In case a register that is both
used and defined, a single virtual register has to be used for both the load and
store.

Actual layout of the `ops` for x86-64 is illustrated by figure TODO. The
representation fits operands for each mode into just 6 operand slots. Not all modes
use all 6 slots, but they are allocated with them nonetheless, since it makes it
universally possible to just change mode and some operands to transform one
instruction into another. It also simplifies memory management, since the
it prevents fragmentation of memory and additionally free instructions kept in
one simple free list.

All 6 operand slots are needed for only one instruction: 3 operand `imul` with
register (1 slot), memory (2 register and 2 immediate slots) and immediate (1
slot) arguments.

Different slots are used for different purposes in different addressing modes.
But the assignment has been kept consistent as long as possible. For example,
most modes have one main register, which is always in the first slot. The single
memory operand always uses the slots 1 through 4. Slot 1 is also used in case
there are two registers involved---in which case memory operand is not used.

The fact that different slots have different purposes in different modes sadly
makes the representation a bit harder to understand. On the other hand, most of
it is hidden behind accessor macros and is really close to the encoding of the
x86-64 instructions. Of course the actual x86-64 encoding is much more compact.
But for example the following two instruction (load and store) are encoded in
the same way in both the backend representation and the actual x86-64 encoding:

\begtt
mov rax, [rdx+2*rcx]
mov [rdx+2*rcx], rdx
\endtt

In x86-64 serialization, the only difference is one bit in the opcode field
(the {\em direction bit}), while in our backend representation they have different
mode (`M_CM`, i.e. register and memory vs `M_Mr`, i.e. memory and register).

So far, the fact that `Oper` is defined as 32-bit unsigned integer has been
neglected. It may seem as a weird choice, considering that when an immediates
are generally considered signed by the architecture. In practice, there are two
good reasons for this:


%is used as either operand of an arithmetic instruction or as a value to
%be stored into memory, or as a displacement for a memory location it is
%understood as {\em signed}, i.e. in 64-bit contexts (both arithmetic and memory
%calculations) it is {\em sign extended} to 64-bits.

\begitems
* Immediates for most x86-64 instructions are limited to 32 bits. So limiting
them to the same range makes it obvious that special handling of larger
immediates is needed.

* Unsigned integers have defined representation and behaviour in C, as well as
on the x86-64. With signed integers this is not true: C doesn't define the
representation of signed numbers, while on the x86-64 signed integers use {two's complement}
representation. When, for example, evaluating constant expressions in the peephole
optimizer, the semantics of the x86-64 should be used, {\em not} the semantics
of machine the compiler is running on. Two's complement semantics should be used
for immediates, which is easier to do portably on unsigned numbers.
\enditems

Notably the \"move immediate into register" is the only x86-64 instruction
supported by our backend, which allows a 64-bit immediate. It stores the immediate
in two 32-bit operand slots. The large immediate is only accessible through two
accessor functions, which promote careful handling.

The fields of the `Inst` structure are accessed through macros:

\begtt
#define IK(inst) ((inst)->kind)
#define IS(inst) ((inst)->subkind)
#define IM(inst) ((inst)->mode)
\endtt

Naming is intentionally very terse, especially the peephole uses them {\em a
lot}---in our opinion long names would bring no sizeable benefits. The \"`I`"
prefix on all the macros stands for \"`Inst`", which accomplishes at least some
namespacing in C.

Another so far neglected aspect of the modes and operands were {\em labels}.
Even though ultimately all instructions compile out the memory location
displacements into up to 32-bit numbers, some addresses are {\em logically}
connected to (often named) objects. In the following example, the address of the
second field of a global variable has an address relative to the start of the
global variable:

\begtt
struct S {
    int a;
    int b;
};

S global_struct;

[...]
    global_struct.b = 5;
[...]
\endtt

As explained in section~\ref[x86-rip], we want to (or even need to) use
RIP-relative addressing for global variables. RIP-relative addressing uses the
instruction pointer (\"register" `rip`) and a 32-bit displacement. However, the
real displacement to `global_struct.b` or even to `global_struct` is not known
until after the final executable is linked together. As a zero-initialized
global variable, `global_struct` will be put into the `.bss` section.
`.bss` sections from all object files are merged into one `.bss` section, so
the real displacements cannot be known until all object files are linked
together. Hence earlier compiler stages need to somehow encode relative address
without knowing the final displacements. Without going into further details, in
ELF object files this is done through relocations. Assemblers usually support
labels, for example our target NASM would allow the following:

\begtt
mov qword [rel global_struct + 8], 5
\endtt

The `rel` keyword is important, as it forces RIP-relative addressing.
%
%RIP-relative addressing
%is perhaps better illustrated in AT\&T assembly syntax, for example as produced
%by GCC:
%
%\begtt
%%movl $5, 8+global_struct(%rip)
%\endtt
%
The assembler doesn't write out the \"label plus offset" information into the
object file, it resolves the address to a relative position in this object
file's (in this case) `.bss` section. Our compiler has to use the NASM syntax
with the label. But even if it did produce object files directly, it would be
necessary to associate the address of global variable's field with the global
variable itself.

For this {\em labels} are used. Like other `Oper`s, labels are just integers. In
this case each integer corresponds to an entry in a label array, which has a
pointer to a `Value`. Hence a label can freely point to global variable,
function, strings literal, etc. But importantly the label operand is stored
separately from the displacement operand---in a distinct operand slot. So there
is still a displacement involved, but only relative to the label. Since labels
are only interesting in RIP-relative addressing, which doesn't use neither base,
nor index register or scale, their slots can actually be reused---label is
stored inside scale's slot and special register value in base's slot is used to
signalize RIP-relative addressing. Displacement is stored in the same way in
both addressing modes.

Instructions also contain pointers to the previous and next one. Like `Value` in
the middle-end, they are linked together in an {\em intrusive linked list}.
Linked lists allow arbitrary insertions, deletions and reorderings in the middle
of a sequence of instructions, which are exactly the operation a peephole
optimizer does. Circular doubly linked list is especially nice for these
operations, because it has no special cases for insertion or deletions at the
end or the beginning of the linked list, simply because there is no real start
or end of the linked---every node may be presumed to be in the middle. For this
to work nicely, the {\em head} of the linked list (i.e. the pointer to the
linked list of instructions) should also be part of the linked list itself. With
our middle-end representation this was fairly easy, since \"instrucions" were
`Value`s with next and previous links, and basic blocks were values as well, so
the next and previous links of basic blocks served the purpose of the \"head"
and \"tail" fields. Similar thing would be needed here to allow machine basic
block to be the head, but also part of the linked list of instructions. This can
be achieved by \"inheriting" the `Inst` struct in machine basic block struct:

\begtt
struct MBlock {
	Block *block;
	size_t index;
	// `insts.next` and `insts.prev` are respectively the head and tail of
	// circular doubly linked list of instructions
	Inst insts;
};
\endtt

Adding `Inst` as field of `MBlock` means that it is now able to be part of the
linked list of instructions. The next and prev pointers serve as head and tail
respectively. Other fields like `mode` or `kind` are also inherited---special
values can be reserved for them and for example peephole optimizer's pattern
matching on instructions can then transparently skip machine basic blocks, if it
gets to them while investigating neighbours of instructions. As a result of the
representation, the rolling window in the peephole optimizer actually
transparently skips patterns that would reach out of bounds, because the kind of
the basic block will match no usual pattern.

Other than that, currently a machine block just links back to a middle-end basic
block. Since currently the backend lacks the ability to do big changes to
control flow, it can reuse the control flow (successors and predecessors) of the
middle-end representation.

Figure TODO shows examples of a few x86-64 as represented in a linked list of
`Inst` structures together with the head of the linked list an `Inst` in
`MBlock`.

\label[sec:imp-lower]
\sec Lowering

Code generation or {\em lowering} is the stage where we go from middle end
machine independent representation of values to machine dependent instructions.
We intended our code generator to be simple, with no case analysis. It receives
as input a RISC-like value based IR and will transform it to RISC-like x86-64
machine instructions.

We mostly solved SSA deconstruction in
section~\ref[sec:impl-ssa-deconstruction] and thus have already assigned virtual
registers. The lowering step should then simply go over the values in the
control flow graph one by one and translate them to corresponding instructions.
For following analysis in register allocation, we will preserve the basic
blocks, and as we don't have to add any new basic blocks, the translation is
done one-to-one in terms of basic blocks. Not as much with instructions, where
one middle end operation may require multiple x86-64 instructions.

In section about implementation of SSA deconstruction
(\ref[sec:impl-ssa-deconstruction]) we claimed that for values we simply use
their index as the virtual register. So in order to transform an operation like
the following addition:

\begtt
v5: int = add v3, v4
\endtt

We would like to emit an add instruction adding virtual register 3 and virtual
register 4 into virtual register 5. However, there is no such instruction on the
x86-64. We only have available two address code instructions (see
section~\ref[x86-two-address-code] for more details), where the result is put
into the first source register. Thus we can add virtual register 3 and virtual
register 4 into virtual register 3\fnote{Notice that we use the letter `t` as in
\"temporary" for virtual registers to distinguish them from values which are
prefixed with letter `v`, however for readability and ease of implementation we
keep the indices themselves mostly the same.}:

\begtt
add t3, t4
\endtt

To move the result into `t5` we could add a copy instruction after:

\begtt
add t3, t4
mov t5, t3
\endtt

However, to also preserve the original value of `t3` (since in general it may be
needed elsewhere), we do the copy {\em before}:

\begtt
mov t5, t3
add t5, t4
\endtt

We do similar translations for most of the ordinary binary operations. Unary
operations are very similar, we go from:

\begtt
v3: int = neg v2
\endtt

to:

\begtt
mov t3, t2
neg t3
\endtt

Instructions which have special register constraints are realized through copies
to or from the required registers (see section~\ref[sec:regalloc-constraints]),
thus effectively we perform {\em live range splitting} (explained more
thoroughly in section~\ref[sec:regalloc-splitting]). For example shifts which
require the shift amount to be in register `cl`:

\begtt
v3 = shl v1, v2
\endtt

\begtt
mov t3, t1
mov rcx, t2
shl t3, cl
\endtt

Or (signed) division which essentially requires the dividend to be 128
bits: lower 64 bits in `rax`, upper in `rdx`. Since we only need 64-bit
division, we can move the 64-bit dividend into `rax` and sign extend it into
`rdx` with the special purpose instruction `cqo` (\"convert quadword to
octoword"). After the division, the quotient is in `rax` and we need to move it
to a temporary as well:

\begtt
v3 = sdiv v1, v2
\endtt

\begtt
mov rax, v1
cqo
idiv v2
mov t3, rax
\endtt


In a RISC-like fashion, we base our operations on registers and don't
investigate whether we can use a more suitable addressing mode. One addressing
mode allows immediate values to be added to registers, i.e. for example we could
translate:

\begtt
v7: int = add v6, 1
\endtt

to:

\begtt
mov t7, v6
add t7, 1
\endtt

However, here we encounter a problem which we have so far neglected: how are
{\em constants} assigned virtual registers? Operations are embedded in the
concrete places in control flow graph and represent the {\em value} resulting
from an operation, which is only valid in the function (and to be concrete
speaking only in operations {\em strictly dominated} by the operation).

Integer constants are represented with the `Constant` struct (see
section~\ref[sec:data-structures-middle-end]). There are two practical problems
with assigning them virtual registers:

\begitems
* Our virtual register assignments are not global across all functions
functions, but only per-function.

* Even if we did for example assign the constant `4` the virtual register 4
across all functions, later we would have to assign it a {\em physical
register}. Blocking a physical for storing constants is unacceptable, if only
because much more constants can be used than there physical registers available.
\enditems

As opposed to operations, constants are {\em always} available. Because they are
also {\em compile time constants}, instead of assigning virtual registers to
constants, we can materialize the constants to registers {\em on demand}, when
needed---and each time into a new, fresh, virtual register. This way, we not
only keep the constants close to their uses (allowing easier peephole
optimization), but also keep the fresh virtual registers very short lived. This
way we alleviate much of the need for {\em rematerialization of constants} in register
allocation (see section~\ref[regalloc-rematerialization]---due to being short
lived, they are not great candidates for spilling, and many constants are easily
folded into other instructions through the use of better addressing modes.

For these reasons we don't put constants like integers into control flow graph
like other operations, i.e. addition for two numbers wouldn't be represented as:

\begtt
v1: int = constant 1
v2: int = constant 2
v3: int = add v1, v2
\endtt

But as:

\begtt
v3: int = add 1, 2
\endtt

For many of the constants (like functions, globals and string literals) we can
also choose either form, e.g. for functions:

\begtt
v4: *(int) -> int = function f
[...]
v6: int = call v4, v5
\endtt

But also:

\begtt
v6: int = call f, v5
\endtt

Materialization of constants has to be done every time we encounter the constant
being used as an operand. As our code generator doesn't do any case analysis, it
suffices to materialize the constant into a register. For example, we would
translate

\begtt
v3 = add 1, 2
v4 = call f, v3, 1, 2
\endtt

To the following:

\begtt
mov t10, 1
mov t11, 2
mov t3, t10
add t3, t11

lea t12, [f]
mov t13, 1
mov t14, 2

mov rdi, t3
mov rsi, t13
mov rdx, t14

call t12
\endtt

Above, the first group instructions realizes the addition including the
materialization of constants. In the second group, the constants for the call
are materialized---the address of the functions is loaded into a register, just
as well as the constants. In the third group, there are copies to registers imposed by the
calling registers---i.e. first argument to `rdi`, etc. And finally in the last
instruction the function is called indirectly through a function pointer.

The code generated by our simple code generator is very inefficient, but
produces correct, obvious code, that can be improved upon in the peephole
optimization stage. The peepholer is not only more suited to do case analysis,
but it can also do it across the code expansions of more operations. For
example, we could make the code generator generate direct function calls
(\"label calls") instead of indirect calls (\"register calls"), but we want to
transform indirect calls to direct even in the case they didn't seem as direct
at first, but only appeared so after other (peephole) optimizations.

Even use of immediates in instructions isn't without case analysis: generally
only 32-bit immediates (which are sign extended) are allowed for instructions.
For example the following the additions have different possible {\em optimized}
versions:

\begtt
v7: int = add v6, 2147483647

v9: int = add v8, 2147483648
\endtt

\begtt
lea t7, [t6 + 2147483647]

mov t10, 2147483648
mov t9, t8
add t9, t10
\endtt

The code generation is mainly driven by a function called `translate_value`,
which is responsible for translating the {\em operations}. Before any operation
is translated, the {\em operands} are translated with
`translate_operand`---mainly translation of operands involves materialization of
constants. Parts of the functions are illustrated below with some of the utility
functions.

\begtt
void translate_value(TranslationState *ts, Value *v) {
	Oper ops[256];
	Value **operands = value_operands(v);
	size_t operand_cnt = value_operand_cnt(v);
	for (size_t i = 0; i < operand_cnt; i++)
		ops[i] = translate_operand(ts, operands[i]);

	Oper res = v->index;

	switch (v->kind) {
		[...]
	}
}
\endtt

\begtt
TODO
\endtt

\sec Peephole optimization

In our compiler, peephole optimization serves the roles of both the instruction
selection and the classic peephole optimization that runs as the last step of
the backend for code cleanup. This is because we chose to use a single back end
representation---machine instructions. Our peephole optimizer shall look at
sequences of instructions and replace them with {\em better} alternatives, where
better is often faster, shorter, or even just more canonical to allow subsequent
optimizations. We should also be careful about optimizations that transform code
that {\em doesn't} allow subsequent optimizations.

In the next few subsections, we will look at optimizations applicable to x86-64
machine instructions (or in general) and also into the implementation in our
back end (section~\ref[sec:imp-peephole-imp]).

\label[sec:imp-peephole-local]
\secc Local optimizations

Local optimizations are the classic peephole optimizations based on small
windows {\em peepholes} into instructions. They are very limited, since they
have only very local knowledge about the code.

As mentioned when discussing coalescing (section~\ref[regalloc-coalescing]), the
register allocator tries to assign virtual registers involved in the same move
instruction the same physical register. In that case we can end up with
instruction like:

\begtt
mov rax, rax
\endtt

Copy instruction where the source is the same as destination can be deleted
freely, because they have no effect---they don't even set flags. Due to our
handling of callee saved registers, even for the simplest functions, like one
returning the integer 1:

\begtt
one:
block0:
        ret 1
\endtt

We generate code that generally looks like this:

\begtt
one:
.L0:
        ; entry
        push rbp
        mov rbp, rsp
        sub rsp, 42
        mov t19, rbx
        mov t20, r12
        mov t21, r13
        mov t22, r14
        mov t23, r15
        mov t24, 1
        mov rax, t24
        mov rbx, t19
        mov r12, t20
        mov r13, t21
        mov r14, t22
        mov r15, t23
        mov rsp, rbp
        pop rbp
        ret
\endtt

In most functions with low register pressure, often none of the 5 callee saved
are spilled. Thus after register allocation we get:

\begtt
one:
.L0:
        ; entry
        push rbp
        mov rbp, rsp
        sub rsp, 42
        mov rbx, rbx
        mov r12, r12
        mov r13, r13
        mov r14, r14
        mov r15, r15
        mov t24, 1
        mov rax, t24
        mov rbx, rbx
        mov r12, r12
        mov r13, r13
        mov r14, r14
        mov r15, r15
        mov rsp, rbp
        pop rbp
        ret
\endtt

Which we simplify to:

\begtt
one:
.L0:
        ; entry
        push rbp
        mov rbp, rsp
        sub rsp, 42
        mov rax, 1
        mov rsp, rbp
        pop rbp
        ret
\endtt

For an implementation of this particular peephole pattern, we need to match the
{\em exact opcode}. In this case we need to match a `mov` from register to
register. In our representation of instruction (described thoroughly in
section~\ref[sec:data-backend]), we split the opcode information into three
parts: {\em kind}, {\em subkind} and {\em mode}. This split is especially
targeted at peephole patterns, where we can get a lot of patterns sharing the
same kind, but with different subkinds, or the patterns can be applicable to
same instruction with different modes (or perhaps with slight adjustments for
each mode).

Kind (\"instruction kind", `IK`) for this pattern would be `IK_MOV`, subkind
`MOV`\fnote{There are multiple variations of the `mov` opcode, in particular for
different sizes (i.e. 8-bit vs 64-bit loads/stores).} and mode `M_Cr` (i.e.
first register only written, second register only read). And we also need to
check whether the two registers are the same. These checks don't need to occur
in the order described here---as long as they all succeed, they can be done in
any order, however filtering first based on the kind and subkind can quickly
filter out instructions not worth checking.

The entire implementation of the pattern could be the following:

\begtt
if (IK(inst) == IK_MOV && IS(inst) == MOV && IM(inst) == M_Cr
		&& IREG(inst) == IREG2(inst)) {
	inst->prev->next = inst->next;
	inst->next->prev = inst->prev;
}
\endtt

Here, the macros `IK`, `IS` and `IM` respectively allow terse access to the
`kind`, `subkind` and `mode` fields of the instructions. If the pattern matches,
we can remove the instruction by unlinking it from the circular doubly linked
list by pointing the previous node's next field to the next instruction and
likewise for the `prev` field of the next instruction.

There are more single instruction peephole patterns that are possible. For
example. For example a comparison of a register with a 0:

\begtt
cmp t12, 0
\endtt

Is the same as using `test` instruction (which performs bitwise and of the two
operands and sets flags based on the result) on the register with itself:

\begtt
test t12, t12
\endtt

This saves 4 bytes on the immediate 0. The implementation of the pattern can
look like this:

\begtt
if (IK(inst) == IK_BINALU && IS(inst) == G1_CMP && IM(inst) == M_ri
		&& IIMM(inst) == 0) {
	IS(inst) = G1_TEST;
	IM(inst) = M_rr;
	IREG2(inst) = IREG(inst);
}
\endtt

We check for kind, subkind, mode and then the special properties we are looking
for, in this case that the 32-bit immediate is zero. To realize the replacement
of the instruction we don't have to allocate a new one and free the old one. We
can use the existing instruction and modify it in place. This saves as quite a
bit of relinking we would have to, but more importantly it allows us to keep
some fields {\em unchanged}. Here we for example don't need to change the first
register---it stays the same for both instructions. In more complicated patterns
this can be more interesting---often we e.g. don't care about how the address of
a load instruction is calculated, but just that it is a load instruction.

When considering peephole patterns we don't just need to find just {\em useful}
patterns, they need to be {\em reachable} as well. For example, the
transformation of comparison with zero to `test` presented above is useful
(produces shorter instruction encoding), but it as of now, it is not
reachable---our code generator never generates comparisons with immediates.

In this case, immediate operands are in fact very common, because they can be
created with simple peephole optimization like this one:

\begtt
mov t13, 0
cmp t12, t13

cmp t12, 0
\endtt

This pattern is more general---certainly we don't need it to be limited to just
zero. But we need to consider the fact, that on the x86-64 immediate operands
can be at most 32-bit signed values, so \"small" positive and negative
immediates (which are the most common ones) are applicable. The pattern is also
not limited to the `cmp` instruction---the pattern will also work with `add` or
`sub`. All of these are conveniently in the `IK_BINALU` subkind, since they
have very similar addressing modes. But the mode is different---`cmp` like
`test`, but unlike e.g. `add` {\em doesn't} write to any register---it just
changes the flags. Thus we don't need to match for any particular {\em subkind},
but we need to check for one of the two {\em modes}. Here an implementation
could look like this:


\begtt
if (IK(inst) == IK_BINALU && (IM(inst) == M_Rr || IM(inst) == M_rr)
		&& IK(prev) == IK_MOV && IS(prev) == MOV
		&& IM(prev) == M_CI && IREG(prev) == IREG2(inst)
		&& pack_into_oper(get_imm64(prev), &IIMM(inst))) {
	inst->mode = IM(inst) == M_Rr ? M_Ri : M_ri;
	IREG2(inst) = R_NONE;
}
\endtt

As hinted above, we check the current instruction (`inst`) for the right kind
and mode, but we also investigate the previous instruction (here already in the
local variable `prev`). We expect it to be a `mov` with mode `M_CI`, i.e. we
write an immediate into a register. The big letter `I` signifies a 64-bit
immediate---moves of immediates into registers are one of the very few
instructions where 64-bit moves are allowed. Hence before applying this pattern
we also need to check whether the immediate from the `mov` instruction actually
fits into 32-bit with the sign bit. Here we do that, with helper functions which
extract the 64-bit immediate from two 32-bit `Oper` slots, and then try to fit
it into the 32-bit `Oper` of the arithmetic instruction. Since the function
performs side effect, but returns a boolean indicating success, we run it last.
Then we update the mode of the instruction accordingly---we will use `i` for the
second operand (since it is now a 32-bit to-be sign extended immediate), but we
need to preserve the read or write property on the first register. Also, since
generally we keep unset fields set to zero, we also set the (now unused) second
register field to 0 (or the symbolic constant `R_NONE` which is defined to be
0)---we don't need to do this, since as long as the mode doesn't need the slot,
it is not going to be read, but the zero-initialized property is sometimes
useful and we chose to preserve it.

Unfortunately, the pattern shown above is still not very useful---it expects two
the immediate move and arithmetic instructions to immediately follow each other.
But for example subtractions are lowered into code like this:

\begtt
mov t13, 10
mov t14, t12
sub t14, t13
\endtt

The copy in the middle prevents the pattern to match. Though if `t14` and `t12`
get coalesced:

\begtt
mov rax, 10
mov rcx, rcx
sub rcx, rax
\endtt

The pattern will match after the previous redundant copy applies:

\begtt
sub rcx, 10
\endtt

Though the code is better, and uses a better addressing mode, we missed an
optimization---by waiting until after register allocation, we had to allocate
a physical register for the short lived temporary. This could have caused a long
lived temporary to get spilled or at least makes the register allocation problem
harder by keeping more interferences, etc. Thus we really do want to make
optimizations in the first peephole pass {\em before} register allocation. Here
we can notice the pattern with the move in the middle and introduce a three
instruction pattern. The new pattern should probably check fully that the middle
instruction is a `mov` in the form we expect, or at least it has to make sure,
that the register with the immediate isn't {\em overwritten} in the middle
instruction, which would make the optimization invalid.

Apart from register or immediate operands, instructions can also have one memory
operand. Instructions often use operands in memory---C variables are nominally
stored on the stack, just like spilled registers. Addresses of variables on the
stack (through `alloca` instructions) are {\em constants} in our implementation,
hence are often very close to their use, making it possible for simple peephole
patterns to optimize the memory access. For example, a load from a local
variable might look like this:

\begtt
lea t25, [rbp-24]
mov t26, [t25]
\endtt

And can be easily optimized to this:

\begtt
mov t26, [rbp-24]
\endtt

The address computation in a `lea` (\"load effective address") instruction uses
the same memory addressing as all other instructions referencing memory. Similar
memory addressing optimizations are also applicable for stores and operands of
arithmetic instructions.

Peephole optimizations don't only involve simple identities or addressing mode
changes. Opportunities arise for example for eliminate a load from a just stored
address:

\begtt
mov [G], t27
mov t28, [G]
\endtt

Instead of loading from the global variable `G`, it is possible to use the
register `t27`. However, all uses of `t28` currently refer to `t28` and we can't
easily changed them without additional bookkeeping. Merging the registers into
one practically is just like coalescing, and we can encourage the register
allocator to coalesce just by rewriting the load into a copy:

\begtt
mov [G], t27
mov t28, t27
\endtt

The advantage to leaving the coalescing to the register allocator is, that it
has the knowledge about interferences and will not combine `t28` with `t27` if
they {\em interfere} (see section~\ref[regalloc-graph-coloring]).

\label[sec:imp-peephole-flag]
\secc Flag based optimizations

Well known arithmetic identities like addition of zero or multiplication by one
are often folded in the middle end. Though, with an optimizing back end,
opportunities for such optimizations often arise again. Hence our peephole
optimizer should be able to do them as well. They are not as straightforward
though. Take as an example addition with zero:

\begtt
add rax, 0
\endtt

While the register `rax` isn't changed by the instruction at all, the {\em
flags} register gets updated based on the {\em result} of addition---in this
case based on `rax`. If a later instruction depends on the flags, we can't just
delete this instruction.

Similar need for flags prevents the use of `lea` for arithmetic, because unlike
arithmetic instructions the `lea` instruction {\em doesn't} set flags. I.e. the
following addition:

\begtt
mov t26, t18
add t26, t34
\endtt

Can only be optimized to `lea` if the flags are not observed after the `add`:

\begtt
lea t26, [t18+t34]
\endtt

Normally our code generator doesn't generate `lea` instructions except for
materialization of constants (see section~\ref[sec:imp-lower]). However, as shown in
the previous section (section~\ref[sec:imp-peephole-local]), we are able to
transform uses of `lea`s in for example loads and stores or other instructions
that allow memory operands. So transforming address calculations involving `add`
or `imul` into `lea` instructions is important to allow these better addressing
modes. Transforming even ordinary arithmetic (i.e. not address calculations)
into `lea` instructions is also almost always an improvement---`lea` unlike most
other instructions can write a different register than any of the read
registers, so its use may constrain register allocation less.

As our code generator doesn't do any case analysis, it inserts explicit `cmp`
instructions for setting the flags. Thus we could perform the optimizations
described above freely. But then we would have to give up the optimizations that
are able to {\em use} the flags. For example a loop might be decrementing a
register until it becomes zero:

\begtt
sub rax, 1
test rax, rax
jz .L3
\endtt

Using `test` to set flags according to `rax`'s value is redundant, since the
flags are already set by the previous arithmetic instruction. We have shown like
optimizations can replace comparison with zero to `test` with itself in previous
the previous section (section~\ref[imp-peephole-local]).

Ideally we would like to support both kinds of optimizations---removing
redundant settings of flags when the flags are already set by arithmetic
instructions, as well as making use of instructions not setting the flags if the
flags are not needed. For this we need to track whether flags set by an
instruction are {\em observed} later. We do that by introducing three flags to
each instruction:

\begtt
struct Inst {
	[...]
	bool writes_flags;
	bool reads_flags;
	bool flags_observed;
	[...]
};
\endtt

And then compute the `flags_observed` property based on how the instructions
read and write flags in a single backwards pass over a basic block:

\begtt
bool flags_needed = false;
for (Inst *inst = mblock->insts.prev; inst != &mblock->insts; inst = inst->prev) {
	inst->flags_observed = flags_needed;

	if (inst->writes_flags)
		flags_needed = false;

	if (inst->reads_flags)
		flags_needed = true;
}
\endtt

This way instructions which set flags which are needed in the future, as well as
instructions through which needed flags pass through are marked. More
complicated data-flow analysis is not needed, since we don't ever assume
anything about the state of flags at basic block starts and can thus assume the
flags are not needed at the end of a basic block.

The boolean flags on `Inst` are flexible, but unnecessarily since whether the
instruction writes or reads flags can be determined by the instruction {\em
kind}, so it would be possible to have an array mapping kinds (as indices) to
the boolean flags. But since in our implementation instructions had spare space
for the two flags due to alignment, we didn't yet find the need for such
improvement.

Now with sufficient information, we may realize simple arithmetic identities
like addition of zero or multiplication by one: depending on whether the flags
are observed, we can either remove the instruction completely, or change it to
`test` of the register with itself, which is a better way of setting the flags
then an arithmetic instruction, if only because it doesn't involve an immediate.

The information about flag observation can be used for even more optimizations.
For example, `cmp` and `test` instructions can be removed if the flags set by
them are not observed:

\begtt
cmp t3, t5 // can be deleted if flags not observed
\endtt

Some instructions are problematic with regards to flags. For example `inc` and
`dec` instruction don't set the carry flag, but do set other flags and shift
instructions (like `shl`) don't set the flags when the shift amount is equal to
zero. The special behavior of `inc` and `dec` isn't that problematic, since we
currently don't use the carry flag for anything. Shifts are more problematic,
since if we expect them to set flags and optimize based on that like above, we
might actually get into situation where the flags are not set and instead a
previous incorrect value of the flag register would be read. For this reason we
don't set `writes_flags` for shift instructions. This way, if flags are needed
after the instruction, the peephole optimizer will be forced to leave the `cmp`
instruction inserted by the code generator. Though this will then be marked as flags
pass-through of flags by our algorithm above, though this isn't a problem with
code patterns generated by our generator, where we always set flags and read
flags in the same expansion, not across expansions.

\label[sec:imp-peephole-use-def]
\secc Use-def based optimizations

In section~\ref[sec:imp-peephole-local] we showed an addressing mode
optimization, which took an immediate move instruction or `lea` loading an
address of local or global variable and folded the constant into other
instruction. For example from:

\begtt
mov t13, 0
cmp t12, t13
\endtt

To:

\begtt
cmp t12, 0
\endtt

However, there are a few problems with this optimization:

\begitems

* The optimization is actually incorrect. We can't remove the definition of
`t13`, since there may be other uses of the register. In practice, since during
code generation we {\em materialize constants} for each use (see
section~\ref[imp-lowering]), there are {\em no} other uses of the constants. But
other optimizations can change that.

* It is not powerful enough. The optimization only takes place when there are
two (physically) consecutive instructions.
\enditems

We can improve on both points by tracking {\em uses} and {\em definitions}. We
call this {\em use-def} based peephole optimization. Our approach and idea are
pretty simple: for each register we track the number of definitions, the number
of uses and if there is only one definition, the defining instruction. These
properties are precalculated once, before the peephole pass, and in the
implementation we actually do it as part of the backwards pass over blocks where
we track flag information.

This information about the number of definitions and uses allows us to remove
instructions that define a register with no uses (provided that the instruction
doesn't have other side effects). Such peephole pattern may be even as abstract
as this:

\begtt
if ((IM(inst) == M_CI || IM(inst) == M_Cr || IM(inst) == M_CM
		|| IM(inst) == M_Cn) && use_cnt[IREG(inst)] == 0) {
	def_cnt[IREG(inst)]--;
	for_each_use(inst, decrement_count, use_cnt);
	inst->prev->next = inst->next;
	inst->next->prev = inst->prev;
}
\endtt

Here we check the {\em mode} of the instruction and if is one of those that
define the first register, and the register has no uses, then we can remove the
definition by unlinking the instructions and decrementing the definition count
of the register. To allow further optimizations, we also decrement the use
counts of the registers used in this instruction---which may in effect
lead to them being unused---though the implications of this on the peephole
optimization are only discussed later, in section~\ref[sec:imp-peephole-imp].

The tracking of the single definition is more interesting for thorough
addressing mode improvements. For example, we can for an instruction in one
addressing mode (say using only registers) check, whether e.g. the second
register is defined as an immediate that fits into 32-bits, or a memory location
computed using `lea`, etc. and use these definitions directly in the instruction
with a better addressing mode. Additionally, if the original register with the
constant is used only once, the definition can be deleted. With a right set of
functions abstracting the check for constant definition and the folding, we can
specify the peephole optimization patterns in a pretty terse way. Just for
illustration:

\begtt
if (IK(inst) == IK_MOV && IS(inst) == MOV && IM(inst) == M_Mr
		&& try_replace_by_immediate(mfunction, inst, IREG2(inst))) {
	IM(inst) = M_Mi;
}
\endtt

Above, if the instruction is a store instruction (moving a value from register
to memory location) and the register is actually a 32-bit constant, then we can
change the addressing mode to \"memory-immediate". The helper function takes
care of checking the (single) constant definition and packing the immediate into
`IIMM(inst)`, which can of course fail if the constant cannot be expressed as a
32-bit signed number. And just for completeness, the folding of memory locations
can be done like this:

\begtt
if (IK(inst) == IK_MOV && IS(inst) == MOV && IM(inst) == M_CM
		&& try_combine_memory(mfunction, inst)) {
	// nothing to do
}
\endtt

This pattern can fold the memory location of the address in a load instruction.
I.e. to go from the following:

\begtt
lea t25, [rbp-24]
mov t26, [t25]
\endtt

To:

\begtt
mov t26, [rbp-24]
\endtt

The function `try_combine_memory` is actually pretty involved, because it not
only allows folds of constants, it tries to combine any two memory locations
together. The function doesn't care about the actual addressing mode of the
instruction the peephole executes on (i.e. the fact that it is a load in the
example above), it only looks at the memory specification. If the {\em base}
register used in the specification (`t25` above) has only one unique definition,
which is a `lea` instruction and the {\em base} register of the `lea` (`rbp`
above) is either `R_NONE` (meaning that RIP-relative, i.e. label addressing is
used) or `rbp` or other register with {\em one unique definition} then it will
try to combine the base, index, scale and displacement fields of the two memory
locations. So for example, even the following two would get folded:

\begtt
lea t27, [a-4]
mov t28, [t27+8]

lea t29, [4*t25] // say t25 has only one definition
mov [t29+16], 1
\endtt

The implementation currently doesn't support all possibilities that {\em could}
be optimized. But sufficient number of them found in real programs using the
advanced addressing modes are supported.

Similar function is used to transform `lea` instructions loading function
addresses and indirect calls into direct calls:

\begtt
if (IK(inst) == IK_CALL && IM(inst) == M_rCALL && try_combine_label(mfunction, inst)) {
	IM(inst) = M_LCALL;
}
\endtt

I.e. this goes from:

\begtt
lea rax, [function]
call rax
\endtt

To:

\begtt
call function
\endtt

\label[sec:imp-peephole-block]
\secc Inter-block optimizations

During code generation (section~\ref[imp-lowering]) we lowered jump and
conditional jump operations at the end of basic blocks into `jmp` and `jcc`
instructions. Depending on how the blocks are ordered, some of the instructions
are unnecessary---unlike our middle end representation, which needs explicit
jumps from each basic block, conditional jumps in machine code fall-through to
the next instruction if the condition is not satisfied. This leaves
opportunities for peephole optimization.

It is no longer enough to consider instructions alone---we need to also consider
which block is the following one. Thus the optimizations depend on particular
linearized {\em order} of the blocks. We will come back to the issue of block
ordering in section~\ref[sec:imp-peephole-imp], but for now it suffices to
assume that when performing peephole optimizations at the end of a block, we
know which block is the following \"fall-through" block.

The most straightforward optimization possible, is to remove an unconditional
jump to the following block:

\begtt
	[...]
	jmp .BB5

.BB5:
	[...]
\endtt

The `jmp` instruction above is always redundant. This also applies to
unconditional jumps which result from translation of {\em conditional jumps}:
T
\begtt
	[...]
	jge .BB4
	jmp .BB3

.BB3:
\endtt

If the following block is not the target of the unconditional jump (`jmp`), but
of the conditional one (`jcc`), we can still perform the optimization, but we
need to {\em invert the condition}:

\begtt
	[...]
	jge .BB7
	jmp .BB8

.BB7:
\endtt

\begtt
	[...]
	jl .BB8

.BB7:
\endtt

Here we changed \"greater-or-equal" condition to \"less-than". This is done
simply by changing the subkind of the instruction, since the subkind is the
x86-64 condition code. There are 16 condition codes and inverse of a condition
code can be found simply by inverting the least significant bit of the condition
code.

It would seem that when the jump instruction is optimized out and implicit
fall-through is used, then the {\em basic block label} can also be deleted. But
this is only possible when there are {\em no other} jumps to the
label---unconditional or not. Deleting the label would essentially mean a merge
of two basic blocks, and basic blocks can only be entered at the start, so
the merge of basic blocks is only possible if there isn't a jump into the
\"middle".

Even when the merge is possible, we can no longer speak about basic blocks,
because there could be jump instructions in the middle---notably, if a block
ended with a conditional jump compiled into `jcc` and `jmp` and `jmp` gets
optimized away and the block gets merged with the following fallthrough block,
the `jcc` instruction leaves the block in the middle. The block is no longer
basic. For example if we optimize from:

\begtt
.BB6
	jge .BB7
	jmp .BB8

.BB7:
	mov t17, t15
	jmp .BB8

.BB8:
	[...]
\endtt

To:

\begtt
.BB6
	jl .BB8
	mov t17, t15
.BB8:
	[...]
\endtt

We see how the label `.BB7` corresponding to block 7 gets eliminated, because
blocks 6 and 7 are merged. The resulting merged block 6 is no longer
basic---it can be left in the middle. Some distinction of {\em blocks} is still
useful, since they still correspond to sequences of instructions which are
executed after each other, but the blocks are no longer basic, because not all
instructions have to execute. But this is fine from the perspective of peephole
optimization, which optimizes physically neighbouring instructions. There we can
take advantage of the fact that the sequence is not entered in the middle, and
we don't mind that it can be left in the middle. And even following up to the
example above, we can notice that the whole purpose of the conditional jump
above is to skip the copy instruction. On the x86-64 architecture the same
behavior can be achieved with a `cmovcc` instruction with an inverted condition
code:

\begtt
.BB6
	cmovge t17, t15
.BB8:
	[...]
\endtt

This again, may leave the block 8 mergeable into block 6 if there are no other
jumps to it. This example is very close to reality---similar short blocks
conditionally entered, unconditionally left, and with zero or more copy
instructions in them are created by critical edge splitting
(see section~\ref[imp-critical-edge]) and SSA deconstruction (see
section~\ref[imp-ssa-deconstruction]). Coalescing is often able to assign the
virtual registers the same {\em physical} register and the copies are often
removed. By removing coalesced copies, transforming simple jumps around copies
into `cmovcc` and by merging blocks through peephole optimizations we can often
get straight code that is even more amenable to peephole optimization.

To track the number of references to a block (which for our purposes here is
more like a label). When last reference of the block is removed from physically
preceding block, the two blocks can be merged together.

\label[sec:imp-peephole-imp]
\secc Implementation

In the previous subsections we presented intra-block, but also inter-block
peephole optimization patterns and how they can be matched and applied with our
representation of instruction. In this section we solve the problem of putting
the peephole optimization together, i.e. how to apply subsequent optimizations
even on the same instruction, how to represent and move the peephole sliding
window, etc.

Considering only intra-block optimization first, we want to go through the
instruction sequence, try to match a pattern and apply the optimizations, and to
readjust the peephole window according to what the optimization does to the
instructions. Our patterns are also not of one uniform size: we have patterns
that match on a single instruction, but also patterns matching up to four
instructions.

What works fairly well, is to iterate over the instruction sequence in forward,
while having a notion of the {\em current} instruction. When iterating over the
instructions we root all the peephole patterns to the current instruction, such
that the current instruction always constitutes the very end of the peephole
window---matching on the context is done only on on {\em previous} instructions.
If a pattern matches, it executes and modifies the instructions and perhaps even
reorders them. Then, a new instruction is set as the current one and new
matching is tried again with the same patterns. Since the {\em current}
instruction is the last one in the peephole window, each optimization should set
the current instruction to be the instruction most further back on which
peephole optimizations are worthwhile, i.e. the instruction furthest back, which
{\em changed}. This is because changed instructions can allow match of new or
different patterns, so it is beneficial to go back and try them.

This construction has another advantage---it may even be used in an online way.
For example, each time the code generator appends a new instruction, the
peephole machine described above could be run on the last instruction,
optimizing the tail of the produced code.

A state machine would be a great match for this kind of problem. The state
machine could in its states remember the previous instructions encountered, and
provide efficient matching and lookup of patterns to execute. In practice, the
patterns, especially the use-def based (section~\ref[sec:imp-peephole-use-def])
don't fit into this model that well. The predicates used for matching the
instructions are not always straightforward and often benefit of being coded in
C, like shown in previous sections. Creating a state machine generator also
doesn't pay off that much until multiple target architectures are needed, so we
did ultimately settle on ad hoc pattern matching and optimization with C code.

In our implementation, we iterate the instruction sequence, try to match each
pattern in turn, if any of them matches, its associated code executes and
also sets the new current instruction, which we try matching next. If no pattern
matches, we set current instruction to the instruction following the current
one.

This is how it looks like in the actual implementation:

\begtt \optparams
Inst *inst = mblock->insts.next;
while (inst != &mblock->insts) {
	if (<pattern 1>) {
		<rule 1>
		inst = <new current 1>;
		continue;
	}

	if (<pattern 2>) {
		<rule 2>
		inst = <new current 2>;
		continue;
	}

	[...]

next:
	inst = inst->next;
}
\endtt

The pattern matching and rule execution is exactly like hinted in previous
sections. If any pattern matches, new `inst` (current instruction) is set to go
back sufficiently and the pattern matching \"restarts". Otherwise the process
continues with the next instruction.

The choice to do peephole optimization on instruction sequences limited to
blocks instead of doing it on whole functions is not an obvious trade-off. In
our architecture we do peephole optimization twice---before and after register
allocation. For register allocation (due to liveness analysis,
section~\ref[TODO]) as well as flag analysis
(section~\ref[sec:imp-peephole-flag]) we want to have {\em basic blocks}. But,
for peephole optimization itself, we don't care about basic block that much. We
can have either blocks that are not basic as hinted in
section~\ref[imp-peephole-block], or just a single instruction sequence for a
whole function. Though still with a single instruction sequence we need a way to
do jumps, which could be represented with {\em label} nodes, similar to what is
done in~\cite[Wulf1975]. By having a linked list of instructions and labels, we
could iterate over the whole function and not handling blocks as a special
case---as labels they would be part of normal patterns. After a label would
become unreferenced, it would be deleted and instructions formerly separated
would become eligible for peephole optimization.

However, since we do a round of peephole optimization before register
allocation's liveness analysis, we would need to derive information about basic
blocks for it specially. Then we would have to be careful about keeping the
basic block boundary representation valid even through the spill stage, which
inserts new instruction (which depending on the boundary representation can be
problematic e.g. if a load is inserted {\em before} the first instruction in a
basic block) or we would have to recompute basic block information after
spilling. Another problematic point for some representations of basic blocks is
the fact, that liveness analysis needs to iterate over the blocks {\em
backwards}, as well as keep them in a worklist for fast data-flow analysis.

In our approach we chose to keep middle end basic blocks tied to machine blocks.
Control flow graph from the middle end representation is used even in the back
end, and merging of basic blocks is only done in the second round of peephole
optimization (after register allocation), which is not followed by any analysis
needing basic blocks. This works fairly well in that we don't introduce many
different or even temporary intermediate representations, but still allows
inter-block peephole optimization, though it has to be handled outside of the
main loop shown above, which is executed for each basic block. The separate
handling of inter-block optimizations looks roughly like this:

\begtt \optparams
if (<there is a following block>) {
	if (<inter-block pattern 1>) {
		<inter-block rule 1>
	}

	[...]

	if (<the following block is not referenced>) {
		<merge following block into the current one>
		inst = <first instruction of following block>;
		goto next;
	}
}
\endtt

%On the contrary, it would even suffice to have all the instructions in a
%function in single linked list of instructions with additional label
%instructions, instead of having each block separately, which then requires
%special handling of block boundaries.
%
%Outside of peephole optimizations not using basic blocks has undesirable side
%effects. Notably a lot of analysis that uses basic blocks and the control flow
%graph is no longer applicable. Such analysis is needed for example for liveness
%(for register allocation), but even for the flag analysis presented in one of
%the previous sections (section~\ref[sec:imp-peephole-flag]).

\secc Practical findings

All in all, the peephole optimization seems to do fairly well. Early deletions of
instructions are particularly important for getting good results out of peephole
optimizations. In particular, dead instruction removal should be done as soon as
possible, in the one pass, not for example left to the second pass. These
removals can bring closer instructions that would have otherwise not been
covered by the same peephole window.

There are a couple of disadvantages to our peephole optimization design:

\begitems

* Patterns are written by hand which is tedious.
* Only patterns noticed by a human are implemented.
* The potential of use of data-flow is not fully realized yet, because deletions
of constants far away from current instruction can still result in new
cascading improvements.

\enditems

\seccc Copy propagation

There are still interesting patterns that have been found and implemented by
hand. For example, the following works surprisingly well:

\begtt
if (IK(inst) == IK_MOV && IM(inst) == M_Cr && IK(prev) != IK_CMOVCC
		&& (IM(prev) == M_CI || IM(prev) == M_Cr || IM(prev) == M_CM)
		&& IREG(prev) == IREG2(inst) && use_cnt[IREG(prev)] == 1) {
	def_cnt[IREG(prev)]--;
	use_cnt[IREG(prev)]--;
	IREG(prev) = IREG(inst);
	prev->next = inst->next;
	inst->next->prev = prev;
	prev->next = inst->next;
	inst = prev;
	continue;
}
\endtt

The pattern essentially checks whether the physically first out of two
instructions (`prev`) writes into a register and the next instruction (`inst`)
copies the register to another register. Then, if the original register has only
one use (the second instruction) the copy is not necessary, and the move can be
realized directly. Other than the change in register, the rule deletes the
redundant copy, and marks the one remaining instruction as current for the next
investigation. Use and definition counts are updated accordingly.

This pattern cover surprising lot of cases, for example it can delete copies in
all of the following:

\begtt
lea t32, [rbp-16]
mov t14, t32

mov t27, 1
mov t18, t27
\endtt

As this optimization removes an instruction it allows more optimizations that
would otherwise be outside of reach. Despite being a simple form of copy
propagation it works so well, because due a lot of copies are present in
translation to x86-64's two address code.

\seccc Duplication of patterns

Even though use-def based optimization can propagate constants or memory
locations, they are currently only tried last. Less general local peephole
optimization patterns are tried first. For example, with the proper use and def
count checks the following pattern from:

\begtt
mov t22, [H]
mov t23, [...]
add t23, t22
\endtt

To:

\begtt
mov t23, ...
add t23, [H]
\endtt

Performs much better than the \"data-flow" version from
section~\ref[sec:imp-peephole-use-def] based on the `try_combine_memory`
function. This is purely because it is able to bring the peephole window back
to the first instruction, which cascades to many other optimizations.

Improvements to allow the use-def based optimizations to go back would be
necessary to prevent duplication in patterns, while keeping the quality of the
generated code. Implementing such improvement is more tricky than it might seem,
because there are performance as well as clarity concerns.

\seccc Avoiding dead ends

As mentioned in section~\ref[sec:imp-peephole-flag], we prefer the `lea`
instruction for arithmetic, because it allows subsequent optimizations to fold
more arithmetic into a single instruction. Again, these optimizations cascade
well, and we can even turn C code like this:

\begtt
int f(int *arr, int a, int b, int c) {
    return arr[c] = arr[a] + arr[b];
}
\endtt

Into the following (without the uninteresting prologue and epilogue):

\begtt
mov rax, [rdi+8*rsi]
add rax, [rdi+8*rdx]
mov [rdi+8*rcx], rax
\endtt

This stems from the great composability of the `lea` instruction. To not miss
these optimizations our peephole optimizer prefers `lea` even for situations
where other instructions might be better:

\begtt
lea t12, [t12+t13]

lea t12, [t12+1]
lea t12, [t12-1]
\endtt

The first instruction could be replaced by a single `add` instruction. On some
Intel microarchitectures~\cite[IntelOptimization], the `add` is recommended,
because for example low end processors use separate Address Generation Unit
(AGU) for `lea` calculations which means doing normal arithmetic there imposes
additional synchronization. The last two instruction are replaceable by `inc`
and `dec` respectively. These have their own problems with flags (see
section~\ref[sec:imp-peephole-flag]), but there are no big penalties on recent
processors.

As using `lea` is generally not worse in any significant way, we believe it is
okay to leave the situation as is.


\sec Register allocation

Register allocation by graph coloring and in particular the Iterated register
coalescing~\cite[George1996] algorithm we implement, does not consist of one
simple step, but of multiple, and additionally, the algorithm is repeated until
it succeeds. The basic means of operation of the algorithm are shown in
figure~\ref[fig:todo], and briefly introduced here:

\begitems \style n

* {\em Liveness analysis}. First we need to analyze where what variable are {\em
live}.

%Before we are able to construct the interference
%graph on which we perform graph coloring, we need to analyze what where
%variables are {\em live}.

* {\em Build interference graph}. From the liveness information we construct the
interference graph.

* {\em Calculate spill costs}. We calculate how costly would spill of each
virtual register be as well as mark some registers unspillable (see
section~\ref[sec:use-of-spilled]).

* {\em Perform iteration of iterated register coalescing graph coloring}. This
is the initialization and main loop of the~\cite[George1996] algorithm.

* {\em Assign registers}. Virtual registers are colored one by one according to
the order and coalescing determined in the previous step. The previous step
always succeeds. Virtual registers are at most marked as {\em potential spills}
and may still be assigned color. If no color is left for some of them in this
step, they are an {\em actual spills} and this step returns the list of
virtual registers to spill.

* {\em Rewrite the program}. If the previous step failed the program is
rewritten to include loads and stores of spilled virtual registers and all the
above steps are attempted again.

* {\em Apply register assignment}. If the graph coloring was successful (i.e.
there were no {\em actual spills}), then the program can be rewritten to change
occurrences of virtual register to their assigned physical registers.

\enditems

The specialty of the Iterated register coalescing algorithm lie mainly in step
4, but it has also great consequences of previous steps. Notably in design of
data structures.

The overall design of the register allocator in our compiler keeps it as a
separate, opaque component. We store all data needed for the register allocator
in a single struct called `RegAllocState`. The user of the register allocator
obtains the state with a call to a {\em create} function and frees the state with a
{\em free} function. Register allocation itself requires the allocated state and
a machine function for which to do the register allocation. The register
allocator returns the function modified to not use any virtual register, which
sometimes means that additional spill code is added. This API allows a single
state to be reused for allocating register for many functions and thus greatly
reducing the costs of memory allocation.

\begtt
RegAllocState *reg_alloc_state_create(Arena *arena);
void reg_alloc_state_free(RegAllocState *ras);

void reg_alloc_function(RegAllocState *ras, MFunction *mfunction);
\endtt

Most of the register allocator is target independent. Only x86-64 dependent
things are currently the routines for creating loads and stores from stack,
allocation of stack slots and the number of physical registers as well as the
descriptor tables (see section~\ref[sec:data-back-end] for more details about
the back end representation).

More information about the steps, as well as design and implementation
considerations are the subject of the following subsections.

\label[sec:imp-liveness]
\secc Liveness analysis

For liveness analysis we use the classic~\cite[Kam1977] approach of iterative
data-flow analysis:

\begitems \style n

* We construct data-flow equations.

* We solve them with an iterative algorithm.

\enditems

In liveness we propagate information in control flow from future to past, each
{\em use} means that a virtual register starts being {\em live} and each {\em
definition} means that a virtual register stops being {\em live}. Concrete
data-flow equations that capture this are given for example by~\cite[Appel1998]:

\def\w#1{{{\it #1}}}

$$ \eqalignno {
\w{in}[n]  &= \w{use}[n] \cup (\w{out}[n] - \w{de\kern-1.5ptf}\kern.5pt[n]) & (1) \cr
\w{out}[n] &= \bigcup_{s \in \w{succ}[n]} \w{in}[s]                         & (2) \cr
} $$

The formulation is based on {\em live-in} and {\em live-out} sets, which capture
the state of liveness along control flow edges. Liveness information in a
control flow nodes is given by the liveness in all {\em successors}, i.e. we
propagate information from the future to past.

Iterative data-flow analysis starts with all {\em live-in} and {\em live-out}
sets empty, and refines them until reaching a fixed point (i.e. the point where
all the sets stabilize and are equal to those in previous iteration). This can
be quite expensive and can require quite a lot of memory in a naive solution, so
in our implementation we perform a few optimizations:

\begitems

* Our control flow edges are {\em basic blocks}. We only keep liveness sets for
the edges to and from basic blocks, which requires much less memory than doing
so for each {\em instruction}. To compute basic block's live-out from the
live-in of all successors equation (2) suffices. To compute live-in of a
block from its live-out we need to iterate over the basic block {\em backwards}
and use equation (1) for each instruction.

* We don't store in memory all live-in and live-out sets, but only live-in sets of
all basic blocks. We can recompute live-out set for any basic blocks according
to equation (2).

* We use a {\em work list} based approach~\cite[Cooper2006] in which we first
insert all blocks into a work list, then as long as the work list is not empty,
we remove blocks from it, recompute its liveness sets and insert its {\em
predecessors} back to the work list. This means that we iterate over each block
once in the beginning, and then only on demand as we find changes in block {\em
successors}.

* Blocks use information from their {\em successors}, which means that if we
processed all blocks after their successors, we could find the solution in a
single iteration. In practice, control flow graphs contain loops, so processing
all successors of a block before the block is not possible, but we can use a
block order which tries to put block successors of a block before it, such as the
post order of the control flow graph, which we already have available (see
section~\ref[sec:data-structures-middle-end-function]).

\enditems

In the actual implementation this becomes the following:

\begtt
void liveness_analysis(RegAllocState *ras) {
	MFunction *mfunction = ras->mfunction;
	WorkList *live_set = &ras->live_set;

	wl_init_all_reverse(&ras->block_work_list, mfunction->mblock_cnt);
	Oper b;
	while (wl_take(&ras->block_work_list, &b)) {
		MBlock *mblock = mfunction->mblocks[b];
		Block *block = mblock->block;
		get_live_out(ras, block, live_set);

		for (Inst *inst = mblock->insts.prev; inst != &mblock->insts; inst = inst->prev)
			live_step(live_set, mfunction, inst);

		if (!wl_eq(live_set, &ras->live_in[b])) {
			WorkList tmp = ras->live_in[b];
			ras->live_in[b] = *live_set;
			*live_set = tmp;
			FOR_EACH_BLOCK_PRED(block, pred)
				wl_add(&ras->block_work_list, (*pred)->mblock->index);
		}
	}
}
\endtt

The same data structure (`WorkList`) is used for representing both the work list
for blocks that need to be processed as well as the {\em liveness
sets}---sets need to support fast unique addition as well as removal and
iteration and clearing, which is pretty similar to work list's needs. Our work list
implementation (section~\ref[sec:imp-worklist]) supports those operations all of
these use cases. Live-out sets are only computed when a block is processed,
`RegAllocState` only holds live-in sets:

\begtt
struct RegAllocState {
	[...]
	WorkList block_work_list;
	WorkList live_set;
	WorkList *live_in;
	[...]
}
\endtt

Block's indices correspond to their positions in `mfunction->mblocks`, which
lists them in {\em reverse postorder}. These indices are also used for
subscripting the `live_in` and other arrays. The arrays, sets and work lists in
`RegAllocState` are allocated with enough capacity for all blocks or all virtual
registers (depending on the use).

\begtt
void get_live_out(RegAllocState *ras, Block *block, WorkList *live_set) {
	wl_reset(live_set);
	FOR_EACH_BLOCK_SUCC(block, succ)
		wl_union(live_set, &ras->live_in[(*succ)->mblock->index]);
}
\endtt

The live step function mostly implements equation $(1)$ for an instruction:

\begtt
void live_step(WorkList *live_set, MFunction *mfunction, Inst *inst) {
	// Remove definitions from live.
	for_each_def(inst, remove_from_set, live_set);
	// Add uses to live.
	for_each_use(inst, add_to_set, live_set);
}
\endtt

Simple iteration over register uses and definitions was one of the motivations
for our representation for instructions (see section~\ref[sec:data-backend]) and
it works out nicely, since we can use generic callback-based iterators for
iterating over all definitions and uses.

\begtt
add t17, t15
\endtt

After the function `liveness_analysis` runs, `live_in` sets for each block are
calculated. From these we can derive the set of live virtual registers at any
program point by starting in the right basic block, computing live-out, and
iterating backwards to the program point of interest updating the live-set with
`live_step`, similarly as we do in the liveness analysis itself.

\secc Build interference graph

Now that we know which virtual registers are live at different program points,
we use them to construct the {\em interference graph}.

Formally, two virtual registers interfere when they are live at the same
time~\cite[Chaitin1981]. We can find the live-set for each program point by
iterating over all blocks (in any order) and instructions (backwards)
maintaining a live-set with the `live_step` function. But adding interference
between all live simultaneously would mean adding $\ell \cdot (\ell - 1)$ at each
program point---this is both expensive and adds too many edges repeatedly if
virtual registers are alive for longer periods of times.

Instead Chaitin~\cite[Chaitin1981] suggests to use a different notion of
interference: \"{\em two virtual registers interfere if one of them is live at
the definition point of other}". This translates much better to an
implementation, because this means at each instruction adding interference for
all definitions in that instruction with all $\ell$ members of live-set (thus for
most instruction this is $\ell$ edges, in general $c \cdot \ell$, where $c$ is a
constant).

The actual implementation of building the interference graph is very similar to
performing liveness analysis (section~\ref[sec:imp-liveness]), except that we
can iterate over each block just once, and call the `interference_step` function
(shown below) instead of `live_step`---apart from updating the liveness
instruction to instruction, we need to add the interference edges:

\begtt
void interference_step(RegAllocState *ras, WorkList *live_set, Inst *inst) {
	if (IK(inst) == IK_MOV && IS(inst) == MOV && IM(inst) == M_Cr) {
		for_each_use(inst, remove_from_set, live_set);
		add_move(ras, inst);
	}

	for_each_def(inst, add_to_set, live_set);

	FOR_EACH_WL_INDEX(live_set, j) {
		for_each_def(inst, add_interference_with, live_set->dense[j]);
	}

	for_each_def(inst, remove_from_set, live_set);
	for_each_use(inst, add_to_set, live_set);
}
\endtt

The most important part above is the loop which for each member of `live_set`
adds interference with each definition. But before doing that, we actually add
all {\em definitions} to the live set. By adding definitions to the live-set and
then adding interferences of all definitions with all live, we also add
interferences across all {\em defined registers}---important, to
prevent assignment of the same physical register to any two definitions in the
same instruction. Adding the definitions to the live set is not really correct
in the iteration, since we ought to add {\em uses}, not {\em definitions}. But
when stepping the liveness for an instruction, we actually remove definitions
from the live set, {\em before} adding the uses, so the fact that we add had
added the definitions to the live set temporarily is not observable from the
outside. The liveness step is realized by the last two calls, which are the same
as in `live_step`.

Before any of that, we also have special handling of copy instructions. For
example an instruction like

\begtt
mov t19, t20
\endtt
%
shouldn't add an artificial interference between the two virtual registers,
because then they {\em can't} be coalesced (recall
section~\ref[sec:regalloc-graph-coloring], we can only coalesce non-interfering
virtual registers). Other instructions can find that the virtual registers are
live at the same time at a {\em different program point}, but the copy
instruction itself shouldn't add the interference. This can be done nicely, by
removing the used register (i.e. `t20`) from the live-set---then it won't be
present when interferences of definitions and live-set members are added.

Last, but not least, we also call `add_move` function for every copy
instruction, we will come back to it when discussing the iterated register
coalescing algorithm in section~\ref[TODO].

\seccc Calling conventions

In sections~\ref[sec:regalloc-constraints, sec:regalloc-graph-coloring] we
claimed that the combination of live range splitting and interferences added
through right definitions and uses can be used to model calling conventions
passing arguments in physical registers, which is done on all commonly used
x86-64 calling conventions (see section~\ref[TODO]). In
section~\ref[sec:imp-lowering-prologue-epilogue] we described how the live range
splitting of callee saved registers is handled. But so far we haven't explained
how the interferences are added in our implementation, since the instruction
representation (see section~\ref[sec:data-backend]) only holds the registers
explicitly listed by the instruction and there is no space for any calling
convention related registers and their uses and definitions.

Before describing the implementation details, let's show on an example how
exactly we want to model uses and definitions on the following source program:
%
\begtt
int f(int c) {
	return g() + h() + c;
}
\endtt

After peephole optimization, right before register allocation:

\begtt
f:
.L0:
        ; entry
        push rbp
        mov rbp, rsp
        sub rsp, 42

        mov t28, rbx
        mov t29, r12
        mov t30, r13
        mov t31, r14
        mov t32, r15

        mov t18, rdi

        call g
        mov t21, rax

        mov rdi, t18
        call h
        lea t26, [t21+rax]
        add t26, t18
        mov rax, t26

        mov rbx, t28
        mov r12, t29
        mov r13, t30
        mov r14, t31
        mov r15, t32

        mov rsp, rbp
        pop rbp
        ret

\endtt

We see the classic prologue and epilogue creating and destroying the stack frame
and since we don't yet know the final stack space required (there might be
spills) a dummy amount is listed and will be fixed up much later. Then there are
the live range splits for the five callee saved registers (`rbx` and `r12`
through `r15`). There is also a live range split for the argument `c`, which is
passed in register `rdi`, but saved into `r18`. Another live range split is
present for the return value of function `g`, which is moved from `rax` to
`t21`. The argument for `h` is also split and passed in `rdi`. A copy of the
return value of `h` has actually been optimized out and `rax` is used directly
in the following addition (`lea` instruction). After both calls all three values
are added in two instructions and the result is moved to `rax`, the return value
register.

The `call` instruction has only one argument---the label of the function to
call. It doesn't in any way communicate explicitly, that the function call
caller saved registers (like `rdi` or `rsi` which are also used for passing
arguments) are not preserved throughout the call. There is also no explicit
definition of the callee saved and argument registers, even though we use them
in the function. As established mainly in
section~\ref[sec:regalloc-constraints]), we want to mainly add:

\begitems
* Uses of actually used {\em argument registers} to the `call` instruction.
Otherwise the definitions which assign parameters to registers would see no uses
and thus in the backwards computation of liveness they wouldn't ever be {\em
added} to the live-set. On the other hand, we need to add uses of only the
registers actually used for argument passing---otherwise they would be added to
the live-set unnecessarily and without any definitions---meaning a lot of
interferences would be added and use of the argument registers would effectively
be forbidden.

* Definitions of all {\em caller saved registers} to the `call` instruction.
Caller saved registers are not preserved by the function call, anything live
across the call has to either be saved in a callee saved register (because every
register is either caller or callee saved) or spilled to memory.

* Uses of all {\em callee saved registers} to the `ret` instruction. The caller
expects the callee saved registers to be preserved, thus modelling uses of them
in the `ret` instruction adds them to the live set for as long as they are not
defined. If we defined them in the entry to the function we would have
effectively added interferences of them with everything else in the function and
thus preserved them correctly, but not allowed them to be allocated.

But in our case, they are split, and their definitions appear right before the
epilogue and thus they are kept in the live set only for a short time. It is
still important that we model the uses in the `ret` instruction though---it
adds interferences to the registers holding the callee saved register values.
These interferences are important for graph coloring algorithm.

* Definitions of {\em argument and callee saved registers} in the entry point.
They are \"passed" by the caller. Adding the definitions isn't strictly needed
as the definitions won't prevent any additional interferences. But they assure
that all registers have at least one definition, which makes it consistent for
the purposes of peephole optimization (where we track the uses an definitions of
registers, see section~\ref[sec:imp-peephole-use-def]).

\enditems

Adding all the uses and definitions as additional {\em operands} to the `Inst`
structure (introduced in section~\ref[sec:data-backend]), is a bit wasteful
considering that to the same kinds of instruction we want to attach same kinds
of defined and used register, except sometimes we want to limit the number to
{\em argument count}.

Because of this regularity, we opted to attach the additional uses and
definitions to instruction {\em modes}. Already existing `M_LCALL` (call label)
`M_rCALL` (indirect call through register) as well as `M_RET` (return
instruction) can be extended with pointers to these additional definitions and
uses.

A bit more problematic is attaching definitions and uses to the entry
point---there is no explicit instruction there, and we don't want to handle
neither function entry nor exit points specially. In the end we chose to
introduce an additional instruction kind `IK_ENTRY` with a mode `M_ENTRY` whose
sole purpose is to hold the additional definitions and uses. The instruction is
otherwise mostly skipped as a no-op instruction. It is present even in the
listing above---it's text printout is \"`; entry`" (i.e. it is a comment).

Code-wise, the definition of the `ModeDescriptor` type is enhanced with a couple
of fields related to these additional definitions and uses:

\begtt
typedef struct {
	[...]
	bool use_cnt_given_by_arg_cnt;
	bool def_cnt_given_by_arg_cnt;
	Oper *extra_defs;
	Oper *extra_uses;
} ModeDescriptor;
\endtt

To avoid having to specify the number of physical registers in these lists, they
are terminated with the special register `R_NONE`. If the number of extra
definitions or uses is determined by the argument count (as is for the number of
either received arguments for the entry instruction or passed arguments in case
of call instruction), it is signified in the mode descriptor by a boolean flag,
and the count is read using the macro `IARG_CNT`, which is one of the operand
fields. These additional definitions and uses are also used for long division,
where `rax` and `rdx` are both read and written implicitly, while there is an
extra register or memory argument specifying the divisor, so listing `rax` and
`rdx` wouldn't fit into the compact representation of operands of x86-64
instructions.

In practice, the mode descriptors are extended like this:

\begtt
#define IARG_CNT(inst) ((inst)->ops[5])

Oper none[] = { R_NONE };

Oper rax_rdx[]       = { R_RAX, R_RDX, R_NONE };
Oper caller_saved[]  = { R_RAX, R_RCX, R_RDX, R_RSI, R_RDI,
                        R_8, R_9, R_10, R_11, R_NONE };
Oper argument_regs[] = { R_RDI, R_RSI, R_RDX, R_RCX, R_8, R_9, R_NONE };

ModeDescriptor mode_descs[] = {
	[M_Rr]    = { [...],  0, 0, none, none },
	[M_rCALL] = { [...],  1, 0, caller_saved, argument_regs },
	[M_ADr]   = { [...],  0, 0, rax_rdx, rax_rdx },
};
\endtt

And the iteration over all uses look like this:

\begtt
void
for_each_use(Inst *inst,
             void (*fun)(void *user_data, Oper *use),
             void *user_data)
{
	ModeDescriptor *mode = &mode_descs[inst->mode];
	for (size_t i = mode->use_start; i < mode->use_end; i++)
		fun(user_data, &inst->ops[i]);

	if (mode->use_cnt_given_by_arg_cnt) {
		size_t use_cnt = IARG_CNT(inst);
		for (size_t i = 0; i < use_cnt; i++)
			fun(user_data, &mode->extra_uses[i]);
	} else {
		for (Oper *use = mode->extra_uses; *use != R_NONE; use++)
			fun(user_data, use);
	}
}
\endtt

The callback-based iterator passes pointers (\"references") to the registers.
This is needed by same users of the functions, and can be used even by other
users of the iterators.

Greater flexibility could be achieved by not keeping the compact encoding and
instead representing the definitions and uses in a more expensive way. Though
the compact operand packing, which is with the fields very close to actual
x86-64 instructions is one of the reasons why peephole optimization can be done
so expressively even with hand-coded rules.

The approach is mostly target independent---each target can have different mode
descriptors. The macro `IARG_CNT` is x86-64 specific, but extending the
approach to have a per-target operand index for the argument count would be
straightforward.

With all the implementation details in place, our example above register
allocates and peephole optimizes successfully to the following:

\begtt
f:
.L0:
        push rbp
        mov rbp, rsp
        sub rsp, 16
        mov [rbp-16], rbx
        mov [rbp-8], r12
        mov r12, rdi
        call g
        mov rbx, rax
        mov rdi, r12
        call h
        lea rax, [rbx+rax]
        add rax, r12
        mov rbx, [rbp-16]
        mov r12, [rbp-8]
        mov rsp, rbp
        pop rbp
        ret
\endtt

The \"entry instruction" is optimized out, since it is no longer needed after
register allocation. Callee saved registers `rbx` and `r12` are used for values
live across call, and as their virtual registers got spilled, the former values
in callee saved registers got automatically saved in stack slots. Except for the
other callee saved registers coalescing didn't help much---the registers were
mostly restricted by the calling convention or interfered with each other due to
being live at the same time.

\seccc Interference graph representation

TODO: move this section after coalescing to justify \"online construction" of
both representations?

\secc Calculate spill costs

\secc Iterated register coalescing

A lot of other graph coloring register allocation approaches could reuse many
parts of our implementation. However the {\em main loop} which we describe in
this section is the main characteristic of the Iterated register coalescing
(\"IRC") algorithm by George and Appel~\cite[George1996], which we also describe
in context of other graph coloring register allocators in
section~\ref[sec:regalloc-graph-coloring]. As we will be working mainly on an
interference graph of virtual registers, we will also be calling virtual
registers {\em nodes} of the interference graph and the interferences {\em
edges}. The ideas and implementation of the algorithm in our implementation are
based also on pseudo code listed in~\cite[George1996], though we chose more
descriptive names, include clarifications and corrections to the algorithm.

The main loop interleaves {\em simplification} and {\em conservative
coalescing}. This leads to more coalescing, while still being as safe as earlier
approaches with regards to uncolorable graphs which mandate spills. The main
loop's purpose apart from the coalescing is mainly to push all the nodes on to a
stack, even the potential spills, because we use optimistic coloring. The actual
coloring of the graph is handled in a later stage, which is common to other
approaches like Briggs' or Chaitin's.

For a start, let's first consider the simplification phase of Briggs' algorithm,
which also does optimistic coloring, but contrary to IRC does all the coalescing
before starting with simplification. The simplifications are based on
Chaitin's heuristic, which says that every node with less than $k$
neighbours (where $k$ is the number of available registers) can be pushed onto
the stack and \"removed from the graph"---later when coloring the node will be
trivially colorable when popped from the stack and \"reinserted back into the
graph". Based on this heuristic we want to partition the nodes not yet pushed
onto the stack into two different categories:

\begitems \style n

* Low ($<k$) degree nodes. Sometimes also called \"insignificant" or \"trivially
colorable".

* High $(\geq k)$ degree nodes. Sometimes also called \"significant".

\enditems

Strict separation of these categories allows our efforts to concentrate on the
low degree nodes---those can be removed from the graph as they already fulfill
the condition imposed by Chaitin's heuristic. Of course, their removal
decreases degrees of neighboring nodes, and occasionally high degree node
becomes low degree and we need to move it from the category to the first.
Partitioning the nodes into two categories initially is easy. As long as the set
of low degree nodes is not empty, the algorithm can simplify the interference
graph and keep constructing the coloring order on the stack.

But when the set of low degree nodes becomes empty, and there are still nodes
left in the high degree set, then we need to choose a node to be a {\em
potential spill} and push it to the stack and remove it from the graph despite
having a high degree. This either allows other nodes to become low degree or
requires even more potential spills. Since the \"remove from graph and push onto
the stack" action is already performed on every node in the low degree set, we
can just move the potential spill to the low degree set and let it be handled by
the simplification mechanism. Consequently we can find more appropriate names
for the two sets:

\begitems
* {\em Simplify set}. Contains nodes intended for simplification.

* {\em Spill set}. Nodes that are currently spill candidates, because their
degree is high. They can move to the simplify set either when their degree
becomes low or they are chosen as potential spill.
\enditems

We can represent the two categories of nodes with a {\em work list}---we already
use work lists for representing other sparse sets (like the live-set in
section~\ref[sec:imp-liveness]), because our work list implementation
(section~\ref[sec:imp-worklist]) supports the set operations efficiently. Our
work list can even be used for the stack yielding the following main loop:


\begtt
while (true) {
	while (wl_take_back(&ras->simplify_wl, &i)) {
		simplify_one(ras, i);
	}
	if (wl_empty(&ras->spill_wl)) {
		break;
	}
	choose_and_spill_one(ras);
}
\endtt

The loop always tries to remove nodes from the simplify work list as long as
possible. If no node is available for simplification, a node from the spill work
list is spilled. If the spill work list is empty, it means that we have already
fallen-through the simplify loop and both work lists are empty, thus we can end
the main loop. If we for now ignore the choice of the best spilled candidate,
even the helper functions are straightforward:


\begtt \optparams
void simplify_one(RegAllocState *ras, Oper i) {
	wl_add(&ras->stack, i);
	for_each_adjacent(ras, i, decrement_degree);
}

void decrement_degree(RegAllocState *ras, Oper i) {
	if (ras->degree[i]-- == ras->reg_avail) {
		wl_remove(&ras->spill_wl, i);
		wl_add(&ras->simplify_wl, i);
	}
}

void choose_and_spill_one(RegAllocState *ras) {
	Oper candidate = <the best spill candidate>;
	wl_remove(&ras->spill_wl, candidate);
	wl_add(&ras->simplify_wl, candidate);
}
\endtt

Simplifications and spills move nodes from the spill work list to simplify work
list. In the case of simplifications, nodes are moved when their degree
decreases below $k$ (represented by `ras->reg_avail`). The simplified nodes
themselves are pushed onto a stack. The interference graph itself is not
modified in any explicit way, we just maintain a degree of each node. Though by
pushing a node onto a stack we remove it from the graph {\em implicitly}---until
the node is popped back, we have to presume it and its edges removed from the
graph.

The simplifications in the Iterated register coalescing are still based on the
same Chaitin heuristic, so the same simplification mechanism applies. But,
because we want to interleave simplifications with coalescing, we need to obey
one limitation: we can't coalesce any node pushed onto the stack, because it is
{\em removed} from the graph.

%Since we don't want to miss any coalescings, we collect 

To not miss any coalescing opportunities, we have to try all coalescings of a
node before pushing it (removing it from the graph). We only remove from the
graph nodes put on to the simplify work list, to prevent putting coalescable
nodes onto the simplify work list, we introduce a third category of nodes: low
degree move-related nodes (nodes that have a low degree, but are either a source
or destination in at least one copy instruction). We than add to the simplify
work list only low degree nodes that are not move-related, i.e. they can't be
coalesced. Though like with high degree nodes, if we can't find opportunities
for simplifications, we have to give up, and move something to the simplify work
list. Instead of choosing a node from the spill work list, it is preferable to
give up any coalescing of a move-related node. We will call giving up on
coalescing {\em freezing} and like with spills, separate nodes which we can
freeze into a separate work list:

\begitems
* {\em Simplify work list}. Contains nodes intended for simplification, i.e. removal
from graph and push onto the stack.

* {\em Freeze work list}. Nodes that are currently freeze candidates, because
they are move-related. They can move to the simplify work list either when they
stop being move-related (i.e. all their moves are either coalesced or given up
individually) or when they are chosen to be frozen (i.e. all their moves are
given up).

* {\em Spill work list}. Nodes that are currently spill candidates, because their
degree is high. They can move to the freeze work list if their degree becomes
low and they are move-related, or they can move to simplify work list if their
degree becomes low and they are not move-related. They can also move to the
simplify work list if they are chosen as potential spill.
\enditems

We also need to distinguish several kinds of moves:

\begitems

* Active moves. Moves that we want to actively check for coalescing.

* Inactive moves. Moves that we already checked for coalescing, but were
declined by the conservative heuristic. If there is a chance that a change might
make the conservative heuristic allow the coalescing, an inactive move should be
made active.

* Given up ({\em frozen}) moves. Moves that are no longer coalescable in any
capacity. Either because the source and destination interfere or we have given
up on them by freezing the source or destination.

\enditems

We also partition the moves into work lists. Representing frozen moves is
actually not needed, since they can be detected by not being present on the
other two work lists.

For the high level algorithm, we will revisit the Briggs' simplification. There
we would simplify in a loop, then fall-through to spill and if any spill was
made, we would go back to simplification, otherwise we would end the loop. This
can be also written like this in C:

\begtt
simplify:
while (wl_take_back(&ras->simplify_wl, &i)) {
	simplify_one(ras, i);
}
if (!wl_empty(&ras->spill_wl)) {
	choose_and_spill_one(ras);
	goto simplify;
}
\endtt

The falling-through is made implicit and instead the backward jumps are made
explicit through `goto`. With IRC we have more chances to add to simplify work
list other than spilling:

\begitems

* Combining two nodes through coalescing decreases degrees of nodes that were
interfering with both. This means that they can be added to the simplify work
list through `decrement_degree`.

* Before resorting to spilling, we want to freeze a node if any can be frozen.
Giving up moves (i.e. keeping move instructions) is better than spilling (i.e.
adding loads and stores).

\enditems

We can thus introduce processing of active moves and freezing, with more explicit
backward jumps to simplification:

\begtt
simplify:
while (wl_take_back(&ras->simplify_wl, &i)) {
	simplify_one(ras, i);
}
if (wl_take(&ras->active_moves_wl, &i)) {
	coalesce_move(ras, i);
	goto simplify;
}
if (wl_take_back(&ras->freeze_wl, &i)) {
	freeze_one(ras, i);
	goto simplify;
}
if (!wl_empty(&ras->spill_wl)) {
	choose_and_spill_one(ras);
	goto simplify;
}
\endtt

This code captures the idea of IRC very well: first simplify as much as
possible, then try coalescing, and go back to simplification. Since
full simplification (removal of all nodes from the graph) is our end goal, if
the simplification work list ever becomes empty with the graph still being
non-empty, we add even coalescable or high degree nodes to the simplify work
list.

The tricky part of the IRC algorithm is getting the {\em transitions} right.
Initially, putting the nodes and moves into right work lists is straightforward:

\begtt
void initialize_worklists(RegAllocState *ras) {
	wl_init_all(&ras->active_moves_wl, move_cnt);
	wl_reset(&ras->inactive_moves_wl);

	size_t vreg_cnt = ras->mfunction->vreg_cnt;
	for (size_t i = ras->first_vreg; i < vreg_cnt; i++) {
		if (is_significant(ras, i)) {
			wl_add(&ras->spill_wl, i);
		} else if (is_move_related(ras, i)) {
			wl_add(&ras->freeze_wl, i);
		} else {
			wl_add(&ras->simplify_wl, i);
		}
	}
}
\endtt

But any following change in degree or moves may require a transition of node or
move into a different work list. `choose_and_spill_one` which transitions a node
from spill work list to simplify work list can stay nearly the same, except it
needs to freeze moves of the potential spill.

%Iteration over all defined and used registers (which is needed by liveness
%analysis) or even just all registers.
%
%Even though most of the fields are common to all addressing modes.
%
%The problem with this representation is, that it allows invalid encodings to 
%
%Depending on the addressing modes, the operands themselves can be different
%things: immediate values, registers, memory locations. Though ultimately, they
%are all made up of just immediates and registers---memory locations are too
%given by expressions involving registers and immediates.
%
%Depending on the addressing modes, the operands can include:
%
%\begitems
%* register,
%* memory location,
%* 
%\enditems
%
%The operands themselves are either 
%
%The basic idea of the representation is, that all of these can b
%
%machine
%instructions are two things that constitute a machine instruction:
%
%
%
%
%
%However, the representation itself is {\em machine
%independent} and can actually be for different {\em instruction sets} if the
%compiler is ever extended.
%
%
%Our backend will be using the classical architecture TODO figure consisting of 
%
%Machine independent representation of machine dependent instructions

\chap Evaluation

Srovnání s GCC apod.

\chap Conclusion

\bibchap
\usebib/s (iso690) vlasami6-dip

\bye

This sequence simulates \"three address code" (see TODO) and the second instruction
is in fact what we started with to show use of spilled pseudoregisters.


As we have discussed in TODO, most instructions on
the x86-64 use the "two address code", where one of the operands is also the
destination for the result. Generally, three address code (where operations have
two operands and one destination that may or may not coincide with one of the
operands), are nicer from the perspective of the compiler---it is (at least in
principle, or while we are operating with pseudoregisters) non-destructive.





Coalescing vs live range splitting.



%The important takeaway is, that during register allocation by
%graph coloring, there are multiple potential causes for the graph becoming
%uncolorable, in any algorithm:
%
%\begitems
%* The graph needs more than $k$ colors to be colored.
%* The algorithm is too inexact to find a coloring.
%\enditems

needs to be noted that
more exact algorithm are necessarily able to find colorings even for graphs
which other algorithms would deem uncolorable. Briggs' approach is a 

The
interference graph has to change to 

\seccc Test

test




CHAITIN RECKLESS COALESCING

BRIGGS CONSERVATIVE COALESCING

GEORGE APPEL ITERATED REGISTER COALESCING



Since graph coloring can fail not only due to the graphs being uncolorable due
to interferences of precolored nodes, but also due to the graph having simply
needing more registers to be colored, we 

Since nodes corresponding to physical registers have a color associated with
them
Machine instruction constraints and calling conventions can be modelled bmachine constraints apply to {\em physical}
registers and by also adding them as nodes the interference graph, we can add
interferences between physical and virtual registers. This allows us to model
all usual constraints:

\begitems
* {\em Instructions allowing only specific input registers.} We 
\enditems

Graph coloring can fail not only because the graph's chromatic number is too
high, but also just because the particular graph coloring algorithm at hand is
{\em unable} to find a coloring.

so such
graphs (programs) need to be changed and graph coloring needs to be attempted
again. The changes






Zkratky:

DAG
JIT
SSA
RISC
CISC
ALU
CFG
IRC
